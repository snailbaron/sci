\documentclass[11pt]{article}
\usepackage[T2A]{fontenc}
\usepackage[utf8]{inputenc}
\usepackage[russian]{babel}
\usepackage{amsmath}
\usepackage{amssymb}

\oddsidemargin=0pt
\textwidth=6.5in
\topmargin=0pt
\headheight=0pt
\headsep=0pt
\textheight=9.5in

\newtheorem{theorem}{Теорема}
\newtheorem{lemma}{Лемма}

\title{Исследование разложимых КС-языков, имеющих вид <<цепочки>>}
\author{Игорь Мартынов}
\begin{document}

\setlength{\parindent}{0pt}
\setlength{\parskip}{8pt}

\maketitle

\section{Основные определения}

\textit{Стохастической КС-грамматикой} называется система $G = <V_T, V_N, R, s>$, где $V_T$ и $V_N$ --- алфавиты терминальных и нетерминальных символов соответственно, $s$ --- аксиома грамматики, $R$ --- множество правил вывода, представимое в виде $R = \cup_{i = 1}^k R_i$, где $k = |V_N|$, и $R_i$ --- множество правил вида
\begin{equation}
    r_{ij} : A_i \xrightarrow{p_{ij}} \beta_{ij}\quad{}\left( A_i \in V_N, \beta_{ij} \in (V_N \cup V_T)^* \right),
\end{equation}
и $p_{ij}$ --- вероятность применения правила $r_{ij}$, причём при фиксированном $i$ вероятности $r_{ij}$ задают вероятностное распределение на множестве $R_i$:
\begin{equation}
    0 < p_{ij} \leqslant 1\quad\text{и}\quad\sum_{j = 1}^{n_i} p_{ij} = 1,\qquad{}i = 1,2,\ldots,k,
\end{equation}
где $n_i = |R_i|$.

Слово $\beta$ называется \textit{непосредственно выводимым} из $\alpha$ (обозначается $\alpha \Rightarrow \beta$), если существуют $\alpha_1, \alpha_2 \in (V_N \cup V_N)^*$, для которых $\alpha = \alpha_1 A_i \alpha_2$, $\beta = \alpha_1 \beta_{ij} \alpha_2$ и в $R$ имеется правило $A_i \xrightarrow{p_{ij}} \beta_{ij}$.

Через $\Rightarrow_*$ обозначим рефлексивное транзитивное замыкание $\Rightarrow$. Если $\alpha \Rightarrow_* \beta$, говорят, что $\beta$ \textit{выводимо} из $\alpha$. Язык, \textit{порождаемый} грамматикой $G$ определяется как $L_G = \left\{ \alpha : s \Rightarrow_* \alpha, \alpha \in V_T^* \right\}$.

Последовательность правил грамматики $\omega(\alpha) = (r_1, r_2, \ldots, r_\gamma)$, последовательное применение которых к $s$ даёт слово $\alpha$, называется \textit{выводом} этого слова. Если на каждом шаге правило применяется к самому левому нетерминалу в слове, вывод назвается \textit{левым}.

Вероятность вывода определяется как $p(\omega(\alpha)) = p(r_1) \cdot p(r_2) \cdot \ldots \cdot p(r_\gamma)$, где $p(r_i)$ --- вероятность соответствующего правила. Вероятность слова определяется как сумма вероятностей всех его левых выводов.

Грамматика $G$ называется \textit{согласованной}, если
\begin{equation}
    \sum_{\alpha \in L_G} p(\alpha) = 1.
\end{equation}
Согласованная грамматика $G$ задаёт распределение вероятностей $P$ на $L_G$, и определяет \textit{стохастический КС-язык} $\mathfrak{L} = (L, P)$. В дальнейшем всюду предполагается, что грамматика согласованна.

По выводу слова может быть построено \textit{дерево вывода}. В корень дерева помещается аксиома $s$, далее на каждом ярусе дерева ко всем нетерминалам этого яруса применяется правило, соответствующее выводу. Символы этого слова записываются слева направо в дереве, присоединяясь к исходному нетерминалу как к родителю.

Обозначим $D_l^t$ --- множество деревьев вывода высоты $t-1$, порождаемых грамматикой $G$ при замене её аксиомы на $A_l$. Аналогично, $D_l^{\leqslant t}$ --- множество деревьев вывода, высота которых не превосходит $t - 1$.

Для исследования вероятностных характеристик стохастической грамматики применяются производящие функции
\begin{equation}
    F_i(s_1, s_2, \ldots, s_k) = \sum_{\substack{j = 1\\r_{ij} \in R}}^{n_i} p_{ij} s_1^{l_1} s_2^{l_2} \ldots s_k^{l_k},
\end{equation}
где $l_m = l_m(i,j)$ --- число вхождений нетерминала $A_m$ в $\beta_{ij}$.

Величины
\begin{equation}
    a^i_j = \left. \frac{\partial F_i(s_1, s_2, \ldots, s_k)}{\partial s_j} \right|_{s_1 = s_2 = \ldots = s_k = 1}
\end{equation}
называются \textit{первыми моментами} грамматики $G$. Матрица $A = (a^i_j)$, составленная из них, называется \textit{матрицей первых моментов} грамматики $G$.


Матрица $A$, по построению, неотрицательна. По теореме Фробениуса, доказанной в~\cite{gantmaher}, существует максимальный по модулю вещественный неотрицательный собственный корень $r$. Известно, что критерием согласованности стохастической КС-грамматики при отсутствии бесполезных нетерминалов является условие $r \leqslant 1$.

Говорят, что нетерминал $A_j$ \textit{непосредственно следует} за нетерминалом $A_i$ (обозначается $A_i \rightarrow A_j$), если в $R$ имеется правило $A_i \xrightarrow(p_{ij}) \alpha_1 A_j \alpha_2$, где $\alpha_1, \alpha_2 \in (V_N \cup V_T)^*$. Транзитивное замыкание отношения $\rightarrow$ обозначается $\rightarrow_*$. Если $A_i \rightarrow_* A_j$, говорят, что $A_j$ \textit{выводится} из $A_i$.

Введём также отношение $\leftrightarrow_*$. Будем считать, что $A_i \leftrightarrow_* A_j$, если одновременно $A_i \rightarrow_* A_j$ и $A_j \rightarrow_* A_i$, либо если $A_i = A_j$. Очевидно, отношение $\leftrightarrow_*$ есть отношение эквивалентности, и потому разбивает множество нетерминалов на классы $V_N = K_1 \cup K_2 \cup \ldots \cup K_m : K_i \cap K_j = \varnothing (i \neq j)$. Класс, содержащий ровно один нетерминал, будем называть \textit{особым}. Множество классов $\{K_1, K_2, \ldots, K_m\}$ обозначим $\mathfrak{K}$.

Если все нетерминалы грамматики образуют один класс, она называется \textit{неразложимой}. В противном случае она называется \textit{разложимой}. Очевидно, разложимой грамматике соответствует разложимая матрица первых моментов.

Говорят, что класс $K_j$ \textit{непосредственно следует} за классом $K_i$ (обозначается $K_i \prec K_j$), если существуют $A_1 \in K_i$ и $A_2 \in K_j$ такие, что $A_1 \rightarrow A_2$. Рефлексивное транзитивное замыкание $\prec$ обозначим $\prec_*$, и назовём отношением \textit{следования}.

Будем говорить, что грамматика имеет вид \textit{<<цепочки>>}, если она разложима, и граф, построенный на множестве $\mathfrak{K}$ по отношению $\prec$, имеет вид $P_m$. Пронумеруем классы грамматики таким образом, что $K_i \prec K_{i+1}, i = 1,2,\ldots,m-1$. Пронумеруем нетерминалы так, что для любых $A_i \in K_p$ и $A_j \in K_q$ $i < j \Leftrightarrow p < q$. После этого матрица первых моментов грамматики приобретает вид:
\begin{equation}
    A = 
    \begin{pmatrix}
        A_{11} & A_{12} & 0 & \cdots & 0 & 0 \\
        0 & A_{22} & A_{23} & \cdots & 0 & 0 \\
        \vdots & \vdots & \vdots & \ddots & \vdots & \vdots \\
        0 & 0 & 0 & \cdots & A_{m-1,m-1} & A_{m-1,m} \\
        0 & 0 & 0 & \cdots & 0 & A_{m,m}
    \end{pmatrix}
\end{equation}

Блоки $A_{i,i} (i = 1,2,\ldots,m)$ являются неразложимыми неотрицательными матрицами. Не уменьшая общности, будем считать их положительными и непериодичными. Этого можно добиться с помощью метода укрупнения правил грамматики. Пусть $r_i$ --- перронов корень матрицы $A_{i,i}$. По построению матрицы $A$, $r = \max_i \{r_i\}$ и $r > 0$.

\section{Свойства матрицы первых моментов}

Обозначим $J = \{ i : r_i = r \} = \{ i_1 < i_2 < \ldots < i_q \}$ Разобьём множество классов $\mathfrak{K}$ на группы классов $\mathfrak{M_1}, \mathfrak{M_2}, \ldots, \mathfrak{M_w}$. При этом $\mathfrak{M_1} = \{K_1, K_2, \ldots, K_{i_1} \}$, и $\mathfrak{M_l} = \{ K_{i_{l-1} + 1}, \ldots, K_{i_l} \}$, где $l > 1$. При таком разбиении в каждой группе $\mathfrak{M_j}$ содержится ровно один класс с номером из $J$.

Тогда матрицу первых моментов можно представить в виде
\begin{equation}
    A = 
    \begin{pmatrix}
        B_{11} & B_{12} & \cdots & 0 & 0 \\
        0 & B_{22} & \cdots & 0 & 0 \\
        \vdots & \vdots & \ddots & \vdots & \vdots \\
        0 & 0 & \cdots & B_{w-1,w-1} & B_{w-1,w} \\
        0 & 0 & \cdots & 0 & B_{w,w}
    \end{pmatrix},
\end{equation}
где $B_{ij}$ --- блок, находящийся на пересечении строк, соответствующих нетерминалам классов группы $\mathfrak{M_i}$, и столбцов, соответствующим нетерминалам классов группы $\mathfrak{M_j}$. Очевидно, каждой из матриц $B_{i,i}$ соответствует перронов корень равный $r$.

Рассмотрим матрицу
\begin{equation}
    A^t = 
    \begin{pmatrix}
        B_{11}^t & B_{12}^{(t)} & \cdots & B_{1,w-1}^{(t)} & B_{1,w}^{(t)} \\
        0 & B_{22}^t & \cdots & B_{2,w-1}^{(t)} & B_{2,w}^{(t)} \\
        \vdots & \vdots & \ddots & \vdots & \vdots \\
        0 & 0 & \cdots & B_{w-1,w-1}^t & B_{w-1,w}^{(t)} \\
        0 & 0 & \cdots & 0 & B_{w,w}^t
    \end{pmatrix}.
\end{equation}
Для установления её вида требуется определить вид блоков $B_{i,j}^{(t)}$ при $j > i$.

Рассмотрим блок $B_{11}$. Разобьём группу $\mathfrak{M_1}$ на подгруппы $(\mathfrak{M_{11}}, \mathfrak{M_{12}}, \mathfrak{M_{13}})$. К группе $\mathfrak{M_{12}}$ отнесём класс с номером из $J$, к группе $\mathfrak{M_{11}}$ --- предшествующие ему классы, к группе $\mathfrak{M_{13}}$ --- последующие классы. В соответствии с таким разбиением $B_{11}$ принимает вид
\begin{equation}
    B = 
    \begin{pmatrix}
        C_{11} & C_{12} & 0 \\
        0 & C_{22} & C_{23} \\
        0 & 0 & C_{33}
    \end{pmatrix},
\end{equation}
А матрица $B^t$ представляется в виде
\begin{equation}
    B^t = 
    \begin{pmatrix}
        C_{11}^t & C_{12}^{(t)} & C_{13}^{(t)} \\
        0 & C_{22}^t & C_{23}^{(t)} \\
        0 & 0 & C_{33}^t
    \end{pmatrix}.
\end{equation}

Известно, что для неразложимой положительной матрицы $A$
\begin{equation}
    A^t = u v r^t (1 + o(1)),
\end{equation}
где $r$ --- перронов корень $A$, $u$ и $v$ --- соответственно правый и левый собственные векторы, соответствующие $r$, причём $u > 0$, $v > 0$, $vu = 1$.

Таким образом, асимптотика матриц $C_{11}^t$, $C_{22}^t$, $C_{33}^t$ известна.

Исследуем собственные векторы матрицы $B_{11}$, соответствующие числу $r$. Пусть $u = (u^{(1)}, u^{(2)}, u^{(3)})$ и $v = (v^{(1)}, v^{(2)}, v^{(3)})$ --- соответственно правый и левый такие собственные векторы. Тогда
\begin{equation}
    \begin{split}
        &C_{11} u^{(1)} + C_{12} u^{(2)} + C_{13} u^{(3)} = r u^{(1)} \\
        &C_{22} u^{(2)} + C_{23} u^{(3)} = r u^{(2)} \\
        &C_{33} u^{(3)} = r u^{(3)}
    \end{split}.
\end{equation}
Поскольку все собственные числа $C_{33}$ строго меньше $r$, $u^{(3)} = 0$ и $u^{(2)}$ --- правый собственный вектор $C_{22}$, относящийся к $r$, а $u^{(1)} = (rE - C_{11})^{-1} C_{12} u^{(2)}$.

Аналогично, рассматривая левый собственный вектор $v = (v^{(1)}, v^{(2)}, v^{(3)})$, имеем систему
\begin{equation}
    \begin{split}
        &v^{(1)} C_{11} = r v^{(1)} \\
        &v^{(1)} C_{12} + v^{(2)} C_{22} = r v^{(2)} \\
        &v^{(1)} C_{13} + v^{(2)} C_{23} + v^{(3)} C_{33} = r v^{(3)}
    \end{split},
\end{equation}
откуда $v^{(1)} = 0$, $v^{(2)}$ --- левый собственный вектор $C_{12}$, и $v^{(3)} = v^{(2)} C_{23} (rE - C_{33})^{-1}$.

Выберем именно такие $u$ и $v$, что $vu = 1$.

Рассмотрим асимптотику матрицы $B_{11}^t$. Нетрудно видеть, что
\begin{equation}
    C_{12}^{(t)} = \sum_{i + j = t-1} C_{11}^i C_{12} C_{22}^j.
\end{equation}
Разобьём эту сумму так, что $C_{12}^{(t)} = \Sigma_1 + \Sigma_2$, где
\begin{equation}
    \begin{split}
        &\Sigma_1 : \left\{
        \begin{split}
            &i = \overline{ t - \lfloor \log \log t \rfloor, t-1 } \\
            &j = \overline{ 0, \lfloor \log \log t \rfloor - 1 }
        \end{split} \right. \\
        &\Sigma_2 : \left\{ 
        \begin{split}
            &i = \overline{ 0, t - \lfloor \log \log t \rfloor - 2 } \\
            &j = \overline{ \lfloor \log \log t \rfloor + 1, t-1 }
        \end{split} \right.
    \end{split}
\end{equation}

Рассмотрим вначале $\Sigma_1$. $C_{22}^t = u_2 v_2 r^t (1 + o(1)) \leqslant c_2$ при любых $t$. $C_{11}^t = u_1 v_1 (r')^t (1 + o(1)) \leqslant c_1 (r')^{t - \lfloor \log \log t \rfloor - 1}$. Отсюда,
\begin{equation}
    \Sigma_1 \leqslant с (r')^{t - \lfloor \log \log t \rfloor} \lfloor \log \log t \rfloor = O(\log \log t (\log t)^{c_3} (r')^t) = o(r^t).
\end{equation}

Для $\Sigma_2$ $C_{22}^j = u_2 v_2 r^j (1 + o(1))$, поэтому $\Sigma_2 = \sum_{i + j = t-1} C_{11}^i C_{12} H r^j (1 + o(1)) = r^{t - 1} \sum_{i = 0}^{t - \lfloor \log \log t \rfloor - 1} \left(\frac{C_{11}}{r}\right)^i C_{12} H (1 + o(1))$. Нетрудно видеть, что $\frac{1}{r} \sum_{i = 0}^{\infty} \left(\frac{C_{11}}{r}\right)^i = (rE - C_{11})^{-1}$. Матрица $(rE - C_{11})^{-1}$ существует, так как все собственные числа $C_{11}$ строго меньше $r$. Отсюда
\begin{equation}
    \Sigma_2 = r^t (rE - C_{11})^{-1} C_{12} u^{(2)} v^{(2)} (1 + o(1)) = C_{12}^{(t)}.
\end{equation}

Аналогично
\begin{equation}
    C_{23}^{(t)} = \sum_{i + j = t-1} C_{22}^i C_{23} C_{33}^j,
\end{equation}
откуда, проводя аналогичные вычисления, имеем
\begin{equation}
    C_{23}^{(t)} = r^t u^{(2)} v^{(2)} C_{23} (rE - C_{33})^{-1} (1 + o(1))
\end{equation}

Подстановкой проверяется, что
\begin{equation}
    C_{13}^{(t)} = \sum_{i + j = t-1} C_{12}^{(i)} C_{23} C_{33}^j
\end{equation}
Подставляя в это выражение асимптотику для $C_{12}^{(t)}$, получаем:
\begin{equation}
    C_{13}^{(t)} = r^t u^{(1)} v^{(2)} C_{23} (rE - C_{33})^{-1} (1 + o(1)) = r^t u^{(1)} v^{(3)} (1 + o(1))
\end{equation}

В результате, имеем:
\begin{equation}
    B_{11} = 
    \begin{pmatrix}
        0 & u^{(1)} v^{(2)} & u^{(1)} v^{(3)} \\
        0 & u^{(2)} v^{(2)} & u^{(2)} v^{(3)} \\
        0 & 0 & 0
    \end{pmatrix}
    r^t + o(r^t)
\end{equation}

Получим теперь асимптотику всей матрицы $A^t$. Вначале пусть $w = 2$. Тогда
\begin{equation}
    A^t = 
    \begin{pmatrix}
        B_{11}^t & B_{12}^{(t)} \\
        0 & B_{22}^t
    \end{pmatrix}
\end{equation}

Матрица $B_{22}^t$ исследуется аналогично $B_{11}^t$, в результате имеем
\begin{equation}
    B_{22}^t = 
    \begin{pmatrix}
        u^{(22)} v^{(22)} & u^{(22)} v^{(32)} \\
        0 & 0
    \end{pmatrix}
    r^t + o(r^t)
\end{equation}

Для $B_{12}^{(t)}$ имеем
\begin{equation}
    B_{12}^{(t)} = \sum_{i + j = t-1} B_{11}^i B_{12} B_{22}^j
\end{equation}
Подставляя выражения для $B_{11}^i$ и $B_{22}^j$, получаем:
\begin{equation}
    B_{12}^{(t)} = 
    \begin{pmatrix}
        0 & u^{(1)} v^{(2)} & u^{(1)} v^{(3)} \\
        0 & u^{(2)} v^{(2)} & u^{(2)} v^{(3)} \\
        0 & 0 & 0
    \end{pmatrix}
    \cdot B_{12} \cdot
    \begin{pmatrix}
        u^{(22)} v^{(22)} & u^{(22)} v^{(32)} \\
        0 & 0
    \end{pmatrix}
    \cdot t r^t + o(t r^t)
\end{equation}

Представляя $B_{12}$ в блочном виде
\begin{equation}
    B_{12} = 
    \begin{pmatrix}
        D_{11} & D_{12} \\
        D_{21} & D_{22} \\
        D_{31} & D_{32}
    \end{pmatrix},
\end{equation}
и производя перемножение, получаем
\begin{equation}
    B_{12}^{(t)} = 
    \begin{pmatrix}
        u'^{(11)} v^{(22)} & u'^{(11)} v^{(32)} \\
        u'^{(21)} v^{(22)} & u'^{(21)} v^{(32)} \\
        0 & 0
    \end{pmatrix}.
\end{equation}

Сформулируем теорему, определяющую вид блока $B_{lh}^{(t)}$ в общем случае.
\begin{theorem}
    \begin{equation}
        B_{lh}^{(t)} = u^{(l)} v^{(h)} t^{s_{lh} - 1} r^t (1 + o(1)),
    \end{equation}
    при $t \rightarrow \infty$, где $u^{(l)}$ и $v^{(h)}$ не зависят от $t$, и $s_{lh}$ --- число классов с номерами из $J$ среди $K_l, K_{l+1}, \ldots, K_h$.
\end{theorem}

\textbf{Доказательство.} Доказательство проведём индукцией по $w$. При $w = 2$ теорема выполняется.

Пусть утверждение теоремы верно для $w-1$ групп. Тогда
\begin{equation}
    A = 
    \begin{pmatrix}
        D_1 & E_1 \\
        0 & B_{w,w}
    \end{pmatrix}
    =
    \begin{pmatrix}
        B_{11} & E_2 \\
        0 & D_2
    \end{pmatrix},
\end{equation}
где
\begin{equation}
    E_1 = 
    \begin{pmatrix}
        B_{1,w} \\
        \vdots \\
        B_{w-1,w}
    \end{pmatrix},
    \quad{}E_2 = 
    \begin{pmatrix}
        B_{12} & \cdots & B_{1,w}
    \end{pmatrix}
\end{equation}

Тогда
\begin{equation}
    A^t = 
    \begin{pmatrix}
        D_1^t & E_1^{(t)} \\
        0 & B_{w,w}^t
    \end{pmatrix}
    =
    \begin{pmatrix}
        B_{11}^t & E_2^{(t)} \\
        0 & D_2^t
    \end{pmatrix}
\end{equation}

Для матриц $D_1$, $D_2$ утверждение теоремы по индукции справедливо. Для доказательства теоремы достаточно рассмотреть
\begin{equation}
    B_{1,w}^{(t)} = \sum_{l = 1}^{w-1} \sum_{i + j = t-1} B_{1,l}^{(i)} B_{l,w} B_{w,w}^j
\end{equation}
По предположению индукции слагаемое, содержащее $B_{1,w-1}^i = O(i^{s_{1,w-1} - 1} r^i)$, преобладает над остальными. Поэтому
\begin{equation}
    B_{1,w}^{(t)} = \sum_{i + j = t-1} B_{1,w-1}^{(i)} B_{w-1,w} B_{w,w}^j (1 + o(1))
\end{equation}

Подставляя по индукции выражение для $B_{1,w-1}$, получаем:
\begin{equation}
    B_{1,w}^{(t)} = 
    \begin{pmatrix}
        u'^{(11)} v^{(2w)} & u'^{(11)} v^{(3w)} \\
        u'^{(21)} v^{(2w)} & u'^{(21)} v^{(3w)} \\
        0 & 0
    \end{pmatrix}
    t^{s_{1,w} - 1} r^t + o(t^{s_{1,w}-1} r^t)
\end{equation}

\section{Вероятности продолжения}

...Здесь про вероятности продолжения...

Результат:
\begin{equation}
    \begin{split}
        &Q_n(t) = c_\mu u^{(\mu)}_{n - k_{\mu-1}} t^{-\left( \frac{1}{2} \right)^{m - \mu}} \\
        &P_n(t) = d_\mu u^{(\mu)}_{n - k_{\mu-1}} t^{-1 -\left( \frac{1}{2} \right)^{m - \mu}}
    \end{split},
\end{equation}
где $n \in I_\mu$, $c_\mu$ и $d_\mu$ заданы.

\section{Математические ожидания числа применений правила в деревьях вывода}

Обозначим через $q^l_{ij}(t,\tau)$ и $\bar{q}^l_{ij}(t,\tau)$ случайные величины, равные числу применений правила $r_{ij}$ в дереве вывода, соответственно, из $D_l^t$ и $D_l^{\leqslant t}$. Пусть также
\begin{equation}
    \begin{split}
        &S^l_{ij}(t) = \sum_{\tau = 1}^{t-1} q^l_{ij}(t,\tau) \\
        &\bar{S}^l_{ij}(t) = \sum_{\tau = 1}^{t-1} \bar{q}^l_{ij}(t,\tau)
    \end{split},
\end{equation}
и $S^l_{ij}(t)$, $\bar{S}^l_{ij}$ --- соответственно число применений правила $r_{ij}$ в дереве из $D_l^t$, $D_l^{\leqslant t}$. Для удобства записи положим
\begin{equation}
    \begin{split}
        S_{ij}(t) = S^1_{ij}(t),&\quad{}\bar{S}_{ij}(t) = \bar{S}^1_{ij}(t) \\
        q_{ij}(t,\tau) = q^1_{ij}(t,\tau),&\quad{}\bar{q}_{ij}(t,\tau) = \bar{q}^1_{ij}(t,\tau)
    \end{split}.
\end{equation}

Рассмотрим математические ожидания некоторых из введённых величин. Обозначим
\begin{equation}
    M^l_{ij}(t) = M[ S^l_{ij}(t) ],\quad{}\bar{M}^l_{ij}(t) = M[ \bar{S}^l_{ij}(t) ].
\end{equation}

Задача данного раздела заключается в вычислении $\bar{M}^l_{ij}(t)$, $M^l_{ij}(t)$ для грамматик в виде <<цепочки>>. Для их нахождения будет удобно использовать три леммы, доказанные в \cite{borisov-disser}.

\begin{lemma}
\label{l:powers}
    Пусть $s,d$ --- натуральные, $m = (m_1, \ldots, m_s)$ --- вектор целых неотрицательных чисел, $y = (y_1, \ldots, y_s)$ --- вектор, и $\bar{m} = \sum_{j = 1}^s m_j$. Тогда
    \begin{equation}
        (1 - y_1)^{n_1} \ldots (1 - y_s)^{n_s} = \sum_{\substack{\bar{m} < d \\ m \geqslant 0}} \binom{n_1}{m_1} \binom{n_2}{m_2} \ldots \binom{n_s}{m_s} (-1)^{\bar{m}} y^m + R_d(n_1, \ldots, n_s, y),
    \end{equation}
    где $y^m = y_1^{m_1} \ldots y_s^{m_s}$, и остаточный член представим в виде
    \begin{equation}
        R_d(n_1, \ldots, n_s, y) = \sum_{\substack{\bar{m} = d \\ m \geqslant 0}} (-1)^d \varepsilon_m(n_1, \ldots, n_s, y) y^m,
    \end{equation}
    причём
    \begin{equation}
        0 \leqslant \varepsilon_m(n_1, \ldots, n_s, y') \leqslant \varepsilon_m(n_1, \ldots, n_s, y) \leqslant \binom{n_1}{m_1} \ldots \binom{n_s}{m_s}
    \end{equation}
    при $0 \leqslant y_i \leqslant y_i' \leqslant 1\quad{}(i = 1,\ldots,s)$.
\end{lemma}

\begin{lemma}
\label{l:x_b_a}
    Пусть $A(t)$ --- последовательность матриц размером $k \times k$, и $A(t) \rightarrow A$ при $t \rightarrow \infty$, причём $A > 0$, и её перронов корень $r = 1$. Пусть $b(t) = b t^{\alpha} (1 + o(1))$ --- последовательность векторов длины $k$, где $b \geqslant 0$, $b \neq 0$, и $\alpha$ --- действительное число. Тогда для последовательности векторов $x(t)$ при $t = 1,2,\ldots$, определяемой рекуррентным соотношением $x(t) = b(t) + A(t) x(t-1)$ при $t \rightarrow \infty$ справедливо соотношение
    \begin{equation}
        \frac{ x_i(t) }{ v x(t) } \rightarrow u_i,
    \end{equation}
    при условии что $x(t_0) > 0$ для некоторого номера $t_0$, где $u,v > 0$ --- соответственно правый и левый собственные векторы матрицы $A$ при нормировке $vu = 1$.
\end{lemma}

\begin{lemma}
\label{l:x_rec}
    Пусть последовательность ${x_t}$, $x_t > 0$ при любом $t \geqslant 0$, удовлетворяет рекуррентному соотношению
    \begin{equation}
        x_{t+1} = a t^{\alpha} (1 + \varepsilon_1(t)) + (1 - b t^{\beta} (1 + \varepsilon_2(t))) x_t,
    \end{equation}
    где $\beta < 0$, $b > 0$, и $\varepsilon_1(t), \varepsilon_2(t) = o(1)$ при $t \rightarrow \infty$. Тогда верны следующие асимптотические равенства:
    \begin{equation}
        \begin{split}
            &(1)\;x_t = \frac{a t^{\alpha + 1}}{\alpha + 1} (1 + o(1))\quad\text{при}\quad\beta < -1,\;\alpha \geqslant 0 \\
            &(2)\;x_t = \frac{a t^{\alpha + 1}}{\alpha + b + 1} (1 + o(1))\quad\text{при}\quad\beta = -1,\;\alpha > -1 \\
            &(3)\;x_t = \frac{a t^{\alpha - \beta}}{b} (1 + o(1))\quad\text{при}\quad{}-1 < \beta < 0
        \end{split}
    \end{equation}
\end{lemma}

Вначале рассмотрим $\bar{M}^q_{ij}(t)$. Пусть $p(\cdot)$ --- условная вероятность дерева $d$ в грамматике $G$, при условии что $d \in D_q^{\leqslant t}$. Рассмотрим множество $D_{ql}^{\leqslant t}$ деревьев из $D_q^{\leqslant t}$, первый ярус которых получен применением правила $r_{ql}$ к корню дерева. Пусть
\begin{equation}
    \bar{P}_{ql}^{ij}(t) = \sum_{d \in D_{ql}^{\leqslant t}} p(d) q_{ij}(d),
\end{equation}
где $q_{ij}(d)$ --- число применений правила $r_{ij}$ в дереве $d$, и $\bar{P}_{ql}^{ij}(t)$ --- вклад деревьев из $D_{ql}^{\leqslant t}$ в матожидание $\bar{M}^q_{ij}(t)$. Для краткости, обозначим $\bar{P}_{ql} = \bar{P}_{ql}^{ij}$. Тогда, очевидно,
\begin{equation}
    \bar{M}^q_{ij}(t) = \sum_{l = 1}^{n_q} \bar{P}_{ql}(t).
\end{equation}

Рассмотрим величину $\bar{P}_{ql}(t)$. Пусть
\begin{equation}
    q_{ij}(d) = q^{(1)}_{ij}(d) + q^{(2)}_{ij}(d),
\end{equation}
где $q^{(1)}_{ij}(d)$ --- число применений правила $r_{ql}$ в дереве $d$ на первом его ярусе, а $q^{(2)}_{ij}(d)$ --- на остальных ярусах. Тогда
\begin{equation}
\label{eq:bar_p_12}
    \bar{P}_{ql}(t) = \sum_{d \in D_{ql}^{\leqslant t}} p(d) q_{ij}(d) = \sum_{d \in D_{ql}^{\leqslant t}} p(d) q^{(1)}_{ij}(d) + \sum_{d \in D_{ql}^{\leqslant t}} p(d) q^{(2)}_{ij}(d) = \bar{P}^{(1)}_{ql}(t) + \bar{P}^{(2)}_{ql}(t)
\end{equation}

Очевидно, $q^{(1)}_{ij}(d) = \delta^q_i \delta^l_j$ (где $delta$ --- символ Кронекера), и следовательно, учитывая что $p(\cdot)$ --- условные вероятности, получаем
\begin{equation}
    \bar{P}^{(1)}_{ql}(t) = \delta^q_i \delta^l_j \frac{ p_{ij} Q_{s_{ij}}(t-1) }{ 1 - Q_q(t) },
\end{equation}
где $Q_X(t)$ --- вероятность наборов деревьев вывода высоты не превосходящей $t-1$, набор корней которых задан характеристическим вектором $X \in \mathbb{N}^k$.

Обозначим также $\delta^i(n) = \left. ( \delta^i_k ) \right|_{i = \overline{1,n}} \in \{0,1\}^n$.

Условную вероятность дерева $p(d)$ при $d \in D_{ql}^{\leqslant t}$ можно выразить как
\begin{equation}
    p(d) = \frac{ p_{ql} }{ 1 - Q_q(t) } p_1(d) p_2(d) \ldots p_{\bar{s}_{ql}}(d),
\end{equation}
где $p_j(d)$ --- вероятность поддерева $d$ с корнем в $j$-м узле первого яруса. Тогда
\begin{equation}
\label{eq:bar_P_2}
    \bar{P}^{(2)}_{ql}(t) = \frac{p_{ql}}{1 - Q_q(t)} \sum_{d \in D_{ql}^{\leqslant t}} \prod_{n = 1}^{\bar{s}_{ql}} p_n(d) \sum_{m = 1}^{\bar{s}_{ql}} q'^m_{ij}(d),
\end{equation}
где $q'^m_{ij}(d)$ --- число применений правила $r_{ij}$ в поддереве дерева $d$ с корнем в $m$-том нетерминале первого яруса.

Выделим в $d$ поддеревья $d_1, d_2, \ldots, d_{\bar{s}_{ql}}$, где $d_j$ --- поддерево с корнем в $j$-м узле первого яруса $d$. Преобразуя $(\ref{eq:bar_P_2})$, получаем
\begin{multline}
    \bar{P}^{(2)}_{ql}(t) = \frac{p_{ql}}{1 - Q_q(t)} \sum_{m = 1}^{\bar{s}_{ql}} \sum_{d \in D_{ql}^{\leqslant t}} \left( \prod_{n = 1}^{\bar{s}_{ql}} p_n(d) \right) q'^m_{ij}(d) = \\
    = \frac{p_{ql}}{1 - Q_q(t)} \sum_{m = 1}^{\bar{s}_{ql}} \sum_{ d_1, \ldots, d_{m-1}, d_{m+1}, \ldots, d_{\bar{s}_{ql}} } p_1(d_1) \ldots p_{m-1}(d_{m-1}) p_{m+1}(d_{m+1}) \ldots p_{\bar{s}_{ql}}(d_{\bar{s}_{ql}}) q'^m_{ij}(d) = \\
    \frac{p_{ql}}{1 - Q_q(t)} \sum_{m = 1}^{\bar{s}_{ql}} Q_{s_{ql} - \delta^m} q_{ij}(d_m) = \frac{p_{ql}}{1 - Q_q(t)} \sum_{m = 1}^k s_{ql}^m \bar{M}^m_{ij}(t-1) Q_{s_{ql} - \delta^m}(t-1)
\end{multline}

Зная $\bar{P}_{ql}(t) = \bar{P}^{(1)}_{ql}(t) + \bar{P}^{(2)}_{ql}(t)$, получаем
\begin{equation}
    \bar{M}^q_{ij}(t) = \frac{1}{1 - Q_q(t)} \left[ \delta^q_i p_{ij} Q_{s_{ij}}(t-1) + \sum_{l = 1}^{n_q} p_{ql} \sum_{m = 1}^k s_{ql}^m \bar{M}^m_{ij}(t-1) Q_{s_{ql} - \delta^m}(t-1) \right]
\end{equation}

Обозначая
\begin{equation}
\label{eq:bar_m_ap}
    \bar{M}'^q_{ij}(t) = \bar{M}^q_{ij}(t) (1 - Q_q(t)),
\end{equation}
имеем
\begin{equation}
\label{eq:bar_m_recur}
    \bar{M}'^q_{ij}(t) = \delta^q_i p_{ij} Q_{s_{ij}}(t-1) + \sum_{l = 1}^{n_q} p_{ql} \sum_{m = 1}^k s_{ql}^m \bar{M}'^m_{ij}(t-1) Q_{s_{ql} - \delta^m}(t-1)
\end{equation}

Рекуррентное соотношение $(\ref{eq:bar_m_recur})$ является опорной точкой для вычисления $\bar{M}^q_{ij}(t)$. Получим аналогичное уравнение для $M^q_{ij}(t)$.

$M^q_{ij}(t) = \sum_{l = 1}^{n_q} P_{ql}(t)$, где $P_{ql}(t)$ --- вклад деревьев из $D_{ql}^t$ в $M^q_{ij}(t)$. Аналогично тому, как это сделано в $(\ref{eq:bar_p_12})$, полагаем $P_{ql}(t) = P^{(1)}_{ql}(t) + P^{(2)}_{ql}(t)$. При этом
\begin{equation}
    P^{(1)}_{ql}(t) = \delta^q_i \delta^l_j \frac{ p_{ij} R_{s_{ij}}(t-1) }{P_q(t)},
\end{equation}
где $R_X(t)$ --- вероятность наборов деревьев из $D^{\leqslant t}$, набор корней которых задан характеристическим вектором $X$, и высота хотя бы одного из которых достигает $t-1$. $P^{(2)}_{ql}(t)$ можно представить в виде
\begin{equation}
    P^{(2)}_{ql}(t) = \sum_{m = 1}^{\bar{s}_{ql}} P^{(2)m}_{ql}(t),
\end{equation}
где $P^{(2)m}_{ql}(t)$ --- вклад деревьев с $m$-м корнем на первом ярусе в $M^q_{ij}(t)$.

Обозначим через $S_1$ вклад в $P^{(2)m}_{ql}(t)$ наборов деревьев, в которых ярус $t$ достигается деревом с корнем в $m$-м нетерминале первого яруса. Очевидно,
\begin{equation}
    S_1 = \frac{ (1 - Q_{z_m}(t-1)) Q_{s_{ql} - \delta^{z_m}}(t-1) M^{z_m}_{ij}(t-1) }{ P_q(t) },
\end{equation}
где $z_m$ --- $m$-й нетерминал первого яруса.

Пусть $S_2$ --- вклад наборов, где ярус $t$ достигается через другие деревья. Тогда
\begin{equation}
    S_2 = \frac{ (1 - Q_{z_m}(t-1)) R_{s_{ql} - \delta^m}(t-1) \bar{M}^{z_m}_{ij}(t-1) }{ P_q(t) }.
\end{equation}

В результате, для $M^q_{ij}$ получаем
\begin{multline}
\label{eq:m_recur}
    M^q_{ij} = \sum_{l = 1}^{n_q} \left( P^{(1)}_{ql}(t) + \sum_{m = 1}^{\bar{s}_{ql}} P^{(2)m}_{ql}(t) \right) = \\
    = \frac{1}{P_q(t)} [ \delta^q_i p_{ij} R_{s_{ij}}(t-1) + \sum_{l = 1}^{n_q} p_{ql} \sum_{m = 1}^k (P_m(t-1) Q_{s_{ql} - \delta^m}(t-1) M^m_{ij}(t-1) + \\
    + (1 - Q_m(t-1)) R_{s_{ql} - \delta^m}(t-1) \bar{M}^m_{ij}(t-1)) ]
\end{multline}

Таким образом, получено рекуррентное соотношение для $M^q_{ij}(t)$, аналогичное $(\ref{eq:bar_m_recur})$.

Из леммы $\ref{l:powers}$ следуют равенства для $Q_X(t$ и $R_X(t)$:
\begin{equation}
\label{eq:qx_rx}
    \begin{split}
        &Q_X(t) = \prod_{i = 1}^k (1 - Q_i(t))^{x_i} = 1 - \sum_{i = 1}^k x_i Q_i(t) + \Theta\left( \sum_{i,j = 1}^k x_i x_j Q_i(t) Q_j(t) \right) \\
        &R_X(t) = Q_X(t) - Q_X(t-1) = \sum_{i = 1}^k x_i P_i(t) + \Theta\left( \sum_{i,j = 1}^k x_i x_j Q_i(t) Q_j(t) \right)
    \end{split}
\end{equation}

Теперь можно приступить к вычислению $\bar{M}'^q_{ij}(t)$ и $M'^q_{ij}(t)$.

\subsection{Случай критического класса}

Рассмотрим вначале случай, когда $I(q) \in J$.

Пусть $q,i \in I_\mu$. Тогда при $m \in I_\nu : \nu > \mu$ $\bar{M}'^m_{ij}(t) = 0$, и для $\bar{M}'^q_{in}(t$ получаем:
\begin{equation}
    \bar{M}'^q_{ij}(t) = \delta^i_q p_{ij} Q_{s_{ij}}(t-1) + \sum_{l = 1}^{n_q} p_{ql} \sum_{m \in I_\mu} s_{ql}^m \bar{M}'^m_{ij}(t-1) Q_{s_{ql} - \delta^m}(t-1)
\end{equation}
Подставляя выражения $(\ref{eq:qx_rx})$ где это необходимо, и учитывая, что $I(q) \in J$, имеем
\begin{equation}
\label{eq:bar_m_one}
    \bar{M}'^q_{ij}(t) = \delta^i_q + \sum_{m \in I_\mu} \sum_{l = 1}^{n_q} p_{ql} s_{ql}^m \bar{M}'^m_{ij}(t-1) - \sum_{m \in I_\mu} \sum_{l = 1}^{n_q} p_{ql} s_{ql}^m \bar{M}'^m_{ij}(t-1) \cdot \sum_{n \in I_\mu} (s_{ql}^n - \delta^m_n) Q_n(t-1) (1 + o(1)).
\end{equation}

Непосредственным взятием производных от производящих функций проверяются выражения для первых и вторых моментов:
\begin{equation}
\label{eq:moments}
    \begin{split}
        &a^q_m = \sum_{l = 1}^{n_q} p_{ql} s_{ql}^m \\
        &b^q_{mn} = \sum_{l = 1}^{n_q} p_{ql} s_{ql}^m (s_{ql}^n - \delta^m_n)
    \end{split}
\end{equation}
Подставляя их в $(\ref{eq:bar_m_one})$, получаем
\begin{equation}
\label{eq:bar_m_ksi}
    \bar{M}'^q_{ij}(t) = \delta^i_q p_{ij} + \sum_{m \in I_\mu} a^q_m \bar{M}'^m_{ij}(t-1) - c_\mu t^{\xi(\mu) - 1} \sum_{\substack{m \in I_\mu \\ n \in I}} b^q_{mn} u^{I(n)}_{n - k_{I(n)-1}} \bar{M}'^m_{ij}(t-1) (1 + o(1))
\end{equation}
где $\xi(\mu)$ --- число классов с перроновым корнем, равным 1 в цепочке $K_\mu, K_{\mu+1}, \ldots, K_m$, и $K_{I(n)} \ni A_n$.

Применяя лемму $\ref{l:x_b_a}$, получаем:
\begin{equation}
\label{eq:bar_m_star}
    \bar{M}'^q_{ij}(t) = u^{(\mu)}_{q - k_{\mu-1}} \cdot \sum_{l \in I_\mu} v^{(\mu)}_{n - k_{\mu-1}} \bar{M}'^l_{ij}(t) = u^{(\mu)}_{q - k_{\mu-1}} M^{(\mu)}_*(t),
\end{equation}
где $M^{(\mu)}_*(t) = \sum_{l \in I_{\mu}} v^{(\mu)}_{n - k_{\mu-1}} \bar{M}'^l_{ij}(t)$, и $v^{(\mu)}$ --- левый собственный вектор матрицы $A_{\mu,\mu}$.

Домножая $(\ref{eq:bar_m_ksi})$ на $v^{(\mu)}_{q - k_{\mu-1}}$ и суммируя по $q$, получаем:
\begin{equation}
    \delta \bar{M}^{(\mu)}_*(t) = v^{(\mu)}_{i - k_{\mu-1}} p_{ij} - c_{\mu} t^\alpha \sum_{q,m,n \in I_\mu} v^{(\mu)}_{q - k_{\mu-1}} b^q_{mn} u^{(\mu)}_{m - k_{\mu-1}} u^{(\mu)}_{n - k_{\mu-1}} = v^{(\mu)}_{i - k_{\mu-1}} p_{ij} - c_{\mu} t^\alpha B_\mu,
\end{equation}
где $\alpha = -\left( \frac{1}{2} \right)^{\xi(\mu) - 1}$.

Нетрудно видеть, что величина $\bar{M}^{(\mu)}_*(t)$ удовлетворяет условиям леммы $\ref{l:x_rec}$. Применяя её, получаем:
\begin{equation}
\label{eq:bar_m_result_1}
    \begin{split}
        &\bar{M}^{(\mu)}_*(t) = \frac{ v^{(\mu)}_{i - k_{\mu-1}} p_{ij} }{ c_{\mu} B_{\mu} + 1 } t (1 + o(1)),\quad\text{если}\;\alpha = -1 \\
        &\bar{M}^{(\mu)}_*(t) = \frac{ v^{(\mu)}_{i - k_{\mu-1}} p_{ij} }{ c_{\mu} B_{\mu} } t^{-\alpha} (1 + o(1)),\quad\text{если}\;\alpha > -1
    \end{split},
\end{equation}
где $\alpha = -\left( \frac{1}{2} \right)^{\xi(\mu) - 1}$.

Пусть теперь $I(q) < I(i)$ ($q$ и $i$ в различных классах). Тогда $\bar{M}'^q_{ij}(t)$ выражается следующим образом:
\begin{equation}
    \bar{M}'^q_{ij}(t) = \delta^i_q p_{ij} Q_{s_{ij}}(t-1) + \sum_{m \in I_\mu \cup I_{\mu+1}} \sum_{l = 1}^{n_q} p_{ql} s_{ql}^m \bar{M}'^m_{ij}(t-1) Q_{s_{ql} - \delta^m} (t-1)
\end{equation}
Учитывая малость $Q_n(t)$ при $I(n) > I(q)$, получаем
\begin{multline}
    \bar{M}'^q_{ij}(t) = O(p_{ij}) + \sum_{m \in I_\mu} \sum_{l = 1}^{n_q} p_{ql} s_{ql}^m \bar{M}'^m_{ij}(t-1) \cdot \left( 1 - \sum_{n \in I_\mu} (s_{ql}^n - \delta^m_n) Q_n(t-1) \right) (1 + o(1)) + \\
    + \sum_{m \in I_{\mu+1}} \sum_{l = 1}^{n_q} p_{ql} s_{ql}^m \bar{M}'^m_{ij}(t-1) (1 + o(1))
\end{multline}

Положим, $\bar{M}'^m_{ij}(t-1) = \overline{\mathcal{M}}'_{\mu+1} t^{\gamma(\mu+1)} (1 + o(1))$. Это выполняется для $\mu + 1 = m$, что видно из полученных соотношений $(\ref{eq:bar_m_result_1})$. Исходя из этого, получим выражение для $\bar{M}'^q_{ij}(t)$. Подставляя дополнительно выражения для первых и вторых моментов, а также для вероятностей продолжения $Q_n(t-1)$, получаем:
\begin{multline}
    \bar{M}'^q_{ij}(t) = \sum_{m \in I_\mu} a^q_m \bar{M}'^m_{ij}(t-1) - c_\mu t^{\alpha(\mu)} \sum_{m,n \in I_\mu} b^q_{mn} u^{(\mu)}_{n - k_{\mu-1}} \bar{M}'^m_{ij}(t-1)(1 + o(1)) + \\
    + \overline{\mathcal{M}}'_{\mu+1} t^{\gamma(\mu+1)} \sum_{m \in I_{\mu+1}} a^q_m u^{(\mu+1)}_{m - k_\mu} (1 + o(1))
\end{multline}
Домножая на $v^{(\mu)}_{q - k_{\mu-1}}$ и суммируя по $q$, получаем
\begin{multline}
    \delta \bar{M}^{(\mu)}_*(t) = \overline{\mathcal{M}}'_{\mu+1} \left( \sum_{\substack{q \in I_\mu \\ m \in I_{\mu+1}}} v^{(\mu)}_{q - k_{\mu-1}} a^q_m u^{(\mu+1)}_{m - k_\mu} \right) \cdot t^{\gamma(\mu+1)} - \\
    - c_\mu t^{\alpha(\mu)} \cdot \left( \sum_{q,m,n \in I_\mu} v^{(\mu)}_{q - k_{\mu-1}} b^q_{mn} u^{(\mu)}_{m - k_{\mu-1}} u^{(\mu)}_{n - k_{\mu-1}} \right) \cdot \bar{M}^{(\mu)}_*(t-1) (1 + o(1)) = \\
    = \overline{\mathcal{M}}'_{\mu+1} b_{\mu+1} t^{\gamma(\mu+1)} (1 + o(1)) - c_\mu B_\mu t^{\alpha(\mu)} \bar{M}^{(\mu)}_*(t-1) (1 + o(1))
\end{multline}

Случай $\alpha(\mu) = -1$ рассматривать не имеет смысла, так как это означает, что не существует критических классов с номерами, превышающими $\mu$. Полагая $\alpha(\mu) > -1$ и применяя лемму $\ref{l:x_b_a}$, получаем
\begin{equation}
\label{eq:bar_m_result_2}
    \bar{M}^{(\mu)}_*(t) = \frac{ \overline{\mathcal{M}}'_{\mu+1} b_{\mu+1} t^{\gamma(\mu+1) - \alpha(\mu)} }{ c_\mu B_\mu }
\end{equation}

Из полученных формул $(\ref{eq:bar_m_result_1})$ и $(\ref{eq:bar_m_result_2})$ нетрудно получить общее выражение для величины $\bar{M}^{(\mu)}_*(t)$, при условии что грамматика имеет вид <<цепочки>> и состоит только из критических классов ($J = \{1,2,\ldots,m\}$).
\begin{equation}
    \bar{M}^{(\mu)}_*(t) = \prod_{j = \mu}^{\nu-1} \left( \frac{ b_{j+1} }{ c_j B_j } \right) \cdot \left( \frac{ v^{(\nu)}_{i - k_{\nu-1}} p_{ij} }{ c_\nu B_\nu + \delta^m_\nu } \right) \cdot t^{ \frac{1}{2}^{m - \nu} \left(2 - \frac{1}{2}^{\nu - \mu} \right) },
\end{equation}
где $\mu = I(q)$, $\nu = I(i)$. Подставляя $(\ref{eq:bar_m_star})$, и затем $(\ref{eq:bar_m_ap})$, непосредственно получаем
\begin{equation}
    \bar{M}^q_{ij}(t) = \frac{ u^{(\mu)}_{q - k_{\mu-1}} }{ 1 - Q_q(t) } \prod_{j = \mu}^{\nu-1} \left( \frac{ b_{j+1} }{ c_j B_j } \right) \cdot \left( \frac{ v^{(\nu)}_{i - k_{\nu-1}} p_{ij} }{ c_\nu B_\nu + \delta^m_\nu } \right) \cdot t^{ \frac{1}{2}^{m - \nu} \left(2 - \frac{1}{2}^{\nu - \mu} \right) }
\end{equation}

Перейдём к вычислению $M^q_{ij}(t)$. Пусть $I(q) = I(i)$. Полагая $M'^q_{ij}(t) = M^q_{ij}(t) P_q(t)$, из $(\ref{eq:m_recur})$ получаем
\begin{multline}
\label{eq:m_ap_first}
    M'^q_{ij}(t) = O(p_{ij}) t^{\beta(\mu)} + \sum_{m \in I_\mu} \sum_{l = 1}^{n_q} p_{ql} s_{ql}^m M'^m_{ij}(t-1) - \\
    - \sum_{m,n \in I_\mu} \sum_{l = 1}^{n_q} p_{ql} s_{ql}^m (s_{ql}^n - \delta^m_n) Q_n(t-1) M'^m_{ij}(t-1) + \\
    + \sum_{m,n \in I_\mu} \sum_{l = 1}^{n_q} p_{ql} s_{ql}^m (s_{ql}^n - \delta^m_n) P_n(t-1) \bar{M}'^m_{ij}(t-1)
\end{multline}

Обозначим
\begin{equation}
    \begin{split}
        &Q^{(\mu)}(t) = c_\mu u^{(\mu)} t^{\alpha(\mu)} \\
        &P^{(\mu)}(t) = d_\mu u^{(\mu)} t^{\beta(\mu)}
    \end{split}
\end{equation}
Подставляя выражения $(\ref{eq:moments})$ для первых и вторых моментов в $(\ref{eq:m_ap_first})$, имеем
\begin{multline}
    M'^q_{ij}(t) = \sum_{m \in I_\mu} a^q_m M'^m_{ij}(t-1) - c_\mu t^{\alpha(\mu)} \sum_{m,n \in I_\mu} b^q_{mn} u^{(\mu)}_{n - k_{\mu-1}} M'^m_{ij}(t-1) (1 + o(1)) + \\
    + d_\mu t^{\beta(\mu)} \sum_{m,n \in I_\mu} b^q_{mn} u^{(\mu)}_{n - k_{\mu-1}} \bar{M}^m_{ij}(t-1) (1 + o(1))
\end{multline}
Обозначая дополнительно $M^q_{ij}(t) = \mathcal{M}_q t^{\left( \frac{1}{2} \right)^{m - \mu}}$, а также учитывая $\beta(\mu) = -1 -\left( \frac{1}{2} \right)^{m - \mu}$ и выражения для первых моментов, получаем
\begin{multline}
\label{eq:m_ap_rec}
    M'^q_{ij}(t) = \sum_{m \in I_\mu} a^q_m M'^m_{ij}(t-1) - c_\mu t^{\alpha(\mu)} \sum_{m,n \in I_\mu} b^q_{mn} u^{(\mu)}_{n - k_{\mu-1}} M'^m_{ij}(t-1) (1 + o(1)) + \\
    + d_\mu \mathcal{M}_\mu t^{-1} \sum_{m,n \in I_\mu} b^q_{mn} u^{(\mu)}_{m - k_{\mu - 1}} u^{(\mu)}_{n - k_{\mu-1}} (1 + o(1))
\end{multline}

Применяя лемму $\ref{l:x_b_a}$, получаем
\begin{equation}
    \begin{split}
        &M'^q_{ij}(t) = u^{(\mu)}_{q - k_{\mu-1}} M^{(\mu)}_*(t) (1 + o(1)) \\
        &M^{(\mu)}_*(t) = \sum_{m \in I_\mu} v^{(\mu)}_{n - k_{\mu-1}} M'^n_{ij}(t)
    \end{split}
\end{equation}

Домножая $(\ref{eq:m_ap_rec})$ на $v^{(\mu)}_{q - k_{\mu-1}}$ и суммируя по $q$, имеем
\begin{equation}
    \delta M^{(\mu)}_*(t) = d_\mu \mathcal{M}_\mu B_\mu t^{-1} - c_\mu t^{\alpha(\mu)} B_\mu M^{(\mu)}_*(t-1) (1 + o(1))
\end{equation}
Применяя лемму $\ref{l:x_rec}$, получаем в результате
\begin{equation}
\label{eq:m_star_result_1}
    M^{(\mu)}_*(t) = \left\{
    \begin{split}
        &d_\mu \overline{\mathcal{M}}'_\mu B_\mu (1 + o(1)),\quad{}\text{при}\;\mu = m, \\
        &\frac{ d_\mu \overline{\mathcal{M}}'_\mu }{ c_\mu } t^{-1 -\alpha(\mu)} (1 + o(1)),\quad{}\text{при}\;\mu < m
    \end{split}
    \right.
\end{equation}

Пусть теперь $I(q) = \mu$, $I(i) = \nu$, $\mu < \nu$, тогда из $(\ref{eq:m_recur})$ получаем
\begin{multline}
    M'^q_{ij}(t) = \delta^q_i p_{ij} R_{s_{ij}}(t-1) + \sum_{m \in I_\mu \cup I_{\mu+1}} \sum_{l = 1}^{n_q} p_{ql} s_{ql}^m \cdot [ Q_{s_{ql} - \delta^m}(t-1) M'^m_{ij}(t-1) + \\
    + (1 - Q_m(t-1)) R_{s_{ql} - \delta^m}(t-1) \bar{M}^m_{ij}(t-1) ],
\end{multline}
откуда
\begin{multline}
    M'^q_{ij}(t) = O\left( t^{\beta(\mu)} \right) + \sum_m a^q_m M'^m_{ij}(t-1) - \sum_{\substack{m \in I_\mu \cup I_{\mu+1} \\ n \in I_\mu }} b^q_{mn} Q_n(t-1) M'^m_{ij}(t-1) (1 + o(1)) + \\
    + \left( \sum_{\substack{ m \in I_\mu \cup I_{\mu+1} \\ n \in I_\mu }} b^q_{mn} P_n(t-1) \bar{M}^m_{ij}(t-1) \right) (1 + o(1))
\end{multline}

%TODO написать подробно
(...здесь подробнее...)

Можем записать
\begin{multline}
    M'^q_{ij}(t) = \sum_{m \in I_\mu} a^q_m M'^m_{ij}(t-1) - \sum_{m,n \in I_\mu} b^q_{mn} Q_n(t-1) M'^m_{ij}(t-1) (1 + o(1)) + \\
   + \sum_{m,n \in I_\mu} b^q_{mn} P_n(t-1) \bar{M}^m_{ij}(t-1) (1 + o(1))
\end{multline}
Домножая на $v^{(\mu)}_{q - k_{\mu-1}}$ и суммируя по $q$, имеем
\begin{equation}
    \delta M^{(\mu)}_*(t) = d_\mu \overline{\mathcal{M}}_\mu B_\mu t^{\beta(\mu) + \left( \frac{1}{2} \right)^{m - \nu} \left( 2 - \left( \frac{1}{2} \right)^{\nu - \mu} \right)} (1 + o(1)) - c_\mu B_\mu t^{\alpha(\mu)} \cdot M^{(\mu)}_*(t-1) (1 + o(1))
\end{equation}
Так как $\mu < m$, $\alpha(\mu) > -1$, поэтому по лемме $\ref{l:x_rec}$ получаем
\begin{equation}
\label{eq:m_star_result_2}
    M^{(\mu)}_*(t) = \frac{ d_\mu \overline{\mathcal{M}}_\mu }{ c_\mu } t^{-1 + \left( \frac{1}{2} \right)^{m - \nu} \left( 2 - \left( \frac{1}{2} \right)^{\nu - \mu} \right)}
\end{equation}

Объединяя результаты $(\ref{eq:m_star_result_1})$ и $(\ref{eq:m_star_result_2})$, получаем
\begin{equation}
    M^{(\mu)}_*(t) = \frac{ d_\mu \overline{\mathcal{M}}_\mu B_\mu }{ \delta^m_\mu (c_\mu B_\mu - 1) + 1 } \cdot t^{-1 + \left( \frac{1}{2} \right)^{m - \nu} \left( 2 - \left( \frac{1}{2} \right)^{\nu - \mu} \right)} (1 + o(1)),
\end{equation}
после чего из $(\ref{eq:m_ap_rec})$
\begin{equation}
    M^q_{ij}(t) = \frac{ u^{(\mu)}_{q - k_{\mu-1}} }{ P_q(t) } \frac{ d_\mu \overline{\mathcal{M}}_\mu B_\mu }{ \delta^m_\mu (c_\mu B_\mu - 1) + 1 } \cdot t^{-1 + \left( \frac{1}{2} \right)^{m - \nu} \left( 2 - \left( \frac{1}{2} \right)^{\nu - \mu} \right)} (1 + o(1))
\end{equation}


    


\begin{thebibliography}{99}
    \bibitem{gantmaher}
        Гантмахер~Ф.Р. \textbf{Теория матриц.} --- 5-е изд., --- М.: ФИЗМАТЛИТ, 2010 --- 560 с. --- ISBN~978-5-9221-0524-8
    \bibitem{borisov-disser}
        Борисов~А.Е. Закономерности в словах стохастических контекстно-свободных языков, порождённых грамматиками с двумя классами нетерминальных символов. Вопросы экономного кодирования.
\end{thebibliography}

\end{document}
