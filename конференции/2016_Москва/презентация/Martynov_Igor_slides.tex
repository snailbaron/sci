\documentclass{beamer}
\usetheme{Copenhagen}
\usepackage[utf8]{inputenc}
\usepackage[russian]{babel}
\usepackage{sansmathaccent}
\usepackage{hyperref}
\hypersetup{unicode=true}
\usepackage{units}
\usepackage{multicol}
\usepackage{qtree}
\usepackage{graphics}

%\uselanguage{russian}
%\languagepath{russian}
%\deftranslation[to=russian]{Theorem}{Теорема}


\renewcommand{\leq}{\leqslant}
\renewcommand{\geq}{\geqslant}

\title{О числе нетерминалов в деревьях вывода разложимой стохастической КС-грамматики}
\author{Мартынов Игорь Михайлович}



\begin{document}
	\setbeamertemplate{navigation symbols}{}
	
	\titlepage
	
	% Титульный слайд
%	\begin{frame}
%		\begin{center}
%			{\LARGE О числе нетерминалов в деревьях вывода разложимой стохастической КС-грамматики }
%		\end{center}
%		
%		\vfill
%		
%	
%		\begin{center}
%			{\Large Матрынов Игорь Михайлович}
%			
%			\vfill
%			
%			{\small Нижегородский государственный университет им. Н.И.~Лобачевского}
%		\end{center}
%	\end{frame}
	
	\setbeamertemplate{footline}[text line]
	{
		\parbox{\linewidth}{\vspace*{-8pt} \color{gray} Мартынов И.М.\hfill О числе нетерминалов в деревьях вывода \hfill\insertpagenumber}
	}

	\begin{frame}
		\frametitle{Контекстно-свободная грамматика (КСГ)}
		$$
			G = <V_N, V_T, S, R>
		$$
		
		$V_N$ и $V_T$ --- конечные алфавиты терминальных и нетерминальных символов
		$S \in V_N$ --- аксиома
		$R = \cup_{i=1}^k R_i$, где $R_i$ --- множество правил вида:
		\begin{equation*}
		r_{ij} : A_i \rightarrow \beta_{ij}\quad (j = 1,2,\ldots,n_i)
		\end{equation*}
		$\beta_{ij} \in (V_N \cup V_T)^*$
		
		Грамматика порождает слово $\alpha \in V_T^*$, если:
		$$
			S \rightarrow \beta_1 \rightarrow \ldots \rightarrow \alpha,
		$$
		где $\rightarrow$ означает замену левой части некоторого правила на его правую часть.
	\end{frame}

	\begin{frame}
		\frametitle{Пример вывода слова}
		\framesubtitle{Язык арифметических выражений c a,b без скобок}
		\begin{multicols}{2}
			{\small
				\begin{equation*}
				\begin{split}
				&V_N = \{E, M, T\}\\
				&V_T = \{+, -, *, /, a, b\}\\
				&S = E\\
				&{} \\
				&\text{Правила вывода:} \\
				&r_{11} : E \rightarrow M + E \\
				&r_{12} : E \rightarrow M - E \\
				&r_{13} : E \rightarrow M \\
				&r_{21} : M \rightarrow T * M \\
				&r_{22} : M \rightarrow T / M \\
				&r_{23} : M \rightarrow T \\
				&r_{31} : T \rightarrow a \\
				&r_{32} : T \rightarrow b
				\end{split}
				\end{equation*}
			}
			
			$$
				\alpha = a + b * a - b
			$$
			
			{\footnotesize
				\renewcommand{\qtreepadding}{1.8pt}
				\Tree [ .E [ .M [ .T a ] ] + [ .E [ .M [ .T b ] * [ .M [ .T a ] ] ] - [ .E [ .M [ .T b ] ] ] ] ]
			}
			
			\begin{multline*}
				E \rightarrow M + E \rightarrow T + E \rightarrow \ldots \rightarrow \alpha
			\end{multline*}
			
		\end{multicols}
	\end{frame}
	

	\begin{frame}
		\frametitle{Стохастическая КС-грамматика (СКСГ)}
		\begin{equation*}
			G = <V_N, V_T, S, R>
		\end{equation*}
		
		$V_N$, $V_T$, $S$ --- имеют тот же смысл.
		
		$R = \cup_{i=1}^k R_i$, где $R_i$ --- множество правил вида:
		\begin{equation*}
			r_{ij} : A_i \xrightarrow{p_{ij}} \beta_{ij}\quad (j = 1,2,\ldots,n_i)
		\end{equation*}
		$\beta_{ij} \in (V_N \cup V_T)^*$, $p_{ij}$ --- вероятность применения правила
		\begin{equation*}
			0 < p_{ij} \leq 1\quad \sum_{j=1}^{n_j} p_{ij} = 1
		\end{equation*}
		
		Вероятность дерева вывода: $p(d) = \prod_{p_{ij} \in d} p_{ij}$.
		
		Вероятность слова: $p(\alpha) = \sum_{d} p(d)$.
		
		Свойство согласованности: $\sum_{\left| \alpha \right| < \infty} p(\alpha) = 1$.
	\end{frame}

	\begin{frame}
		\frametitle{Пример вывода слова в СКСГ}
		\framesubtitle{Язык арифметических выражений c a,b без скобок}
		\begin{multicols}{2}
			{\small
				\begin{equation*}
				\begin{split}
				&V_N = \{E, M, T\},\;V_T = \{+, -, *, /, a, b\}\\
				&S = E\\
				&r_{11} : E \xrightarrow{1/4} M + E \\
				&r_{12} : E \xrightarrow{1/4} M - E \\
				&r_{13} : E \xrightarrow{1/2} M \\
				&r_{21} : M \xrightarrow{1/3} T * M \\
				&r_{22} : M \xrightarrow{1/3} T / M \\
				&r_{23} : M \xrightarrow{1/3} T \\
				&r_{31} : T \xrightarrow{1/4} a \\
				&r_{32} : T \xrightarrow{3/4} b
				\end{split}
				\end{equation*}
			}
			
			$$
			\alpha = a + b * a - b
			$$
			
			{\footnotesize
				\renewcommand{\qtreepadding}{1.8pt}
				\Tree [ .E [ .M [ .T a ] ] + [ .E [ .M [ .T b ] * [ .M [ .T a ] ] ] - [ .E [ .M [ .T b ] ] ] ] ]
			}
			
		\end{multicols}
	\end{frame}
	

	\begin{frame}
		\frametitle{Отношение следования на множестве нетерминалов}
		$A_i \rightarrow A_j$, если существует правило $A_i \xrightarrow{p_{ij}} \alpha A_j \beta$, где $\alpha, \beta \in (V_N \cup V_T)^*$.
		\vspace{5pt}
		
		$(\rightarrow_*)$ --- рефлексивное транзитивное замыкание $(\rightarrow)$.
		\vspace{5pt}
		
		$A_i \leftrightarrow_* A_j$, если $A_i \rightarrow_* A_j$ и $A_j \rightarrow_* A_i$.
		\vspace{5pt}
		
		Разобьём нетерминалы на классы по отношению $(\leftrightarrow_*)$. Тогда:
		\begin{equation*}
			\forall A_i, A_j \in K\quad A_i \rightarrow_* A_j
		\end{equation*}
		Не рассматриваем особые классы: $K = \{ A \}$, $A \not \rightarrow A$.
		\vspace{5pt}
		
		$K_1 \prec K_2$, если $\exists A_1 \in K_1, A_2 \in K_2 : A_1 \rightarrow A_2$.
		\vspace{5pt}
		
		$(\prec_*)$ --- рефлексивное транзитивное замыкание $(\prec)$.
		
		Особые классы: $K_i = \{A_j\}$, не $A_j \rightarrow A_j$. Предполагаем, что их нет.
	\end{frame}
	
	\begin{frame}
		\frametitle{Производящие функции}
		\begin{equation*}
			F_i(s_1, s_2, \ldots, s_k) = \sum_{j=1}^{n_i} p_{ij} s_1^{l_1} s_2^{l_2} \ldots s_k^{l_k},
		\end{equation*}
		где $l_s$ --- число нетерминалов $A_s$ в правой части правила $A_i \xrightarrow{p_{ij}} \beta{ij}$.
		\vspace{15pt}
		
		Производящие функции с параметром $t$:
		\begin{equation*}
			F_i(t, \bar{s}) = \left\{
			\begin{split}
				&F_i(\bar{s}),  & &\text{при}\;\; t = 1 \\
				&F_i(\bar{F}(t-1, \bar{s})), & &\text{при}\;\; t = 2,3,\ldots
			\end{split}
			\right.
		\end{equation*}
	\end{frame}
	
	\begin{frame}
		\frametitle{Первые моменты}
		\begin{equation*}
			\left. \frac{\partial F_i(s_1, s_2, \ldots, s_k)}{\partial s_j} \right|_{s_1 = s_2 = \ldots = s_k = 1} = a^i_j
		\end{equation*}
		$a^i_j$ --- математическое ожидание числа нетерминалов $A_j$ при однократном применении случайного правила к $A_i$.
		\vspace{10pt}
		
		Матрица $A = (a^i_j)$ --- матрица первых моментов.
		\vspace{10pt}
		
		$r$ --- максимальное по модулю собственное число матрица $A$ (перронов корень). Будем рассматривать критический случай $r = 1$.
	\end{frame}
	
	\begin{frame}
		\frametitle{Пример: матрица первых моментов}
		\framesubtitle{Язык арифметических выражений c a,b без скобок}
		\begin{multicols}{2}
			{\small
				\begin{equation*}
				\begin{split}
				&V_N = \{E, M, T\},\;V_T = \{+, -, *, /, a, b\}\\
				&S = E\\
				&r_{11} : E \xrightarrow{1/4} M + E \\
				&r_{12} : E \xrightarrow{1/4} M - E \\
				&r_{13} : E \xrightarrow{1/2} M \\
				&r_{21} : M \xrightarrow{1/3} T * M \\
				&r_{22} : M \xrightarrow{1/3} T / M \\
				&r_{23} : M \xrightarrow{1/3} T \\
				&r_{31} : T \xrightarrow{1/4} a \\
				&r_{32} : T \xrightarrow{3/4} b
				\end{split}
				\end{equation*}
			}
	
		$$
		A =
			\begin{pmatrix}
				\frac{1}{2} & 1 & 0 \\
				0 & \frac{2}{3} & 1 \\
				0 & 0 & 0
			\end{pmatrix}
		$$
		
		Классы: $\{E\}$, $\{M\}$, $\{T\}$.
		
		$\{T\}$ --- особый класс!
		
		Как избавиться:
		
		\begin{equation*}
		\begin{split}
		&r_{21} : M \xrightarrow{1/12} a * M\\
		&r_{22} : M \xrightarrow{3/12} b * M\\
		&\cdots
		\end{split}
		\end{equation*}
			
		\end{multicols}
	\end{frame}
	
	\begin{frame}
		\frametitle{Матрица первых моментов}

		Упорядочим классы:				
		\begin{equation*}
			K_i \prec_* K_j \Rightarrow i \leq j
		\end{equation*}
		
		Матрица первых моментов примет вид:
		\begin{equation*}
			A =
			\begin{pmatrix}
				A_{11} & A_{12} & A_{13} & \cdots & A_{1,m-1} & A_{1,m} \\
				0      & A_{22} & A_{23} & \cdots & A_{2,m-1} & A_{2,m} \\
				0      & 0      & A_{33} & \cdots & A_{3,m-1} & A_{3,m} \\
				\vdots & \vdots & \vdots & \ddots & \vdots & \vdots \\
				0      & 0      & 0      & \cdots & A_{m-1,m-1} & A_{m-1,m} \\
				0      & 0      & 0      & \cdots & 0      & A_{mm}
			\end{pmatrix}
		\end{equation*}
		
		Блоки $A_{ii} (i = 1,2,\ldots,m)$ неразложимы, имеют перроновы корни $r_i$.
		
		Перронов корень всей матрицы $r = \max\{r_i\}$.
		
		$r = 1$ --- критический случай.
	\end{frame}
	
	\begin{frame}
		\frametitle{Матрица первых моментов (2)}
		
		$s_{ij}^*$ -- максимальное число классов с перроновым корнем $r$ в цепочке $K_i \prec \ldots \prec K_j$.
		
		Дополнительно упорядочим классы по возрастанию $s_{1i}^*$. При равных $s_{1i}^*$, сначала критические.
		
		\begin{equation*}
		A =
		\begin{pmatrix}
		B_{11} & B_{12} & \cdots & B_{1,w-1} & B_{1,w} \\
		0      & B_{22} & \cdots & B_{2,w-1} & B_{2,w} \\
		\vdots & \vdots & \ddots & \vdots & \vdots \\
		0      & 0      & \cdots & 0      & B_{ww}
		\end{pmatrix}
		\end{equation*}

		$\mathcal{M}_1$ : $s_{1i}^* \leq 1$. $\mathcal{M}_k$ : $s_{1i}^* = k$ ($k > 1$).
	\end{frame}
	
	\begin{frame}
		\frametitle{Степень матрицы первых моментов}
		Пусть $B_{ij}^{(t)}$ --- блок матрицы $A^t$ на позиции блока $B_{ij}$ матрицы $A$.
		
		При $t \rightarrow \infty$
		\begin{equation*}
			B_{ij}^{(t)} \sim H_{ij} \cdot r^t \cdot t^{s^*_{ij} - 1}
		\end{equation*}
		\begin{flushright}
			\textit{Л. П. Жильцова}
		\end{flushright}
		
		В критическом случае:
		$$
			B_{ij}^{(t)} \sim H_{ij} \cdot t^{s^*_{ij} - 1}
		$$
			
	\end{frame}

	\begin{frame}
    	\frametitle{Граф классов нетерминалов}
    	
    	\begin{picture}(300,200)

            \put(145, 195){$K_1$}
        	\put(150, 180){\circle{20}}
        	
        	\put(90, 150){$K_2$}
        	\put(110, 140){\circle*{20}}
        	\put(200, 150){$K_3$}
        	\put(190, 140){\circle{20}}
        	
        	\put(145, 115){$K_4$}
        	\put(150, 100){\circle*{20}}
        	
        	\put(90,  70){$K_5$}
        	\put(110, 60){\circle{20}}
        	\put(200, 70){$K_6$}
        	\put(190, 60){\circle{20}}
        	
        	\put(60,  30){$K_7$}
        	\put(80,  20){\circle{20}}
        	\put(150, 30){$K_8$}
        	\put(140, 20){\circle*{20}}
        	\put(200, 30){$K_9$}
        	\put(190, 20){\circle*{20}}

            \put(143, 173){\vector(-1,-1){27}}
            \put(157, 173){\vector(1,-1){27}}
            \put(117, 133){\vector(1,-1){27}}
            \put(183, 133){\vector(-1,-1){27}}
            \put(143, 93){\vector(-1,-1){27}}
            \put(157, 93){\vector(1,-1){27}}
            \put(104, 51){\vector(-3,-4){18}}
            \put(116, 51){\vector(3,-4){18}}
            \put(190, 50){\vector(0,-1){20}}
        	
    	\end{picture}
	\end{frame}
	

	
	% Теорема от мат. ожиданиях
	\begin{frame}
    	\frametitle{Число применений правила вывода}
    	
    	\begin{theorem}
        	Пусть $q_{ij}(t)$ --- число применений правила $r_{ij} : A_i \rightarrow \beta_{ij}$ в случайном дереве вывода высоты $t$, и $q_i(t)$ --- число нетерминалов $A_i$ в таком дереве. Тогда
        	    	
        	\begin{equation*}
            	\begin{aligned}
                	& M(q_{ij}(t)) \sim c_i \cdot p_{ij} \cdot t^{(\frac{1}{2})^{s_1 - s_{1l} - 1}} \\
                	& M(q_i(t)) \sim d_i \cdot t^{(\frac{1}{2})^{s_1 - s_{1l} - 1}}
                \end{aligned} \quad
                (t \rightarrow \infty)
            \end{equation*}
            где $c_i$ и $d_i$ --- некоторые константы, и $p_{ij}$ --- вероятность применения правила $r_{ij}$.
        \end{theorem}
	\end{frame}
	
	% Теорема о дисперсии.
	\begin{frame}
    	\frametitle{Соотношение числа нетерминалов}
    	
    	\begin{theorem}
            Для любой пары нетерминалов $A_i \in K_h$ и $A_j \in K_l$, такой что $s_{1h} = s_{1l}$, в случайном дереве высоты $t$
    	
        	\begin{equation*}
            	D(\frac{q_i(t)}{q_j(t)}) \rightarrow 0 \quad (t \rightarrow \infty)
        	\end{equation*}

    	\end{theorem}
	\end{frame}
	
	% Список литературы
	\begin{frame}
		\frametitle{Использованная литература}
		{\footnotesize
		\begin{itemize}
			\item \textbf{Севастьянов Б. А.} Ветвящиеся процессы. --- M.: Наука, 1971 --- 436 с.
			\item \textbf{Фу К.} Структурные методы в распознавании образов. М.: Мир, 1977
			\item \textbf{Гантмахер Ф. Р.} Теория матриц. --- 5-е изд., --- М.: ФИЗМАТЛИТ, 2010
			\item \textbf{Ахо А., Ульман Дж.} Теория синтаксического анализа, перевода и компиляции. Том 1. М.: Мир, 1978
			\item \textbf{Жильцова Л. П.} О матрице первых моментов разложимой стохастической КС-грамматики. УЧЁНЫЕ ЗАПИСКИ КАЗАНСКОГО ГОСУДАРСТВЕННОГО УНИВЕРСИТЕТА, Том 151, кн. 2, 2009
			\item \textbf{Жильцова Л. П.} Закономерности применения правил грамматики в выводах слов стохастического контекстно-свободного языка // Математические вопросы кибернетики. Выр. 9. М.: Наука, 2000. С. 100-126.
			\item \textbf{Борисов А. Е.} Закономерности в словах стохастических контекстно-свободных языков, порождённых грамматиками с двумя классами нетерминальных символов. Вопросы экономного кодирования.
		\end{itemize}
		}
	\end{frame}
	
	\begin{frame}
		\begin{center}
			{\Huge \textbf{Спасибо за внимание!}}
		\end{center}
	\end{frame}
\end{document}