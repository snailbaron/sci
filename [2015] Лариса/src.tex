\documentclass[%twoside,%
11pt,a4paper]{article}
   % \usepackage{daiostyle}
    %\usepackage{fancystyle}

%\documentclass[12pt,russian]{article}
%\documentstyle{article}
\usepackage{amsmath,amssymb,latexsym}
%\usepackage[cp866]{inputenc}
\usepackage[russian]{babel}
\usepackage[T2A]{fontenc}
%\setlength{\topmargin}{15mm} \setlength{\headheight}{0pt}
%\setlength{\headsep}{0pt} \setlength{\topskip}{0pt}
\renewcommand{\refname}{Cписок литературы}

\topmargin=13mm           % веpхнее поле свеpх одного дюйма (до колонтитула)
\textwidth=160mm           % шиpина текста
\oddsidemargin=0.6cm % отступ от левого однодюймового поля
\textheight=235mm % высота текста, pассчитываемая так:
\headsep=0mm % pасстояние от колонтитула (хоть и пустого) до текста
\hoffset=0mm % сдвиг всей стpаницы впpаво
\voffset=-10mm % сдвиг всей стpаницы вниз


\date{}

\title{О свойствах вероятностных характеристик деревьев вывода в разложимых стохастических КС-грамматиках.
Докритический случай  }

\author{Л.~П.~Жильцова, И.~М.~Мартынов}

\begin{document}

\maketitle

\section{Введение}

\sloppy{ Автором в ~\cite{zhil1,zhil2} рассматривались вопросы, связанные с кодированием сообщений, являющихся словами стохастического контекстно-свободного языка (стохастического КС-языка), при условии, что матрица первых моментов грамматики неразложима, непериодична, и ее максимальный по модулю собственный корень (перронов корень) строго меньше единицы (докритический случай). При неразложимой матрице первых моментов нетерминальные символы грамматики образуют один класс.
В настоящей работе рассматриваются стохастические КС-грамматики с произвольным числом классов нетерминальных символов.

 При исследовании возможностей экономного кодирования слов, структурные и вероятностные свойства которых моделируются стохастической КС-грамматикой, в качестве меры эффективности кодирования рассматривается стоимость кодирования, которая определяется на множестве "`длинных"' слов. В качестве множества таких слов целесообразно рассматривать множество слов КС-языка, имеющих деревья вывода фиксированной высоты $t$  при $t\rightarrow\infty$. При этом возникает необходимость в вычислении математических ожиданий числа применений различных правил КС-грамматики в словах языка, имеющих дерево вывода высоты $t$. (добавить)




\section{Основные определения и предварительные сведения}

Для изложения результатов о контекстно-свободных языках будем
использовать определения КС-языка и стохастического КС-языка из ~\cite{aho,fu}.

Стохастической КС-грамматикой называется система
$G=\langle V_T,V_N,R,s \rangle,$ где $V_T$ и $V_N$ - конечные множества
терминальных и нетерминальных символов
(терминалов и нетерминалов) соответственно;
$s \in V_N$ - аксиома, $R$ -- множество правил.
Множество $R$ можно представить в виде
$R=\cup_{i=1}^k R_i$, где $k$ - мощность алфавита
$V_N$ и $R_i=\{r_{i1},\ldots, r_{i,n_i} \}.$
Каждое правило $r_{ij}$ из $R_i$ имеет вид
$$
r_{ij}:A_i\stackrel{p_{ij}}{\rightarrow }\beta _{ij},\,\,j=1,...,n_i,
$$
где $A_i\in V_N,\beta _{ij}\in (V_T\cup V_N)^{*}$ и $p_{ij}$ - вероятность
применения правила $r_{ij}$, удовлетворяющая следующим условиям:
$$
0<p_{ij}\leq 1 \,  \,\mbox {и}\,\, \sum_{j=1}^{n_i} p_{ij}=1.
$$

%Применение правила грамматики к слову в алфавите $V_T \cup V_N$ состоит
%в замене вхождения нетерминала из левой части правила на слово, стоящее
%в его правой части. КС-язык определяется как множество всех слов в алфавите
%$V_T,$ выводимых из аксиомы $s$ с помощью конечного числа применений правил грамматики. В работе рассматриваются бесконечные КС-языки.
				
				
%Каждому слову $\alpha$ КС-языка соответствует последовательность правил
%грамматики (вывод), с помощью которой $\alpha$ выводится из аксиомы $s.$
%Для определенности в качестве вывода будем рассматривать левый вывод, когда очередное правило грамматики применяется к самому левому вхождению нетерминала в слово. Вероятность вывода определяется как произведение вероятностей правил,
%его образующих.

Для слов $\alpha $ и $\beta$
из $(V_T\cup V_N)^{*}$ будем говорить, что $\beta$
непосредственно выводимо из $\alpha$ (и записывать $\alpha \Rightarrow \beta $), если
$\alpha=\alpha_1 A_i \alpha_2,$ $\beta =\alpha_1 \beta_{ij}\alpha_2$
для некоторых $\alpha_1,\alpha_2 \in ~(V_T\cup V_N)^{*},$
и в грамматике $G$ имеется правило $A_i\stackrel{p_{ij}}{\rightarrow }\beta
_{ij}.$

Обозначим через $\Rightarrow_{*}$ рефлексивное транзитивное замыкание
отношения $\Rightarrow$.
КС-язык, порождаемый грамматикой $G,$ определяется как множество слов $L_G=~\{
\alpha : s \Rightarrow_{*} \alpha, \alpha \in V_T^* \}.$ В работе рассматриваются бесконечные языки.

Каждому слову $\alpha$ КС-языка соответствует последовательность правил
грамматики (вывод), с помощью которой $\alpha$ выводится из аксиомы $s.$
Вероятность вывода определяется как произведение вероятностей правил,
образующих вывод. Вероятность слова $\alpha$ определяется как
сумма вероятностей всех различных левых выводов слова $\alpha$
(при левом выводе очередное правило применяется к самому левому нетерминалу
в слове).

Грамматика $G$ называется согласованной, если
%%
$$
\lim _{N\rightarrow \infty }\sum_{\alpha \in L, |\alpha |\le N}
p(\alpha)=1
$$
(здесь $|x|$ - длина слова $x$).
В работе рассматриваются согласованные KC-грамматики.
Согласованная КС-грам\-ма\-ти\-ка $G$ индуцирует распределение вероятностей
$P$ на множестве слов порождаемого КС-языка $L.$ 


Левому выводу соответствует дерево вывода, которое строится следующим образом.
Корень дерева помечается аксиомой $s.$ Пусть при выводе слова
$\alpha$ на очередном шаге в процессе левого вывода применяется
правило $A_i \stackrel{p_{ij}}{\rightarrow } b_{i_1} b_{i_2}\ldots
b_{i_m},$ где $b_{i_l} \in {V_N \cup V_T}$ $\,(l=1,\ldots,m).$
Тогда из самой левой вершины-листа дерева, помеченной символом
$A_i$ (при обходе листьев дерева слева направо), проводится $m$
дуг в вершины следующего яруса, которые помечаются слева направо
символами $b_{i_1}, b_{i_2},\ldots, b_{i_m}$ соответственно. Если правило грамматики имеет вид $A_i \stackrel{p_{ij}}{\rightarrow } \lambda,$ где $\lambda - $ пустое слово, в следующий ярус проводится одна дуга, и новая вершина помечается $\lambda.$
  
После построения дуг и вершин для всех правил грамматики в выводе слова
языка все листья дерева помечены символами из $A_t\cup \left\{\lambda\right\}$ и само
слово получается при обходе кроны дерева слева направо.
Высотой дерева называется максимальная длина пути от корня к листу. Вероятность дерева вывода определяется как вероятность соответствующего ему левого вывода.

Ярусы дерева будем нумеровать следующим образом.
Корень дерева расположен в нулевом ярусе. Вершины дерева,
смежные с корнем, образуют первый ярус, и т.д.
Дуги, выходящие из вершин $j-$го яруса, ведут к вершинам $(j+1)-$го яруса.

Рассмотрим многомерные производящие функции
$$
F_i( s_1,s_2,\ldots ,s_k) ,\,\,i=1,\ldots ,k,
$$
где переменная $s_i$ соответствует нетерминальному символу $A_i$ \cite{sev}.
Функция $F_i( s_1,s_2,\ldots ,s_k) $ строится по множеству правил $R_i$ с
одинаковой левой частью $A_i$ следующим образом.

Для каждого правила $A_i\stackrel{p_{ij}}{\rightarrow }\beta_{ij}$
выписывается слагаемое
$$
q_{ij=}p_{ij}\cdot s_1^{l_1}\cdot s_2^{l_2}\cdot \ldots \cdot s_k^{l_k},
$$
где $l_m$ - число вхождений нетерминального символа $A_m$ в правую часть
правила $(m=1, \ldots ,k).$
Тогда
$$
F_i( s_1,s_2,\ldots ,s_k) =\sum_{j=1}^{n_i}q_{ij}.
$$

Пусть
$$
a^{i}_{j}=\frac{\partial F_i( s_1,\ldots ,s_k) }{\partial s_j}\mid
_{s_1=s_2=\ldots =s_k=1.}
$$

Квадратная матрица $A$ порядка $k,$ образованная элементами $a^{i}_{j},$
называется {\it матрицей первых моментов} грамматики $G.$

Так как матрица $A$ неотрицательна, существует максимальный по модулю
действительный неотрицательный собственный корень (перронов корень) \cite{gant}.
Обозначим этот корень через $r$.

Известно необходимое и достаточное условие согласованности стохастической
КС-грамматики \cite{fu}:
стохастическая КС-грамматика при отсутствии бесполезных нетерминалов
(т.е. не участвующих в порождении слов языка) является согласованной тогда
и только тогда, когда $r\leq 1.$

В работе рассматривается докритический случай  $r < 1.$
Основные результаты относятся к стохастическим
КС-грамматикам с разложимой матрицей \cite{gant} первых моментов.

Введем некоторые обозначения.
Будем говорить, что нетерминал $A_j$ непосредственно следует за нетерминалом  $A_i$
(и обозначать $A_i \rightarrow A_j$), если в грамматике существует правило вида $A_i\stackrel{p_{il}}{\rightarrow}\alpha_1 A_j \alpha_2$, где $\alpha_1,\alpha_2 \in (V_T \cup V_N)^*.$
Рефлексивное транзитивное замыкание отношения $\rightarrow$ обозначим $\rightarrow _*.$

Грамматика называется неразложимой, если для любых двух различных нетерминалов $A_i$ и $A_j$ верно
$A_i \rightarrow_* A_j.$ В противном случае она называется разложимой. Классом нетерминалов назовем максимальное по включению подмножество $K \in V_N,$ такое, что $A_i \rightarrow_* A_j$ для любых $A_i, A_j \in K.$

Для различных классов $K_1$ и $K_2$ будем говорить, что класс $K_2$ непосредственно следует за классом $K_1$ (и обозначать $K_1 \prec K_2$), если
существуют $A_1 \in K_1$ и $A_2 \in K_2$, такие, что $A_1 \rightarrow A_2.$
Рефлексивное транзитивное замыкание отношения $\prec$ обозначим через $\prec _*$ и назовем отношением следования.

Очевидно, множество классов нетерминалов является разбиением множества $V_N,$ и отношение $\prec_*$ устанавливает на множестве классов нетерминалов частичный порядок.

Будем полагать, что классы нетерминалов перенумерованы числами от $1$ до $m$ таким образом, что из $K_i \prec K_j$ и $i\neq j$ следует $i < j.$

Соответствующая разложимой грамматике разложимая матрица \cite{gant} первых моментов $A$ имеет следующий вид:
\begin{equation}
A=\left(
\begin{array}{ccccc}
A_{11} & A_{12} & \ldots & A_{1 m-1} & A_{1m} \\
0 &    A_{22}   & \ldots & A_{2 m-1} & A_{2m} \\
\ldots & \ldots & \ldots & \ldots& \ldots \\
0 & 0 &  \ldots & A_{m-1 m-1} & A_{m-1 m} \\
0 & 0 &  \ldots & 0 & A_{m m} \\
\end{array}
\right).
\label{1}
\end{equation}

Один класс нетерминалов в матрице первых моментов представлен множеством подряд идущих строк и соответствующим множеством столбцов с теми же номерами.
Для класса $K_i$ квадратная подматрица, образованная соответствующими строками и столбцами, обозначается $A_{ii}$.
Блоки, расположенные ниже главной диагонали, нулевые в силу упорядоченности классов нетерминалов. Подматрица $A_{ij}$ является нулевой, если $K_i \nprec K_j.$

Будем считать, что в грамматике нет особых классов, т.е. классов, состоящих из одного нетерминала, для которых $A_{ii}=0.$ Этого всегда можно добиться, применяя метод укрупнения правил грамматики, 
описанный в \cite{zhil1}.  

Для каждого класса $K_i$ матрица $A_{ii}$ неразложима.
Обозначим через $r_i$ перронов корень матрицы $A_{ii}.$ Для неразложимой матрицы перронов корень является положительным и простым по теореме Фробениуса \cite{gant}.  Очевидно, $r=\max_{i} \{r_i\},$ и $r>0.$

Пусть $J=\{i_1,i_2, \ldots , i_l\}$ --- множество всех номеров $i_j$ классов, для которых $r_{i_j}=r.$
Назовем $J$ определяющим множеством.

Зафиксируем пару $(l, h), $ $l,h \in \{1,2,\ldots,m\},$ и рассмотрим всевозможные последовательности классов $K_{i_1} \prec K_{i_2} \prec \ldots \prec K_{i_s},$ где $i_1=l, i_s=h.$ Среди всех таких последовательностей выберем ту, которая содержит наибольшее число классов с номерами из $J$. Это число обозначим через $s_{l h}.$ В случае $K_l \nprec_* K_h$ положим $s_{l h}=0.$

%Множество последовательностей $\Pi_1,$ содержащих $q_1$ классов с номерами из $J,$ обозначим через ${\cal M}.$ Последовательности из ${\cal M}$ будем называть максимальными путями. Множество классов с номерами из $J,$ принадлежащих максимальным путям, обозначим через $J({\cal M}).$


%Положим $q_l=s_{1l}.$
Дополнительно переупорядочим классы по неубыванию величины $s_{1l},$ причем при одинаковых значениях $s_{1l}$ сначала поставим классы с номерами из множества $J.$

%Разобьем полученную последовательность классов $K_1,K_2, \ldots,K_m$ на группы классов ${\cal M}_1,{\cal M}_2, \ldots, {\cal M}_w,$ при этом класс $K_l$ отнесем к группе ${\cal M}_1$ при $s_{1l} \le 1$, и к группе ${\cal M}_j$ при $s_{1l} = j$ $(j>1).$
%Заметим, что каждая группа содержит хотя бы один класс с номером из определяющего множества $J.$

%Для краткости далее будем называть группу классов просто группой.
%Для пары групп ${\cal M}_i$ и ${\cal M}_j$ определим $s_{ij}^*$ как $\max_{K_l \in {\cal M}_i,K_h \in {\cal M}_j}\{s_{lh}\}$.
%Положим $s_{11}^*=1.$

Среди последовательностей вида
%$$
\begin{equation}
K_{i_1} \prec K_{i_2} \prec \ldots \prec K_{i_s},
%$$
\label{2a}
\end{equation}
где $i_1=l$ и $i_s$ принимает всевозможные значения, выберем последовательности с наибольшим числом классов с номерами из $J$. Это число обозначим через $q_l.$
% а сами такие последовательности назовем максимальными путями.
Максимальным путем назовем последовательность вида (\ref{2a}) при $i_1=1,$ содержащую $q_1$ классов с номерами из $J.$
Множество всех классов с номерами из $J$, принадлежащих максимальным путям, обозначим $J_{MAX}.$

%%Отметим, что для любого $K_i \in {\cal M}_j $ при $q_i>0$ найдется такой класс $K_l \in {\cal M}_{j+1},$ что $K_{i} \prec_* K_{l}.$
%Для группы ${\cal M}_{j}$ через $q_j^*$ обозначим $\max_{K_i \in {\cal M}_j}\{q_i\}.$
%Для групп ${\cal M}_i$ и ${\cal M}_j$ определим $s_{ij}^*$ как $\max_{K_l \in {\cal M}_i,K_h \in {\cal M}_j}\{s_{lh}\}$.

%Матрицу первых моментов будем также представлять в виде
%вид (\ref{2})
%\begin{equation} %(**)
%A=\left(
%\begin{array}{ccccc}
%B_{11} & B_{12} & \ldots & B_{1 w} \\
%0 &    B_{22}   & \ldots & B_{2w} \\
%\ldots & \ldots & \ldots & \ldots \\
%0 & 0 &  \ldots & B_{w w} \\
%\end{array}
%\right)
%,
%\label{2}
%\end{equation}
%где $B_{lh}$ -- подматрица на пересечении строк для классов из группы  ${\cal M}_l$ и столбцов для классов из ${\cal M}_h.$
%Заметим, что любая группа содержит хотя бы один класс с номером из определяющего множества $J.$
%Поэтому каждая матрица $B_{ll}$ имеет перронов корень $r.$ 
Запись $A_{lh}^{(t)}$ будем применять для обозначения соответствующей подматрицы  матрицы $A^t.$

\medskip

\textbf{Теорема 1} \cite{zhil3}.
{\label{zhilteo1}
%{\bf Теорема 1.}
{\em При $t\rightarrow \infty:$ 
$$
%\begin{equation}\label{zhileqv3}
%$$
a)\,\,A_{ij}^{(t)}=H_{ij} \cdot t^{s_{ij}-1} r^t (1+o(1)) \,\,\mbox {при} \,\, s_{ij}>0,  
%\end{equation}
$$
где $H_{ij}$ -- неотрицательная матрица, не зависящая от $t,$
$$
b)\,\, A_{ij}^{(t)}=o(r^t)\,\, \mbox {при} \,\, s_{ij}=0. 
$$

}

%Отметим, что строки матрицы $B_{ll},$ соответствующие классам $K_i \in {\cal M}_{l2},$ т.е. классам, для которых $i \in J,$ пропорциональны компонентам правого собственного вектора матрицы $A_{ii},$ а столбцы, соответствующие классам $K_j \in {\cal M}_{l2},$ пропорциональны компонентам левого собственного вектора матрицы $A_{jj}.$

Подробное описание строения матрицы первых моментов приведено в \cite{zhil3}. 

\medskip

Пусть $G=\langle V_T,V_N,R,s\rangle$ - стохастическая КС-грамматика, где $V_N=\{A_1,A_2,\ldots,A_k \}.$ 
Будем полагать $s=A_1.$ 
%Пусть $A_i$ - некоторый нетерминальный символ грамматики $G$. 
Через $L_i$ обозначим язык, порожденный грамматикой $G_i,$ которая получается из $G$ заменой аксиомы на нетерминал $A_i.$ Будем считать, что $L=L_1$ для исходного языка $L.$
Обозначим через $D_i$ множество деревьев вывода слов из $L_i$ и через $Q_l(t)$ - вероятность множества деревьев вывода из $D_i$, высота которых больше $t.$
Эту вероятность назовем вероятностью продолжения по аналогии с теорией ветвящихся процессов.

Пусть $\left(A_{j+1},A_{j+2},\ldots,A_{j+k_i}\right)$ - последовательность нетерминалов, образующих класс $K_i,$ где $k_i$ -- число нетерминалов в $K_i,$ и $j+1$ -- номер первого по порядку нетерминала в $K_i.$

Через $Q^{(i)}(t)$ обозначим вектор вероятностей продолжения $Q^{(i)}(t)=\left(Q_{j+1}(t),Q_{j+2}(t), \ldots,Q_{j+k_i}(t)\right)^{T}.$

\medskip

\textbf{Теорема 2 \cite{zhil4}.}
{\em
При $t \rightarrow \infty$
$$
Q^{(i)}(t)=U^{(i)} \cdot t^{q_i-1}\cdot r^t \cdot (1+o(1)),
$$ 
где $U^{(i)}$ - некоторый положительный вектор. 
}

\medskip

Отметим, что в случае $r_i =r$ вектор $U^{(i)}$ пропорционален правому собственному вектору матрицы $A_{ii}$, соответствующему $r.$ Подробное описание свойств $Q^{(i)}(t)$ содержится в \cite{zhil4}.
\medskip

Обозначим $D_l^t$ множество всех деревьев вывода высоты $t$ для слов из $L_l=L(G_l).$ Вероятность множества $D_l^t$ обозначим через $P_l(t).$ Очевидно, $P_l(t)=Q_l(t-1)-Q_l(t).$

Из теоремы 2 вытекает
\medskip

\textbf{Следствие.}
{\em
Пусть нетерминал $A_l \in K_i.$ Тогда
\begin{equation}
%$$
P_l(t)=u_l \cdot t^{q_i-1}r^{t-1}\cdot (1-r)\cdot (1+o(1)),
%$$
\label{24}
\end{equation}
где $u_l$ -- компонента вектора $U^{(i)},$ соответствующая нетерминалу $A_l.$}

\medskip
\section{Моменты}
\medskip

{\sloppy
Пусть $\Xi=(\xi_1,\ldots,\xi_k) -$ случайный вектор,
$\alpha^*=(\alpha_1, \ldots,\alpha_k )$ --- фиксированный вектор
с целочисленными неотрицательными компонентами и
$\alpha=\alpha_1+\ldots+\alpha_k.$ Обозначим
$$
\Xi^{[\alpha^*]}=\xi_1^{[\alpha_1]} \ldots \xi_n^{[\alpha_k]},
$$
где $x^{[a]}=x(x-1)\ldots (x-a+1).$
Математическое ожидание $M\Xi^{[\alpha^*]}$ будем называть $\alpha^*$-{\em моментом\/} $\Xi$ \cite{sev}.

%Пусть $A_i$ - некоторый нетерминальный символ грамматики $G$. Через $L_i$ обозначим язык, порожденный грамматикой $G_i,$ которая получается из $G$ заменой аксиомы на $A_i.$ Будем считать, что аксиомой исходной грамматики является нетерминал $A_1$ и $L=L_1$ для исходного языка $L.$
%Через $D_i$ обозначим множество деревьев вывода для слов из $L_i.$

Пусть $x^i_j(t) $  --- число нетерминалов $A_j$ в дереве вывода из
$D_i$ на ярусе $t.$ Через $M^i_{\alpha^*}(t)$ обозначим
$\alpha^*-$момент вектора
$X^i(t)=(x^i_1(t),\ldots,x^i_k(t)).$

Примем специальные обозначения для моментов первых
четырех порядков.
Факториальные моменты первого порядка будем обозначать через $a^i_j(t).$
Для факториальных моментов второго порядка введем обозначения
$b^i_{jn}(t).$ Таким образом, $b^i_{jj}(t)=Mx^i_j(t)(x^i_j(t)-1)$ и
$b^i_{jn}(t)=Mx^i_j(t)x^i_n(t)$ при $j\neq n.$
Для факториальных моментов третьего и четвертого порядков введем обозначения
$c^i_{jnq}(t)$ и $f^i_{jnql}(t)$ соответственно.

Нетрудно заметить, что $a^i_j(1)$ -- элементы матрицы первых моментов,
для которых мы ввели ранее обозначения $a^i_j.$
Будем также применять далее обозначения
$b^i_{jn}$ для $b^i_{jn}(1).$

Нас интересуют оценки для первых четырех моментов.

Свойства первых моментов исследованы в \cite{zhil3}, так как $a^i_j(t)$ - элемент матрицы $A^t$. 
Для вторых моментов известна следующая формула из \cite{sev}:
%$$
\begin{equation}
b^i_{jn}(t)=
\sum_{\tau=1}^t \sum_{l,m,s}a^i_l(t-\tau) b^l_{ms}
a^m_j(\tau-1) a^s_n(\tau-1).
%$$
\label{25}
\end{equation}
Пусть $a^i_l$ принадлежит подматрице $A_{h_i h_l},$ $a^m_j$ --- подматрице $A_{h_m h_j},$ и $a^s_n$ --- подматрице $A_{h_s h_n}.$
Подставим в (\ref{25}) представление для первых моментов:
$$
b^i_{jn}(t)=\sum_{\tau=1}^t \sum_{l,m,s} c_{il} \cdot \left(\left(t-\tau\right)^{\delta_1}\cdot  r^{t-\tau}+o\left(\left(t-\tau\right)^{\delta_1}\cdot r^{t-\tau}\right)\right)\cdot b^l_{ms} \cdot c_{mj}\times
$$
$$
\left(\left( \tau-1\right)^{\delta_2}r^{\tau-1}+o\left(\left(\tau-1\right)^{\delta_2}\cdot r^{\tau-1}\right)\right) \cdot
c_{sn}\cdot\left(\left(\tau-1\right)^{\delta_3}\cdot r^{\tau-1}+
o\left(\left(\tau-1\right)^{\delta_3}\cdot r^{\tau-1}\right)\right).
$$
Здесь $\delta_1=s_{h_i h_l}-1,$ $\delta_2=s_{h_m h_j}-1,$ и $\delta_3=s_{h_s h_n}-1,$ и $c_{il},$ $c_{mj}$ и  $c_{sn}$ ---
коэффициенты в соответствующих элементах матрицы $A^t.$


Проведя несложные преобразования в полученном равенстве, получим:
$$
b^i_{jn}(t)= r^t \cdot 
\sum_{l}   c_{il}\,t^{\delta_1} \cdot \left(\sum_{m,s} b^l_{ms}c_{mj} c_{sn}\right)\times
$$
$$
\sum_{\tau=1}^t\left(\left(1-\frac{\tau}{t} \right)^{\delta_1}+o \left(\left(1-\frac{\tau}{t} \right)^{\delta_1}\right)\right)
\cdot \left(\left(\tau-1\right)^{\delta_2+\delta_3}r^{\tau-2} \cdot \left( 1+o \left(1\right) \right)\right)
$$
Ряд
$\sum_{\tau=1}^{\infty} \left(\tau-1\right)^{\delta_2+\delta_3}  r^{\tau-2}$ сходится, поэтому величина 
$$
c_{il}\, \left(\sum_{m,s} b^l_{ms}c_{mj} c_{sn}\right)\cdot
\sum_{\tau=1}^t\left(\left(1-\frac{\tau}{t} \right)^{\delta_1}+o \left(\left(1-\frac{\tau}{t} \right)^{\delta_1}\right)\right)
\cdot \left(\left(\tau-1\right)^{\delta_2+\delta_3}r^{\tau-2} \cdot \left( 1+o \left(1\right) \right)\right)
$$
ограничена сверху.  Обозначим ее значение через $g^i_{jn}(l).$
Отметим, что $g^i_{jn}(l)>0$ в тех случаях, когда существуют $m$ и $s$ такие, что $b^l_{ms}>0,$
$$
K_{h_i}\prec_*K_{h_l}\prec_*K_{h_m}\prec_*K_{h_j} \, \, \, \mbox{и} \, \, \, K_{h_i}\prec_*K_{h_l}\prec_*K_{h_s}\prec_*K_{h_n}.
$$
Условие $b^l_{ms}>0$ выполняется тогда и только тогда, когда в грамматике существует правило с нетерминалом $A_l$ в левой части, содержащее в правой части оба нетерминала $A_m$ и $A_s.$

При $t\rightarrow \infty$
$$
b^i_{jn}(t)=\sum_{l} g^i_{jn}(l) \cdot t^{\delta_1} r^t (1+o(1)),
$$
где суммирование ведется по тем $l,$ для которых $g^i_{jn}(l)>0$.

Очевидно, определяющими в этой сумме являются слагаемые с теми значениями $l,$ для которых ${\delta_1}$ имеет наибольшее значение. Обозначим его через $\delta^i_{jn}.$ Поэтому формулу для $b^i_{jn}(t)$ можно записать в следующем виде:
%$$
\begin{equation}
b^i_{jn}(t)= g^i_{jn} t^{\delta^i_{jn}} r^t  \cdot (1+o(1)).
\label{50}
\end{equation}
%$$
Здесь $g^i_{jn}=\sum_{l} g^i_{jn}(l),$ где суммирование ведется по значениям $l,$ удовлетворяющим перечисленным выше условиям.
Так как $l\leq j$ и $l \le n,$ то $\delta^i_{jn} \leq \max \{s_{h_ih_j}~-~1,s_{h_ih_n}~-~1\}.$ Поэтому $b^i_{jn}(t) \le O\left(~ a^i_j(t)~+~a^i_n(t)~\right).$

Используя результаты из \cite{sev}, запишем формулу для третьего момента:
$$
c^i_{jnq}(t)=\sum_{\tau=1}^t \sum_l a^i_l(t-\tau)\cdot z^l_{jnq}(\tau-1).
$$
В этой формуле $z^l_{jnq}(\tau-1)$ состоит из конечного числа
слагаемых двух типов. Слагаемые первого типа имеют вид: $C
a^s_q(\tau~-~1) \cdot a^m_n(\tau~-~1) \cdot a^l_j(\tau~-~1) $ для некоторых
$s,m,l,$ где $C$ --- некоторая константа, зависящая от слагаемого;
слагаемые второго типа имеют вид: $C
a^l_j(\tau-1) \cdot b^m_{nq}(\tau-1)$ для некоторых $l,m$ и константы
$C.$

Поэтому вычисление $c^i_{jnq}(t)$ сводится к вычислению конечного числа
сумм вида
$$
S_1(t)= \sum_{\tau=1}^t a^i_l(t-\tau)\cdot a^l_j(\tau-1)\cdot a^s_q(\tau-1)
\cdot a^m_n(\tau-1)
$$
и вида
$$
S_2(t)= \sum_{\tau=1}^t a^i_l(t-\tau)\cdot a^s_j(\tau-1)\cdot b^m_{nq}(\tau-1)
$$
для некоторых значений $l,m,s.$

Оценим $S_1(t)$  и $S_2(t)$, используя оценки
$a^i_j(t)=O(t^{s_{ij}-1} r^t ),\,$ $b^i_{jl}(t) \le O\left(a^i_j(t)+a^i_l(t)\right).$ Применяя очевидное неравенство $s_{ij} \le w,$ где $w=\max_{i,j} \{ s_{ij} \}$, получим, что $S_1(t)\leq O(t^{w-1}r^t)$ и $S_2(t)\leq O(t^{w-1}r^t)$.
Поэтому
$$
c^i_{jnq}(t)\leq O\left(t^{w-1}r^t \right).
$$

%Аналогичный результат может быть получен и для четвертого момента.
%Для него известна формула \cite{sev}:
%$$
%f^i_{j_1 j_2 j_3 j_4}(t)=
%\sum_{\tau=1}^t \sum_{j=1}^k a^i_j(t-\tau) z^j_{j_1 j_2 j_3 j_4}(\tau-1),
%$$
%где
%$z^j_{j_1 j_2 j_3 j_4}(\tau-1)$ состоит из конечного числа слагаемых
%следующих четырех видов:
%$$
%S_1(\tau-1)=  a^{i_1}_{j_1}(\tau-1)\cdot a^{i_2}_{j_2}(\tau-1)
%\cdot a^{i_3}_{j_3}(\tau-1)\cdot a^{i_4}_{j_4}(\tau-1);
%$$
%$$
%S_2(\tau-1)=  a^{i_1}_{j_1}(\tau-1)\cdot a^{i_2}_{j_2}(\tau-1)
%\cdot b^{i_3}_{j_3 j_4}(\tau-1);
%$$
%$$
%S_3(\tau-1)= b^{i_1}_{j_1 j_2}(\tau-1)
%\cdot b^{i_2}_{j_3 j_4}(\tau-1);
%$$
%$$
%S_4(\tau-1)=  a^{i_1}_{j_1}(\tau-1)\cdot c^{i_2}_{j_2 j_3 j_4}(\tau-1)
%$$
%для некоторых значений $i_1, i_2, i_3, i_4.$

%Используя оценки для первых трех моментов и проведя элементарные
%преобразования, получаем оценку по порядку для четвертого момента:
%$$
%f^i_{j_1 j_2 j_3 j_4}(t)\leq O\left(t^{w-1}r^t \right).
%$$


\medskip

\section{Закономерности в деревьях вывода слов стохастического КС-языка
}

\vspace*{3mm}

\medskip

Для доказательства основного результата раздела предварительно докажем две леммы.

Через $R_X(n)$ обозначим выражение
\begin{equation}
\prod_{j=1}^k \left(1-Q_j(n)\right)^{x_j}-
\prod_{j=1}^k \left(1-Q_j(n-1)\right)^{x_j},
\label{z1}
\end{equation}
где $X=(x_1, \ldots, x_k)$ --- целочисленный неотрицательный вектор, $Q_j(n)$ --- вероятности продолжения ($j=1,\ldots,k$), $k$ --- общее число нетерминалов в грамматике.


\medskip

\textbf{Лемма 1}.
{\em
При $n \rightarrow \infty$
\begin{equation}
\left(1-\sum_{j=1}^k x_j Q_j(n-1)\right) \cdot \sum_{j=1}^k \left(x_j \cdot \frac {P_j(n)}{1-Q_j(n-1)} \right)\le R_X(n) \le
\sum_{j=1}^k \left(x_j \cdot \frac{P_j(n)}{1-Q_j(n)}\right).
\label{z2}
\end{equation}
\/}

\medskip

{\bf Доказательство.}

В $R_X(n)$ вынесем за скобки $\prod_{j=1}^k \left(1-Q_j(n-1)\right)^{x_j}:$ 

%%$$
\begin{equation}
R_X(n)=\prod_{j=1}^k \left(1-Q_j(n-1)\right)^{x_j} \cdot  \left( \prod_{j=1}^k \left(\frac{1-Q_j(n)}{1-Q_j(n-1)} \right)^{x_j}-1\right).
\label{z3}
\end{equation}

%$$
Заметим, что $\frac{1-Q_j(n)}{1-Q_j(n-1)}=1+\frac {P_j(n)}{1-Q_j(n-1)}.$

К выражению $\left(1+\frac {P_j(n)}{1-Q_j(n-1)}\right)^{x_j}$ применим формулу для бинома Ньютона и ограничимся первыми двумя слагаемыми. Получим неравенство $\left(1+\frac {P_j(n)}{1-Q_j(n-1)}\right)^{x_j} \geq 1+ x_j \cdot \frac {P_j(n)}{1-Q_j(n-1)}.$ Применяя это неравенство к (\ref{z3}), получим нижнюю оценку для $R_X(n):$
$$
R_X(n)\geq \prod_{j=1}^k \left(1-Q_j(n-1)\right)^{x_j} \cdot\left( \prod_{j=1}^k \left(1+x_j \cdot \frac {P_j(n)}{1-Q_j(n-1)}\right)^{x_j} -1 \right) \geq 
$$
$$
\geq \prod_{j=1}^k \left(1 -Q_j(n-1)\right)^{x_j} \cdot \left( \sum_{j=1}^k x_j \cdot \frac {P_j(n)}{1-Q_j(n-1)} \right).
$$
Для оценки $\prod_{j=1}^k \left(1 -Q_j(n-1)\right)^{x_j}$ используем следующее равенство, доказанное в \cite{sev}:
\begin{equation}
(1-y_1)^{n_1} \ldots (1-y_k)^{n_k}=
1- \Delta, \,\,
\mbox{где} \,\, 0 \le \Delta \le \sum_{j}n_j y_j, \, \, \mbox{и} \,\, \, 0\leq y_j\leq 1.
\label{z4}
\end{equation}

Положив $Q_j(n-1)$ в качестве $y_j$ и $x_j$ в качестве $n_j,$ получим неравенство
 $\prod_{j=1}^k \left(1 -Q_j(n-1)\right)^{x_j} \geq 1-\sum_{j=1}^k x_j \cdot Q_j(n-1).$
Используя это неравенство, получаем нижнюю оценку для $R_X(n):$
$$
R_X(n)\geq 
\left(1-\sum_{j=1}^k x_j Q_j(n-1)\right) \cdot \sum_{j=1}^k \left(x_j \cdot \frac {P_j(n)}{1-Q_j(n-1)} \right).
$$

Найдем теперь верхнюю оценку для $R_X(n).$
Аналогично (\ref{z3}) можно записать, что
$$
R_X(n)=\prod_{j=1}^k \left(1-Q_j(n)\right)^{x_j} \cdot  \left(1- \prod_{j=1}^k \left(\frac{1-Q_j(n-1)}{1-Q_j(n)} \right)^{x_j}\right)=
$$
\begin{equation}
=\prod_{j=1}^k \left(1-Q_j(n)\right)^{x_j} \cdot  \left(1- \prod_{j=1}^k \left(1-\frac {P_j(n)}{1-Q_j(n)}\right)^{x_j}\right).
\label{z5}
\end{equation}

Положив в (\ref{z4}) $\frac {P_j(n)}{1-Q_j(n)}$ в качестве $y_j$ и $x_j$ в качестве $n_j,$ получаем неравенство 
$$
 \prod_{j=1}^k \left(1-\frac {P_j(n)}{1-Q_j(n)}\right)^{x_j} \geq \left(1-\sum_{j=1}^k x_j \cdot \frac {P_j(n)}{1-Q_j(n)}\right).
$$
Используя эту оценку в (\ref{z5}), находим, что
$$
R_X(n)\le
\prod_{j=1}^k \left(1-Q_j(n) \right)^{x_j}\cdot
\sum_{j=1}^k \left(x_j \cdot \frac{P_j(n)}{1-Q_j(n)}\right)\le 
\sum_{j=1}^k \left(x_j \cdot \frac{P_j(n)}{1-Q_j(n)}\right).
$$

\medskip
Лемма доказана.

Из леммы 1 как следствие вытекает 

\medskip

???
\textbf {Лемма 2.}
{\em
При $n \rightarrow \infty$
$$
R_X(n)=
\left(1+\psi_X(n)\right)
\sum_{j=1}^k x_j P_j(n),
$$
где $-\tilde{c}_1 n^{q_1-1}r^n \cdot \sum x_j \le \psi_X(n)\le \tilde{c}_2n^{q_1-1 }r^n ,$ и
$\tilde{c}_1$ и $\tilde{c}_2$ -- положительные константы.%%\end{lemma}
}

\medskip

{\bf Доказательство.}

Применим теорему 2 для $Q_l(n),$ учитывая, что $q_1 \ge q_i$ для любого $i.$
Тогда при $n \rightarrow \infty$
$$
R_X(n)\le \sum_{l=1}^k x_l P(D_l^{n})\cdot
(1+\tilde{c}_2 n^{q_1-1 } r^n )
, \,\, \mbox{где}\, \, \tilde{c}_2>0.
$$

Для получения нижней оценки для $R_X(n)$ используем следующее равенство, доказанное в \cite{sev}:
%$$
%$$
Применяя (\ref{z2}) и (\ref{z3}), а также учитывая, что $\varphi_l(n)\geq 0$ для любого $l,$ получим, что
$$
R_X(n)\ge
\left(1-\sum_{j=1}^k x_j Q_j(n-1) \right)\sum_{l=1}^k x_l P(D_l^{n}))\geq
\left(1-\tilde{c}_1 n^{q^{*}_1-1} r^{n}\sum_{j=1}^k x_j\right) \sum_{l=1}^k x_l P(D_l^{n})
$$
для некоторой положительной константы $\tilde{c}_1.$

Лемма доказана.

\nopagebreak
\medskip

%Пусть ${\cal L}$ --- язык, порожденный стохастической КС-грамматикой $G.$
Будем полагать, как и ранее, что аксиомой исходной грамматики $G$
является нетерминал $A_1.$
Рассмотрим $D^t_1$ --- множество деревьев из $D_1$ высоты $t.$
Для $d \in D_1^t$ через $p_t(d)$ будем обозначать условную вероятность дерева $d,$ т.е. $p_t(d)=\frac{p(d)}{P(D_1^t)}.$

\medskip

Через $M_i(t,\tau)$ обозначим условное математическое ожидание числа вершин
на ярусе $\tau$, помеченных нетерминалом $A_i,$ в деревьях вывода высоты $t.$

Для нетерминала $A_l \in K_j$ положим $q_l'=q_j$ и $s_{1l}'=s_{1j}.$

%Введем обозначения:
%$f_{il}=\frac{d_l \cdot g_{il}^1}{d_1}$ и $f_{i}=\frac {d_i \cdot c_{1i}}{d_1},$ где
% $d_l$ --
\medskip

\textbf {Теорема 3.}
{\em Пусть $G$ --- стохастическая КС-грамматика с разложимой
матрицей первых моментов, для которой $r<1.$ 

Тогда для любого $i \in \{1, \ldots, k\}$ при $\tau \rightarrow \infty $
и $t-\tau \rightarrow \infty $
выполняется асимптотическое равенство
$$
M_i(t,\tau)\sim \frac {f_i \cdot(t-\tau)^{q_i'-1} \cdot \tau^{s_{1i}'-1} }{t^{q_1-1}}+
\sum_{l =1}^k \frac{f_{il} \cdot (t-\tau)^{q_l'-1}\cdot \tau^{\delta_{il}^1}}{t^{q_1-1}},
$$
в котором $f_{i},$ $f_{il}$ - неотрицательные константы и $\delta_{il}^1$ определено в $(\ref{50}).$
%и $N_i$ -- множество номеров $l,$ удовлетворяющих следующим условиям:

%1) нетерминал $A_l$ принадлежит классу $K_{j_1}$ с $j_1\in J_{MAX},$

%2) $K_{j_1} \prec_* K_{j_2},$ где $A_i \in K_{j_2}.$
}

\medskip

\textbf {Доказательство.}
Представим $M_i(t,\tau)$ в виде
$$
M_i(t,\tau)=\sum_{d \in D^t_1} p_t(d) z_{i}(d,\tau)=\frac{1}{P(D_1^t)}\sum_{d \in D^t_1} p(d) z_{i}(d,\tau),
$$
где $z_i (d,\tau)$ --  число вершин на ярусе $\tau$ дерева $d,$
помеченных нетерминалом $A_i.$

Рассмотрим неотрицательный целочисленный вектор $X=(x_1,\ldots,x_k),$
который будем называть далее вектором нетерминалов.
Используя вектор $X,$ мы можем записать, что
$$
M_i(t,\tau)= \frac{1}{P(D_1^t)}\sum_{X \ne 0} \Delta_X,
$$
где $\Delta_X$ -- вклад в математическое ожидание тех деревьев вывода
из $D^t_1$, которые на ярусе $\tau$ содержат $x_j$ вершин, помеченных
нетерминалом $A_j$ $(j=1,\ldots, k).$ Множество таких деревьев обозначим
через $D_X^t(\tau).$

Пусть $d \in D_X^t(\tau).$ Выделим в $d$ поддерево $d_0$ и последовательность
поддеревьев $(d_1,$ $d_2, \ldots d_n),$ где $n=\sum_{l=1}^k x_l.$
Поддерево $d_0$ получено из $d$ удалением всех вершин на ярусах
$\tau+1,$ $\tau+2, \ldots, t$ и инцидентных им дуг. Последовательность
$(d_1,$ $d_2, \ldots d_n)$ образуют все поддеревья, корни которых расположены
на ярусе $\tau$ дерева $d.$
При этом корни поддеревьев $d_1,$ $d_2, \ldots d_m$ расположены в дереве $d$
последовательно в порядке обхода вершин яруса $\tau$ слева направо, и каждое
дерево $d_l$ $(l=1,\ldots,n)$ содержит все дуги и вершины дерева $d,$
лежащие на путях от корня $d_l$ к листьям дерева $d.$

Выделим в $D_X^t(\tau)$ множество деревьев, имеющих в качестве поддерева
$d_0$ одно и то же дерево. Обозначим это множество через $D_0.$
Нетрудно понять, что
\begin{equation}
P\left(D_0\right)=
p(d_0)\cdot
\left(\prod_{l=1}^k (1-Q_l(t-\tau))^{x_l} -
\prod_{l=1}^k (1-Q_l(t-\tau-1))^{x_l}\right),
\label{z6}
\end{equation}
где $Q_l(n)$ --- суммарная вероятность деревьев из множества $D_l$,
высота которых больше $n,$ и, следовательно,
$\left(1-Q_l(n)\right)$ -- суммарная вероятность деревьев из
$D_l,$ высота которых не превосходит~$n.$

Обозначим через $\delta_1(X)$ выражение
$\prod_{l=1}^k (1-Q_l(t-\tau))^{x_l} $
и через $\delta_2(X)$ --- выражение
$ \prod_{l=1}^k (1-Q_l(t-\tau-1))^{x_l}.$

В (\ref{z6}) величина $p(d_0)\cdot\delta_1(X) $
есть суммарная вероятность деревьев, определяемых поддеревом $d_0,$
высота которых не превосходит $t,$ так как каждое поддерево с корнем
на ярусе $\tau$ имеет высоту, не превосходящую $(t-\tau).$

Вторая величина $p(d_0)\cdot\delta_2(X) $
есть суммарная вероятность деревьев, определяемых поддеревом $d_0,$
высота которых не превосходит $(t-\tau-1).$

Разность этих величин равна, очевидно, суммарной вероятности
деревьев высоты $t$, определяемых поддеревом $d_0,$ и значение
$\delta_1(X)-\delta_2(X)$ не зависит от порядка следования
вершин на ярусе $\tau,$ помеченных нетерминалами.

Выражение для $\delta_1(X)-\delta_2(X)$ в обозначениях леммы 1 есть $R_X(n)$ при $n=t-\tau.$  
 Поэтому
$$
P(D_X^t(\tau))=\sum_{d_0}\left( p(d_0)\cdot R_X(t-\tau)\right) =
R_X(t-\tau)\sum_{d_0}p(d_0),
$$
где суммирование ведется по всем возможным поддеревьям $d_0$
деревьев из $D_X^t(\tau).$

Через $D_X(\tau)$ обозначим множество всех деревьев вывода из $D_1,$ которым на ярусе $\tau$ соответствует вектор нетерминалов $X.$ Рассмотрим дерево из $D_X(\tau).$ Для каждой вершины этого дерева, помеченной некоторым нетерминалом $A_l,$ суммарная вероятность возможных деревьев с корнем в этой вершине
и листьями, помеченными только символами из $V_T\cup \{\lambda\}$, равна $P(D_l).$
Ввиду согласованности исходной грамматики $P(D_l)=1$ для любого $l$.
Поэтому $\sum_{d_0}p(d_0)$ равна вероятности деревьев вывода из $D_1,$ имеющих $x_l$ вершин на ярусе
$\tau$, помеченных нетерминалом $A_l$ $(l=1,\ldots,k):$
$$
\sum_{d_0}p(d_0)=
\sum_{d_0}p(d_0) \cdot P(D_1)^{x_1} \cdot P(D_2)^{x_2}\cdot \ldots
\cdot P(D_k)^{x_k}=P\left(D_X(\tau)\right).
$$

Далее будем обозначать $P\left(D_X(\tau)\right)$ через $P_X(\tau).$
Таким образом,
$$
M_i(t,\tau)=\frac{1}{P(D^t_1)} \sum_{X \ne 0}\left( P_X(\tau)\cdot
R_X(t-\tau)\cdot x_{i}\right).
$$
Применяя лемму 2 для представления $R_X(t-\tau),$ получим:
$$
M_i(t,\tau)=\frac{1}{P(D^t_1)} \sum_{X \ne 0}\left( P_X(\tau)\cdot x_{i}\cdot
\left(1+\psi_X(t-\tau)\right)
\sum_{l=1}^k x_l P\left(D_l^{t-\tau}\right)\right)=
$$
$$
\sum_{l=1}^k\frac{P(D_l^{t-\tau})}{P(D^t_1)}
\sum_{X \ne 0}\left( P_X(\tau)\cdot x_{i}\cdot
x_{l}\cdot\left(1+\psi_X(t-\tau)\right)\right).
$$

Отдельно вычислим
$S_1=
\sum_{X \ne 0}\left( P_X(\tau)\cdot x_{i}\cdot x_{l}\right) $
и
$S_2=
\sum_{X \ne 0}\left( P_X(\tau)\cdot x_{i}\cdot x_{l} \cdot \psi_X(t-\tau)\right).$

Используя первые и вторые моменты, мы можем записать, что
$S_1=b^1_{il}(\tau)$ при $i \ne l$ и
$S_1=b^1_{ii}(\tau)+a^1_i(\tau)$ при $l=i.$

Учитывая оценку из леммы 2 для $\psi_X(n)$ и используя
первые три момента, получим нижнюю и верхнюю
оценки для $S_2:$
$$
S_2 =
\sum_{X \ne 0} \left(P_X(\tau)\cdot x_{i}\cdot x_{l} \cdot \psi_X(t-\tau)\right)\ge
$$
$$
-\tilde{c}_2 \tau^{q_1-1} r^{\tau} \sum_{X \ne 0} \left(P_X(\tau)\cdot x_{i}\cdot x_{l} \cdot
\sum_j x_{j}\right)=- \tilde{c}_2 \tau^{q_1-1} r^{\tau}\sum_j c^{1*}_{ilj}(\tau),
$$
где
$$
c^{1*}_{ilj}(\tau)=c^{1}_{ilj}(\tau)
\,\, \mbox { при} \,\, i \ne l, \,\,i \ne j \,\,\mbox { и}\,\, j \ne l,
$$
$$
c^{1*}_{iii}(\tau)=c^{1}_{iii}(\tau)+3 b^1_{ii}(\tau)-a^1_{i}(\tau),
$$
и
$$
c^{1*}_{ijj}(\tau)=c^{1}_{ijj}(\tau)+b^1_{ij}(\tau),\,\,
c^{1*}_{iij}(\tau)=c^{1}_{iij}(\tau)+b^1_{ij}(\tau).
$$
Применяя оценки для первых трех моментов, получим, что
$S_2 \ge -c \cdot\tau^{2 q_1-2} r^{2 \tau},$
где $c$ --- некоторая положительная константа.

С другой стороны,
$$
S_2  \le c_1 \tau^{q_1-1} r^{\tau}
\sum_{X \ne 0} \left(P_X(\tau)\cdot x_{i}\cdot x_{l}\right)= c_1\tau^{q_1-1} r^{\tau}\cdot S_1.
$$
Так как $S_1=b^1_{il}(\tau)$ при $i \ne l$ и $S_1=b^1_{ii}(\tau)+a^1_i(\tau),$
то, с учетом оценок для моментов, получаем, что $S_2=O(\tau^{2q_1-2}r^{2\tau}).$

Вернемся к вычислению $M_i(t,\tau):$
$$
M_i(t,\tau)=
%$$
%$$
\sum_{l\ne i} \frac{P(D_l^{t-\tau})}{P(D^t_1)}
b^1_{il}(\tau)+ \frac{P(D_i^{t-\tau})}{P(D^t_1)}
\left(b^1_{ii}(\tau)+a^1_i(\tau) \right)+
\sum_{l=1}^k\frac{P(D_l^{t-\tau})}{P(D^t_1)} \cdot O\left(\tau^{2q^{*}_1-2}r^{2\tau}\right).
$$

Раскрывая моменты и используя лемму 2, после
несложных преобразований получим:
$$
M_i(t,\tau)=
$$
$$
\sum_{l=1}^{k} \frac{d_l \cdot(1-r)\cdot (t-\tau)^{q_l'-1} \cdot r^{t-\tau-1} \cdot (1+\phi_l(t-\tau))}
{d_1 \cdot (1-r)\cdot t^{q_1-1} \cdot r^{t-1} \cdot (1+\phi_1(t))}\cdot
\left(g_{il}^1 \cdot r^{\tau}\cdot \tau^{\delta_{il}^1}  \left(1+\psi_{il}(\tau)\right)\right)+
$$
$$
\frac{d_i \cdot (1-r) \cdot (t-\tau)^{q_i'-1} r^{t-\tau-1}(1+\phi_i(t-\tau)) \cdot c_{1i}\cdot
\tau^{s_{1i}'-1} r^{\tau}(1+\varphi_{1i}(n)(\tau))}
{d_1\cdot (1-r)\cdot t^{q_1-1}r^{t-1} \cdot(1+\phi_1(t))}+ O\left(\tau^{2q^{*}_1-2}r^{2\tau}\right),
$$
где $\phi_i(n)=o(1),$ $\psi_{il}(n)=o(1),$ $\varphi_{1i}(n)=o(1),$ $q_l'=q_j$ для $A_l \in K_j,$ и $s_{1i}'=s_{1m}$ для  $A_i \in K_m.$

Отсюда следует, что
\begin{equation}
M_i(t,\tau)=
 \frac{1}{d_1}\left(\sum_{l=1}^{k} \frac{d_l \cdot g_{il}^1 \cdot (t-\tau)^{q_l'-1}\cdot \tau^{\delta_{il}^1}}{t^{q_1-1}}+ \right.
\label{32}
\end{equation}
$$
\left.
 \frac {d_i \cdot c_{1i}\cdot(t-\tau)^{q_i'-1} \cdot \tau^{s_{1i}'-1} }{t^{q_1-1}}\right) \left(1+\xi(\tau,t-\tau)\right),
$$
где $\xi(\tau,t-\tau) \rightarrow 0$ при $\tau,t-\tau\rightarrow \infty.$ Теорема доказана.

%$$
%\delta_i=\frac {d_i \cdot c_{1i}}{d_1} \cdot \frac{(t-\tau)^{q_i'-1} \cdot \tau^{s_{1i}'-1} }{t^{q_1-1}} ,
%$$
%если $A_i \in K_{j},$ $j \in J,$ и $K_{j}$ принадлежит некоторому максимальному пути; $\delta_i=0$ в противном случае.
\medskip

Рассмотрим подробнее слагаемые в (\ref{32}).
Определяющими в сумме являются те значения $l,$ для которых $g_{il}^1>0$ и $q_l'~+~\delta_{il}^1~=~q_1.$ Равенство справедливо при одновременном выполнении следующих условий:

1) нетерминал $A_l$ принадлежит классу $K_{j_1}$ с $j_1\in J_{MAX},$

2) $A_i \in K_{j_2},$ для которого $K_{j_1} \prec_* K_{j_2}.$

Обозначим $N_i$ множество номеров $l,$ для которых выполняются условия 1) и 2).

Отметим, что слагаемое $\frac {d_i \cdot c_{1i}\cdot(t-\tau)^{q_i'-1} \cdot \tau^{s_{1i}'-1} }{t^{q_1-1}} $ влияет на значение $M_i(t,\tau)$ при
$s_{1i}'~+~q_i'~-~1~=~q_1.$ Это равенство выполняется в случае, если $A_i \in K_{j_2},$ где $j_2\in J_{MAX}.$
Поэтому равенство (\ref{32}) при $N_i \ne \emptyset$ можно записать в виде
$$
M_i(t,\tau)=
 \left(\sum_{l \in N_i} \frac{f_{il} \cdot (t-\tau)^{q_l'-1}\cdot \tau^{\delta_{il}^1}}{t^{q_1-1}}+ \frac {f_i \cdot(t-\tau)^{q_i'-1} \cdot \tau^{s_{1i}'-1} }{t^{q_1-1}}\right) \left(1+\xi(\tau,t-\tau)\right),
$$
где $f_{il}=\frac{d_l \cdot g_{il}^1}{d_1},$ $f_{i}=\frac {d_i \cdot c_{1i}}{d_1}$ и $\xi(\tau,t-\tau) \rightarrow 0$ при $\tau,t-\tau\rightarrow \infty.$
Очевидно, $M_i(t,\tau)\le O(1/t)$ при $N_i = \emptyset.$
Поэтому справедливо

\medskip
\textbf{Следствие.}
{\em
$$
1) M_i(t,\tau)=
 \left(\sum_{l \in N_i} \frac{f_{il} \cdot (t-\tau)^{q_l'-1}\cdot \tau^{\delta_{il}^1}}{t^{q_1-1}}+ \frac {f_i \cdot(t-\tau)^{q_i'-1} \cdot \tau^{s_{1i}'-1} }{t^{q_1-1}}\right) \left(1+\xi(\tau,t-\tau)\right)
$$
при $N_i \ne \emptyset;$

$2)\,\,\, M_i(t,\tau)\le O(1/t)$ при $N_i = \emptyset$.

}

\medskip

Пусть $r_{ij}$ --- произвольное правило грамматики $G.$
Через $s_l^{(ij)}$ обозначим число нетерминалов $A_l$
в правой части правила $r_{ij}.$
Условное математическое ожидание числа применений правила $r_{ij}$
в деревьях вывода высоты $t$ на ярусе $\tau$ будем обозначать через
$M_{ij}(t,\tau).$

\medskip

\textbf {Теорема 4.}
{\em Пусть $G$ --- стохастическая КС-грамматика с разложимой
матрицей первых моментов, для которой перронов корень $r<1$ , и $D^t_1$ -- множество деревьев вывода
высоты $t.$

Тогда при $\tau \rightarrow \infty $ и $t-\tau \rightarrow \infty $
выполняется следующее асимптотическое равенство:
$$
M_{ij}(t,\tau)\sim \frac{p_{ij}}{t^{q_1-1}} \left(\sum_{l =1}^k f_{il} \cdot (t-\tau)^{q_l'-1}\cdot \tau^{\delta_{il}^1}+
\frac{1}{{r }} \sum_{m=1}^k f_m\cdot s_m^{(ij)}\cdot
 (t-\tau)^{q_m'-1}\tau^{s_{1i}'-1}\right).
$$
}
В формулировке теоремы $p_{ij}$ --- вероятность правила $r_{ij},$ задаваемая в исходной грамматике,
$s_m^{(ij)}$ -- число нетерминалов $A_m$ в правой части правила $r_{ij},$
а величины $q_l',$ $\delta_{il}^1,$ $f_{il}$ и $f_m$
имеют тот же смысл, что и в теореме 3.

\medskip

\textbf {Доказательство.} \nopagebreak
%Представим $M_{ij}(t,\tau)$ в виде
%$$
%M_{ij}(t,\tau)=\sum_{d \in D^t_1} p_t(d) z_{ij}(d,\tau),
%$$
Обозначим $z_{ij}(d,\tau)$ число вершин на ярусе $\tau$ дерева $d,$
помеченных нетерминалом $A_i,$ к которым применено правило $r_{ij}.$
Используя неотрицательный целочисленный вектор $X=(x_1,\ldots,x_k),$ можно записать, что
$$
M_{ij}(t,\tau)= \sum_{X \ne 0}
\sum_{d \in D_X^t(\tau)} p_t(d) z_{ij}(d,\tau),
$$
где $D_X^t(\tau)$ введено в доказательстве теоремы 3.

Представим $z_{ij}(d,\tau)$ в виде суммы случайных величин
$I_1+I_2+ \ldots+I_{x_i},$
где $I_m=1,$ если к $m$-й по порядку вершине среди вершин,
помеченных нетерминалом $A_m$ на ярусе $\tau,$
применено правило $r_{ij},$ и $I_m=0$ в противном случае
$(m=1,2, \ldots, x_i).$
Тогда
$$
M_{ij}(t,\tau)= \sum_{X \ne 0}
\sum_{d \in D_X^t(\tau)} p_t(d)\cdot (I_1+I_2+\ldots +I_{x_i}).
$$
Очевидно, что случайные величины $I_m$ $(m=1,2, \ldots, x_i)$ --
одинаково распределены на $D_X^t(\tau),$ поэтому
$$
M_{ij}(t,\tau)=\frac{1}{P(D^t_1)}\sum_{X \ne 0}
P\left(D_{X,1}^t(\tau)\right) \cdot x_i,
$$
где $P(D_{X,1}^t(\tau))$ -- суммарная вероятность тех деревьев
из $D_X^t(\tau),$ в которых правило $r_{ij}$ применено к первой по порядку
вершине на ярусе $\tau,$ помеченной $A_i.$

Подсчитаем вероятность $P\left(D_{X,1}^t(\tau)\right):$
$$
P\left(D_{X,1}^t(\tau)\right)=p_{ij} \cdot
P_X(\tau) \times
%$$
%$$
\left[\prod_{m=1}^k \left(1-Q_m(t-\tau)\right)^{x_m^{\prime}}\cdot
\prod_{m=1}^k \left(1-Q_m(t-\tau-1)\right)^{s_m^{(ij)}}- \right.
$$
%$$
\begin{equation}
\left. \prod_{m=1}^k \left(1-Q_m(t-\tau-1)\right)^{x_m^{\prime}}\cdot
\prod_{m=1}^k \left(1-Q_m(t-\tau-2)\right)^{s_m^{(ij)}}\right].
%$$
\label{33}
\end{equation}
Здесь $X^{\prime}=(x_1^{\prime},\ldots,x_k^{\prime})=
(x_1,\ldots,x_{i-1},x_i-1,x_{i+1},\ldots, x_k)$ и
$S=(s_1^{(ij)}, \ldots, s_k^{(ij)}),$ где $s_m^{(ij)}$ равно
числу нетерминалов $A_m$ в правой части правила $r_{ij}$ $(m=1,\ldots, k).$
Величина $P_X(\tau)$ имеет тот же смысл, что и в доказательстве теоремы 3.
Выражение в квадратных скобках в (\ref{33})
аналогично выражению $R_X(t-\tau).$  При этом
с помощью множителей $\left(1-Q_m(t-\tau-1)\right)^{s_m^{(ij)}}$ и
$\left(1-Q_m(t-\tau-2)\right)^{s_m^{(ij)}}$
учитывается тот факт, что к первому нетерминалу $A_i$ на ярусе $\tau$
применено правило $r_{ij},$ которому на ярусе $\tau+1$ соответствует
$s_m^{(ij)}$ вершин, помеченных нетерминалом $A_m$ $(m=1, \ldots, k).$

Проведем несложные преобразования в (\ref{33}):
$$
P\left(D_{X,1}^t(\tau)\right)=p_{ij} \cdot P_X(\tau) \cdot
\frac{1}{1-Q_i(t-\tau)}\cdot \prod_{m=1}^k \left(1-Q_m(t-\tau-1)\right)
^{s_m^{(ij)}}\times
$$
$$
\left[\prod_{m=1}^k \left(1-Q_m(t-\tau)\right)^{x_m}-
\frac{1-Q_i(t-\tau)}{1-Q_i(t-\tau-1)}\times \right.
$$
$$
\left. \prod_{m=1}^k \left(\left(1-Q_m(t-\tau-1)\right)^{x_m} \cdot
\frac{\left(1-Q_m(t-\tau-2)\right)^{s_m^{(ij)}}}
{\left(1-Q_m(t-\tau-1)\right)^{s_m^{(ij)}}}\right)\right].
$$
Очевидно, что
$$
\frac{1-Q_i(t-\tau)}{1-Q_i(t-\tau-1)}=
1+\frac{Q_i(t-\tau-1)-Q_i(t-\tau)}{1-Q_i(t-\tau-1)}=
$$
$$
1+\frac{P(D_i^{t-\tau})}{1-Q_i(t-\tau-1)}=
1+P(D_i^{t-\tau})+\frac{P(D_i^{t-\tau})\cdot Q_i(t-\tau-1) }{1-Q_i(t-\tau-1)}.
$$

Применим теорему 2 и следствие из нее для оценки $Q_i(t-\tau-1)$ и $P(D_i^{t-\tau})$.
Получим, что
$$
\frac{1-Q_i(t-\tau)}{1-Q_i(t-\tau-1)}= 1+P(D_i^{t-\tau})+O((t-\tau)^{2(q_1-1)}\cdot r^{2(t-\tau)}).
$$

Проводя аналогичные преобразования и учитывая, что $s_m^{(ij)}$ -- константа,
определяемая правой частью правила $r_{ij},$ мы можем записать, что
$$
\prod_{m=1}^k \frac{(1-Q_m(t-\tau-2))^{s_m^{(ij)}}}
{(1-Q_m(t-\tau-1))^{s_m^{(ij)}}}=
%\prod_{m=1}^k\left(1+
%\frac{P(D_m^{t-\tau-1})}
%{1-Q_m(t-\tau-1)}\right)^{s_m^{(ij)}}=
%$$
%$$
1-\sum_{m=1}^k s_m^{(ij)}\cdot P\left(D_m^{t-\tau-1}\right)+
O\left((t-\tau)^{2(q_1-1)}\cdot r^{2(t-\tau)}\right).
$$
Поэтому
$$
P\left(D^t_{X,1}(\tau)\right)=
p_{ij} \cdot P_X(\tau) \cdot
\left(1+O\left((t-\tau)^{q_1-1}\cdot r^{t-\tau}\right)\right)
\left[\prod_{m=1}^k \left(1-Q_m(t-\tau)\right)^{x_m}- \right.
$$
$$
\prod_{m=1}^k \left(1-Q_m(t-\tau-1)\right)^{x_m} \cdot
\left(1+P(D_i^{t-\tau})+O((t-\tau)^{2(q_1-1)}\cdot r^{2(t-\tau)})\right) \times
$$
$$
\left.\left(1-\sum_{m=1}^k s_m^{(ij)}\cdot P\left(D_m^{t-\tau-1}\right)+
O\left((t-\tau)^{2(q_1-1)}\cdot r^{2(t-\tau)}\right)\right)\right]=
$$
$$
p_{ij} \cdot P_X(\tau) \left[ R_X(t-\tau) +
%p_{ij} \cdot P_X(\tau)
 \prod_{m=1}^k \left(1-Q_m(t-\tau-1)\right)^{x_m} \times
\right.
$$
$$
\left.\left(\sum_{m=1}^k s_m^{(ij)}P(D_m^{t-\tau-1})-P(D_i^{t-\tau}) \right)\right]
\cdot
\left(1+O\left((t-\tau)^{q_1-1}\cdot r^{t-\tau}\right)\right).
$$
(Здесь $R_X(t-\tau)$ -- величина, рассмотренная в лемме 4.)

Вернемся к вычислению $M_{ij}(t,\tau),$
учитывая оценку $$
\prod_{m=1}^k \left(1-Q_m(t-\tau-1)\right)^{x_m}=
1- O\left((t-\tau)^{q_1-1}\cdot r^{t-\tau}\sum_{m=1}^k x_m\right),
$$
следующую из (\ref{50}).
Тогда
$$
M_{ij}(t,\tau)=
\frac{1}{P(D_1^t)} \left[\sum_{X \ne 0} p_{ij} \cdot P_X(\tau) \cdot R_X(t-\tau)\cdot x_i+ \right.
\left(\sum_{m=1}^k s_m^{(ij)} P\left(D_m^{t-\tau-1}\right)-
P\left(D_i^{t-\tau}\right)\right)\times
$$
$$
\left.\sum_{X \ne 0} p_{ij} \cdot P_X(\tau) \cdot x_i
\cdot \left(1- O\left((t-\tau)^{q_1-1}\cdot r^{t-\tau}\sum_{m=1}^k x_m\right)\right)\right]\cdot
\left(1+O\left((t-\tau)^{q_1-1}\cdot r^{t-\tau}\right)\right).
$$
Величина
$$
\frac{1}{P(D_1^t)} \cdot
\sum_{X \ne 0} P_X(\tau) \cdot R_X(t-\tau)\cdot x_i
$$
есть $M_i(t,\tau)$ из теоремы 3, и
$$
\sum_{X \ne 0} P_X(\tau) \cdot x_i=
a_i^1(\tau)=c_{1i}\cdot \tau^{s_{1i}'-1}r^{\tau}(1+o(1)),
$$
где $a_i^1 (\tau)$ -- элемент матрицы $A^{\tau},$
$A$ -- матрица первых моментов, и
%$c_{1i}$ и
$s_{1i}'$ имеет тот же смысл, что и в теореме 3.

Кроме того,
$$
\sum_{X \ne 0} P_X(\tau) \cdot x_i \cdot O\left((t-\tau)^{q_1-1}\cdot r^{t-\tau}\sum_{m=1}^k x_m\right)=
$$
$$
O\left((t-\tau)^{q_1-1}\cdot r^{t-\tau}\right)\sum_{m=1}^k b^1_{im}(\tau)=
O\left(\tau^{q_1-1} \cdot (t-\tau)^{q_1-1 }r^{t}\right),
$$
где $b^1_{im}(\tau)$ -- вторые моменты.
Следовательно,
$$
M_{ij}(t,\tau)= \left(M_i(t,\tau)\cdot p_{ij} +
\frac{p_{ij} \cdot c_{1i}\cdot \tau^{s_{1i}'-1}r^{\tau}\cdot (1+o(1)) }{P(D_1^t)} \times \right.
$$
$$
\left. \left(\sum_{m=1}^k s_m^{(ij)}\cdot
P(D_m^{t-\tau-1})- P(D_i^{t-\tau}) \right)\right)
\cdot \left(1+ O\left((t-\tau)^{q_1-1} r^{t-\tau}\right)\right).
$$

Применяя теорему 3 к $M_i(t,\tau)$ и формулу (\ref{24}) к $P(D_m^{n}),$ после проведения несложных преобразований
$M_{ij}(t,\tau)$ можем представить в следующем виде:
$$
M_{ij}(t,\tau)=
\frac{p_{ij}}{t^{q_1-1}} \left(\sum_{l =1}^k f_{il} \cdot (t-\tau)^{q_l'-1}\cdot \tau^{\delta_{il}^1}+
\frac{1}{{r }} \sum_{m=1}^k f_m\cdot s_m^{(ij)}\cdot
 (t-\tau)^{q_m'-1}\tau^{s_{1i}'-1}\right)+
$$
$$
\xi^1_{ij}(t)+\xi^2_{ij}(\tau)+\xi^3_{ij}(t-\tau),
$$
где $\xi^1_{ij}(t)=o(1)$, $\xi^2_{ij}(\tau)=o(1)$ и
$\xi^3_{ij}(t-\tau)=o(1).$

Обозначим сумму $\xi^1_{ij}(t)+\xi^2_{ij}(\tau)+\xi^3_{ij}(t-\tau)$ через
$\xi_{ij}(t,\tau).$
Очевидно, $\xi_{ij}(t,\tau) \rightarrow 0$ при
$\tau \rightarrow \infty$ и $t-\tau \rightarrow \infty.$
Поэтому
$$
M_{ij}(t,\tau) =
$$
%$$
\begin{equation}
\frac{p_{ij}}{t^{q_1-1}} \left(\sum_{l =1}^k f_{il} \cdot (t-\tau)^{q_l'-1}\cdot \tau^{\delta_{il}^1}+
\frac{1}{{r }} \sum_{m=1}^k f_m\cdot s_m^{(ij)}\cdot
 (t-\tau)^{q_m'-1}\tau^{s_{1i}'-1}\right)+ \xi_{ij}(t,\tau).
%$$
\label{34}
\end{equation}
Теорема доказана.

\medskip

Сделаем несколько выводов из теоремы 4.

1. $M_{ij}(t,\tau)$ ограничено константой
при $\tau \rightarrow \infty,$ $t-\tau \rightarrow \infty.$

2. Величина $\sum_{m=1}^k f_m \cdot s_m^{(ij)}\cdot
 (t-\tau)^{q_m'-1}\tau^{s_{1i}'-1}$ имеет большее значение для тех правил, которые
содержат в правой части большее количество нетерминальных символов.

3. Величина $\sum_{l \in N_i} f_{il} \cdot (t-\tau)^{q_l'-1}\cdot \tau^{\delta_{il}^1}$ имеет одно и то же значение для всех
правил грамматики с одинаковой левой частью $A_i.$

\medskip

Пусть $S_{ij}(t)=q_{ij}(t,0)+q_{ij}(t,1)+\ldots +q_{ij}(t,t-1),$
где $q_{ij}(t,\tau)$ --- число правил $r_{ij}$ на ярусе $\tau$ в дереве
из $D_1^t$; $S_{ij}(t)$ --- число правил $r_{ij}$ в дереве вывода из $D_1^t$.

Рассмотрим случайную величину $\frac{S_{ij}(t)}{t}$ --- среднее число правил $r_{ij}$,
приходящееся на один ярус дерева вывода из $D_1^t.$

\medskip

{\textbf Теорема 5.}
{\em Пусть $G$ --- стохастическая КС-грамматика с разложимой
матрицей первых моментов, для которой перронов корень $r<1$ , и $D^t_1$ -- множество деревьев вывода
высоты $t.$

Тогда при $t \rightarrow \infty $ выполняется следующее асимптотическое равенство:
$$
M\left(\frac{S_{ij}(t)}{t}\right) \sim w_{ij},
$$
где $w_{ij}$ - константа, определяемая грамматикой $G.$}
\medskip

{\bf Доказательство.}

Разобьем $S_{ij}(t)$ на три части:
$$
S_{ij}(t)=S_{ij}^{(1)}(t)+S_{ij}^{(2)}(t)+S_{ij}^{(3)}(t),
$$
где
$$
S_{ij}^{(1)}(t)=q_{ij}(t,0)+\ldots+q_{ij}(t,\tau_0-1),
$$
$$
S_{ij}^{(2)}(t)=q_{ij}(t,\tau_0)+\ldots+q_{ij}(t,t-\tau_0-1),
$$
$$
S_{ij}^{(3)}(t)=q_{ij}(t,t-\tau_0)+\ldots+q_{ij}(t,t-1),
$$
и положим $\tau_0=\lfloor \log\log t \rfloor$ (здесь и далее логарифм
берется по основанию 2).
Число слагаемых в $S_{ij}^{(1)}(t)$ и в $S_{ij}^{(3)}(t)$ равно
$\lfloor \log\log t \rfloor,$ а в $S_{ij}^{(2)}(t)$ равно
$t-2 \lfloor \log\log t \rfloor.$

Найдем математические ожидания $M\left(S_{ij}^{(1)}(t)\right),$
$M\left(S_{ij}^{(2)}(t)\right)$ и
$M\left(S_{ij}^{(3)}(t)\right).$

Величину $M\left(S_{ij}^{(1)}(t)\right)$ можно представить в следующем виде:
$$
M\left(S_{ij}^{(1)}(t)\right)=M_{ij}(t,0)+M_{ij}(t,1)+\ldots +
M_{ij}(t,\tau_0-1).
$$

Число правил $r_{ij}$ на ярусе $\tau$ в дереве из $D_1^t$ обозначим $q_{ij}(t,\tau).$
Оценим $q_{ij}(t,\tau)$ для $\tau < \tau_0.$
Обозначим через $k_{max}$ максимальное число нетерминалов в~правой части правил грамматики $G.$ Тогда
$q_{ij}(t,\tau)\leq k_{max}^{\tau}< k_{max}^{\tau_0}.$
Поэтому
$
M_{ij}(t,\tau)< k_{max}^{\tau_0}
$
и
$$
M\left(S_{ij}^{(1)}(t)\right)\leq k_{max}^{\tau_0}  \tau_0\leq
k_{max}^{\log \log t}
  \log \log t=\log^{c_1} t \,  \log \log t \leq \log^{c_2} t,
$$
где $c_1=\log k_{max},$ $\, c_2=c_1+1.$

Для $t-\tau_0 \leq \tau < t$ имеем:
$$
M_{ij}(t,\tau)\leq M_{i}(t,\tau) =
\frac 1 {P(D^t)} \sum_{X} P_X(\tau)   R_X(t-\tau)   x_i \leq
\frac{1}{P(D^t)} \sum_{X} P_X(\tau)   x_i=
$$
$$
\frac{1}{P(D^t)}  a^1_{i}(\tau)\le
O\left(\frac{\tau^{q_1-1}}{t^{q_1-1}\cdot r^{t-\tau}}\right)\le O\left(\frac 1{r^{t-\tau}}\right).
$$
Поэтому
$$
M\left(S_{ij}^{(3)}(t)\right) \leq
\sum_{t-\tau_0}^{t-1} O\left(\frac{1}{r^{t-\tau}}\right)
=O\left(\frac {\tau_0}{r^{\tau_0}}\right)
=O\left(\frac {\log \log t}{r^{\log \log t}}\right)=O\left(\log ^{c_3} t\right)
$$
для некоторой константы $c_3>0.$

Для $\tau,$ удовлетворяющего условию
$\tau_0 \leq \tau \leq t-\tau_0-1,$ применим теорему 4:
$$
M\left(S_{ij}^{(2)}(t)\right)=
\sum_{\tau=\lfloor \log\log t \rfloor}^{t-\lfloor \log\log t \rfloor-1}
\frac{p_{ij}}{t^{q_1-1}} \left(\sum_{l =1}^k f_{il} \cdot (t-\tau)^{q_l'-1}\cdot \tau^{\delta_{il}^1}+ \right.
$$
$$
\left. \frac{1}{{r }} \sum_{m=1}^k f_m\cdot s_m^{(ij)}\cdot
(t-\tau)^{q_m'-1}\tau^{s_{1i}'-1}\right)+
\sum_{\tau=\lfloor \log\log t \rfloor}^{t-\lfloor \log\log t \rfloor-1} \xi(t,\tau).
$$

Оценим величину
$
\delta=\frac{1}{t^{n_1+n_2}}\cdot \sum_{\tau=\lfloor \log\log t \rfloor}^{t-\lfloor \log\log t \rfloor-1}
(t-\tau)^{n_1}\cdot \tau^{n_2}:
$
$$
\delta=
\sum_{\tau=\lfloor \log\log t \rfloor}^{t-\lfloor \log\log t \rfloor-1} \left(1-\frac{\tau}{t}\right)^{n_1} \left(\frac{\tau}{t}\right)^{n_2}=
$$
$$
\sum_{\tau=\lfloor \log\log t \rfloor}^{t-\lfloor \log\log t \rfloor-1}
\sum_{n=0}^{n_1} (-1)^{n} C_{n_1}^n \left( \frac{\tau}{t} \right)^{n+n_2} =
\left(\sum_{n=0}^{n_1} (-1)^{n} C_{n_1-1}^n \cdot \frac{t}{n+n_2+1}  \right)\cdot (1+o(1)).
$$
Очевидно, величина $\sum_{n=0}^{n_1} (-1)^{n} C_{n_1-1}^n \cdot \frac{1}{n+n_2+1}$ является константой, зависящей от $n_1$ и $n_2,$ обозначим ее $\alpha(n_1,n_2)$.
Применяя обозначение $\alpha(n_1,n_2),$ мы можем записать:
$$
\delta=\alpha(n_1,n_2)\cdot t\cdot (1+o(1)).
$$
Применим полученную оценку к вычислению $M\left(S_{ij}^{(2)}(t)\right),$ учитывая равенства
$q_l'~+~\delta_{il}^1~=~q_1$ и $q_i'+s'_{1i}-1=q_1:$
$$
M\left(S_{ij}^{(2)}(t)\right)=p_{ij}\cdot
\left[\sum_{l =1}^k f_{il} \cdot \alpha(q_l'-1,\delta_{il}^1)+
\frac{1}{{r }} \sum_{m=1}^k f_m\cdot s_m^{(ij)}\cdot
\alpha(q_m'-1,s_{1i}'-1)\right]t \cdot(1+o(1)).
$$
Константу в квадратных скобках обозначим $w_{ij}.$

Применяя полученные оценки для $M\left(S_{ij}^{(1)}(t)\right),$
$M\left(S_{ij}^{(2)}(t)\right)$ и
$M\left(S_{ij}^{(3)}(t)\right),$ находим, что при $t \rightarrow \infty$
$$
M\left(\frac{S_{ij}(t)}{t}\right)=w_{ij}+o(1)+
O\left(\frac{\log^{c_2} t}{t}\right) +
O\left(\frac{\log ^{c_3} t}{t}\right)=w_{ij}+o(1).
$$

Теорема доказана.

\medskip

\section {
Энтропия и нижняя оценка стоимости кодирования}

Пусть $L$ - стохастический язык, т.е. язык, на множестве слов которого задано распределение вероятностей.

%Пусть $G$ - стохастическая КС-грамматика с однозначным выводом, т.е. грамматика, в которой каждое слово порождаемого языка $L$ имеет единственное дерево вывода.
Под энтропией стохастического языка $L$ будем понимать величину
$$
H(L)=-\lim_{N \rightarrow \infty} \sum_{\alpha \in L, \left|\alpha\right|\le N} p(\alpha) \log p(\alpha).
$$
Если энтропия конечна, будем применять запись $H(L)=-\sum_{\alpha \in L} p(\alpha) \log p(\alpha).$

Кодированием языка $L$ назовем инъективное отображение
$$
f:  L \rightarrow \{ 0,1 \}^+.
$$

В качестве $L$ рассмотрим язык, порождаемый стохастической КС-грамматикой с однозначным выводом, т.е. грамматикой, в которой каждое слово из $L$ имеет единственное дерево вывода.
Через $L^t$ обозначим множество всех слов из $L,$ каждое из которых имеет дерево вывода высоты $t.$ Для $\alpha \in L^t$ через $p_t(\alpha)$
обозначим условную вероятность появления слова $\alpha,$ т.е.
$p_t(\alpha)=\frac{p(\alpha)}{P(L^t)}.$
В силу однозначности вывода $P(L^t)=P(D^t_1).$

Стоимостью кодирования $f$ назовем величину
\begin{equation}
C(L,f)= \lim_{t \rightarrow \infty}
 \frac {\sum_{ \alpha \in L^t} p_t(\alpha) \cdot
 |f(\alpha)|}
 {\sum_{ \alpha \in L^t} p_t(\alpha) \cdot | \alpha|}
\label{40}
\end{equation}
(здесь $|x|$ -длина последовательности $x$).

Величина $C(L,f)$ характеризует число двоичных разрядов,
приходящихся на кодирование одного символа слова языка.

Через $F(L)$ обозначим класс всех
инъективных отображений из $L$ в $ \{ 0,1 \}^+,$ для которых
существует $C(L,f).$

Стоимостью оптимального кодирования языка $L$ назовем величину
$$
C_0(L)= \inf_{f \in F(L)} C(L,f).
$$

Предварительно получим асимптотическую формулу для энтропии множества слов $L^t.$ По определению имеем
$$
H(L^t)=-\sum_{\alpha \in L^t} p_t(\alpha) \log p_t(\alpha).
$$
Следовательно,
$$
H(L^t)=-\sum_{\alpha \in L^t} p_t(\alpha)\left( \log p(\alpha)-\log P( L^t)\right)=
$$
$$
\frac{1}{P(L^t)}\cdot \left( -\sum_{\alpha \in L^t} p(\alpha) \log p(\alpha) \right)+\log P(L^t).
$$

Для слова $\alpha$ обозначим через $q_{ij}(\alpha)$ число применений правила $r_{ij}$ при его выводе. Вероятность слова $\alpha$ равна
$p(\alpha)=\prod_{i=1}^k \prod_{j=1}^{n_i} (p_{ij})^{q_{ij}}.$ Следовательно, $\log p(\alpha)= \sum_{i=1}^k \sum_{j=1}^{n_i} q_{ij}(\alpha) \log p_{ij}.$ Поэтому
$$
H(L^t) = \frac{1}{P(L^t)} \cdot \left( -\sum_{\alpha \in L^t} p(\alpha) \cdot \sum_{i=1}^k \sum_{j=1}^{n_i} q_{ij}(\alpha) \log p_{ij} \right)+ \log P(L^t)=
$$
$$
\frac{1}{P(L^t)} \cdot \left( - \sum_{i=1}^k \sum_{j=1}^{n_i} \log p_{ij} \cdot \sum_{\alpha \in L^t} p(\alpha) q_{ij}(\alpha)  \right)+
\log P(L^t).
$$
Очевидно, что $\sum_{\alpha \in L^t} p(\alpha) q_{ij}(\alpha) = {P(L^t)}\cdot M(S_{ij}(t)).$
Используя теорему 5, выражение для энтропии можно переписать в виде
$$
H(L^t) = -t\cdot (1+o(1))\sum_{i=1}^k \sum_{j=1}^{n_i} w_{ij}\log p_{ij} +
\log P(L^t).
$$
Ввиду однозначности вывода, с использованием (\ref{24}), имеем
$$
\log P(L^t)=\log P(D^t_1)=t \log r+O(\log t).
$$
Поэтому
$$
H(L^t) =t \cdot \left( \log r - \sum_{j=1}^{n_i} w_{ij}\log p_{ij}\right)+o(t).
$$
Полученный результат сформулируем в виде следующей теоремы.

\medskip

\textbf {Теорема 6.}
{\em Пусть $G$ --- однозначная стохастическая КС-грамматика с разложимой
матрицей первых моментов, для которой перронов корень $r<1,$ и
$L^t$ - множество всех слов из $L,$ порождаемого $G,$ с деревьями вывода высоты $t.$ Тогда
$$
H(L^t) =t \cdot \left( \log r - \sum_{j=1}^{n_i} w_{ij}\log p_{ij}\right)+o(t),
$$
где $w_{ij}$ определяются теоремой $5$.
}

\medskip

Таким образом, энтропия $H(L^t)$ линейно зависит от высоты $t$ дерева вывода, как и в неразложимом случае \cite{zhil2}.

Используя энтропию, оценим стоимость оптимального кодирования $C_0(L).$
Обозначим через $f^*$ кодирование множества $L^t,$ минимизирующее величину
$$
M_t(f)=\sum_{\alpha \in L^t} p_t(\alpha)\cdot \left|f(\alpha)\right|.
$$
Очевидно, для любого кодирования $f \in F(L)$ верно неравенство $M_t(f) \ge M_t(f^*).$ Оценим  $M^*(L^t)=M_t(f^*),$ используя следующую теорему, доказанную в \cite{bor}.

\medskip

\textbf {Теорема 7.}
{\em Пусть $L_k$ -- последовательность стохастических языков, для которой $H(L_k) \rightarrow \infty$ при $k \rightarrow \infty.$ Тогда
$$
\lim_{k \rightarrow \infty} \frac{M^*(L_k)}{H(L_k)}=1.
$$
}
\medskip

Поскольку  $H(L)^t \rightarrow \infty$ при $t \rightarrow \infty,$ из теоремы 7 следует, что $M_t(f^*)/H(L^t) \rightarrow 1$ при $t \rightarrow \infty.$

Найдем величину $\sum_{ \alpha \in L^t} p_t(\alpha) \cdot | \alpha|.$ Пусть правило $r_{ij}$ содержит в правой части $l_{ij}$ терминальных символов. Очевидно, $\left|\alpha\right|=\sum_{ij} q_{ij}(\alpha)\cdot l_{ij}.$ Поэтому
$$
\sum_{ \alpha \in L^t} p_t(\alpha) \cdot | \alpha|=\sum_{ij} l_{ij} M(S_{ij}(t))= t\cdot \sum_{ij} l_{ij} w_{ij} +o(t).
$$
Следовательно, справедлива

\medskip

{\bf Теорема 8.}
{\em
Пусть $L$ - стохастический КС-язык, порожденный разложимой стохастической КС~-~грамматикой с однозначным выводом, для которой перронов корень $r$ матрицы первых моментов меньше $1$. Тогда стоимость любого кодирования $f \in F(L)$ удовлетворяет неравенству
$$
C(L,f) \ge C_0(L)= \frac {\log r-\sum_{ij} w_{ij}\log p_{ij}}{\sum_{ij} l_{ij} w_{ij}}.
$$
}

\medskip
}

\begin{thebibliography}{8}

\bibitem{aho} Ахо А., Ульман Дж. Теория синтаксического анализа,
перевода и компиляции. Том 1. --- М.: Мир, 1978.

\bibitem{bor} Борисов А.Е. Кодирование слов стохастического КС-языка, порожденного разложимой грамматикой
с двумя нетерминалами // Вестник Нижегородского университета им. Н.И. Лобачевского. Серия Математика, 2004. --- Выпуск 1(2). --- С.18 -- 28.

\bibitem{gant} Гантмахер Ф. Р. Теория матриц. --- М.: Наука, 1967.

\bibitem{zhil1} Жильцова Л.П. Закономерности применения правил грамматики
в выводах слов стохастического контекстно-свободного языка //
Математические вопросы кибернетики. --- М.: Наука. --- 2000. --- Вып.9.
--- С. 101 -- 126.

\bibitem{zhil2} Жильцова Л.П. О нижней оценке стоимости кодирования и
асимптотически оптимальном кодировании стохастического
контекстно-свободного языка // Дискретный анализ и исследование операций.
--- 2001.--- Серия 1. --- Том 8, N3. ---
Новосибирск: Издательство Института математики СО РАН. --- С. 26 -- 45.

\bibitem{zhil3} Л.П. Жильцова. О матрице первых моментов разложимой стохастической КС-грамматики // Ученые записки Казанского государственного университета. Физико-математические науки. --- Том 151, --- книга 2, --- 2009. --- С. 80 -- 89.

\bibitem{zhil4} Л.П. Жильцова. О вероятностях продолжения деревьев вывода в разложимых стохастических КС-грамматиках. Докритический случай // Вестник Нижегородского университета им. Н.И. Лобачевского,  ---№ 4, ---2012, № 4,. ---С. 217 -- 224.

\bibitem{sev} Севастьянов В. А. Ветвящиеся процессы. --- М.: Наука, 1971.

\bibitem{fu} Фу К. Структурные методы в распознавании образов.
--- М.: Мир, 1977.

\end{thebibliography}

}
}
\end{document}
