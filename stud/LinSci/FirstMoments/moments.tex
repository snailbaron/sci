\documentclass[11pt]{article}
\usepackage[T2A]{fontenc}
\usepackage[utf8]{inputenc}
\usepackage[russian]{babel}
\usepackage{amsmath}
\usepackage{amssymb}

\oddsidemargin=0pt
\textwidth=6.5in
\topmargin=0pt
\headheight=0pt
\headsep=0pt
\textheight=9.5in

\newtheorem{theorem}{Теорема}

\title{Вид матрицы первых моментов для грамматик в виде <<цепочки>>}
\author{Игорь Мартынов}
\begin{document}

\setlength{\parindent}{0pt}
\setlength{\parskip}{8pt}

\maketitle

\section{Основные определения}

\textit{Стохастической КС-грамматикой} называется система $G = <V_T, V_N, R, s>$, где $V_T$ и $V_N$ --- алфавиты терминальных и нетерминальных символов соответственно, $s$ --- аксиома грамматики, $R$ --- множество правил вывода, представимое в виде $R = \cup_{i = 1}^k R_i$, где $k = |V_N|$, и $R_i$ --- множество правил вида
\begin{equation}
    r_{ij} : A_i \xrightarrow{p_{ij}} \beta_{ij}\quad{}\left( A_i \in V_N, \beta_{ij} \in (V_N \cup V_T)^* \right),
\end{equation}
и $p_{ij}$ --- вероятность применения правила $r_{ij}$, причём при фиксированном $i$ вероятности $r_{ij}$ задают вероятностное распределение на множестве $R_i$:
\begin{equation}
    0 < p_{ij} \le 1\quad\text{и}\quad\sum_{j = 1}^{n_i} p_{ij} = 1,\qquad{}i = 1,2,\ldots,k,
\end{equation}
где $n_i = |R_i|$.

Слово $\beta$ называется \textit{непосредственно выводимым} из $\alpha$ (обозначается $\alpha \Rightarrow \beta$), если существуют $\alpha_1, \alpha_2 \in (V_N \cup V_N)^*$, для которых $\alpha = \alpha_1 A_i \alpha_2$, $\beta = \alpha_1 \beta_{ij} \alpha_2$ и в $R$ имеется правило $A_i \xrightarrow{p_{ij}} \beta_{ij}$.

Через $\Rightarrow_*$ обозначим рефлексивное транзитивное замыкание $\Rightarrow$. Если $\alpha \Rightarrow_* \beta$, говорят, что $\beta$ \textit{выводимо} из $\alpha$. Язык, \textit{порождаемый} грамматикой $G$ определяется как $L_G = \left\{ \alpha : s \Rightarrow_* \alpha, \alpha \in V_T^* \right\}$.

Последовательность правил грамматики $\omega(\alpha) = (r_1, r_2, \ldots, r_\gamma)$, последовательное применение которых к $s$ даёт слово $\alpha$, называется \textit{выводом} этого слова. Если на каждом шаге правило применяется к самому левому нетерминалу в слове, вывод назвается \textit{левым}.

Вероятность вывода определяется как $p(\omega(\alpha)) = p(r_1) \cdot p(r_2) \cdot \ldots \cdot p(r_\gamma)$, где $p(r_i)$ --- вероятность соответствующего правила. Вероятность слова определяется как сумма вероятностей всех его левых выводов.

Грамматика $G$ называется \textit{согласованной}, если
\begin{equation}
    \sum_{\alpha \in L_G} p(\alpha) = 1.
\end{equation}
Согласованная грамматика $G$ задаёт распределение вероятностей $P$ на $L_G$, и определяет \textit{стохастический КС-язык} $\mathfrak{L} = (L, P)$. В дальнейшем всюду предполагается, что грамматика согласованна.

По выводу слова может быть построено \textit{дерево вывода}. В узел дерева помещается аксиома $s$, далее на каждом ярусе дерева ко всем нетерминалам этого яруса применяется правило, соответствующее выводу. Символы этого слова записываются слева направо в дереве, присоединяясь к исходному нетерминалу как к родителю.

Для исследования вероятностных характеристик стохастической грамматики применяются производящие функции
\begin{equation}
    F_i(s_1, s_2, \ldots, s_k) = \sum_{\substack{j = 1\\r_{ij} \in R}}^{n_i} p_{ij} s_1^{l_1} s_2^{l_2} \ldots s_k^{l_k},
\end{equation}
где $l_m = l_m(i,j)$ --- число вхождений нетерминала $A_m$ в $\beta_{ij}$.

Величины
\begin{equation}
    a^i_j = \left. \frac{\partial F_i(s_1, s_2, \ldots, s_k)}{\partial s_j} \right|_{s_1 = s_2 = \ldots = s_k = 1}
\end{equation}
называются \textit{первыми моментами} грамматики $G$. Матрица $A = (a^i_j)$, составленная из них, называется \textit{матрицей первых моментов} грамматики $G$.


Матрица $A$, по построению, неотрицательна. По теореме Фробениуса,
% TODO: нужна ссылка
существует максимальный по модулю вещественный неотрицательный собственный корень $r$. Известно, что критерием согласованности стохастической КС-грамматики при отсутствии бесполезных нетерминалов является условие $r \le 1$.

Говорят, что нетерминал $A_j$ \textit{непосредственно следует} за нетерминалом $A_i$ (обозначается $A_i \rightarrow A_j$), если в $R$ имеется правило $A_i \xrightarrow(p_{ij}) \alpha_1 A_j \alpha_2$, где $\alpha_1, \alpha_2 \in (V_N \cup V_T)^*$. Транзитивное замыкание отношения $\rightarrow$ обозначается $\rightarrow_*$. Если $A_i \rightarrow_* A_j$, говорят, что $A_j$ \textit{выводится} из $A_i$.

Введём также отношение $\leftrightarrow_*$. Будем считать, что $A_i \leftrightarrow_* A_j$, если одновременно $A_i \rightarrow_* A_j$ и $A_j \rightarrow_* A_i$, либо если $A_i = A_j$. Очевидно, отношение $\leftrightarrow_*$ есть отношение эквивалентности, и потому разбивает множество нетерминалов на классы $V_N = K_1 \cup K_2 \cup \ldots \cup K_m : K_i \cap K_j = \varnothing (i \neq j)$. Класс, содержащий ровно один нетерминал, будем называть \textit{особым}. Множество классов $\{K_1, K_2, \ldots, K_m\}$ обозначим $\mathfrak{K}$.

Если все нетерминалы грамматики образуют один класс, она называется \textit{неразложимой}. В противном случае она называется \textit{разложимой}. Очевидно, разложимой грамматике соответствует разложимая матрица первых моментов.

Говорят, что класс $K_j$ \textit{непосредственно следует} за классом $K_i$ (обозначается $K_i \prec K_j$), если существуют $A_1 \in K_i$ и $A_2 \in K_j$ такие, что $A_1 \rightarrow A_2$. Рефлексивное транзитивное замыкание $\prec$ обозначим $\prec_*$, и назовём отношением \textit{следования}.

Будем говорить, что грамматика имеет вид \textit{<<цепочки>>}, если она разложима, и граф, построенный на множестве $\mathfrak{K}$ по отношению $\prec$, имеет вид $P_m$. Пронумеруем классы грамматики таким образом, что $K_i \prec K_{i+1}, i = 1,2,\ldots,m-1$. Пронумеруем нетерминалы так, что для любых $A_i \in K_p$ и $A_j \in K_q$ $i < j \Leftrightarrow p < q$. После этого матрица первых моментов грамматики приобретает вид:
\begin{equation}
    A = 
    \begin{pmatrix}
        A_{11} & A_{12} & 0 & \cdots & 0 & 0 \\
        0 & A_{22} & A_{23} & \cdots & 0 & 0 \\
        \vdots & \vdots & \vdots & \ddots & \vdots & \vdots \\
        0 & 0 & 0 & \cdots & A_{m-1,m-1} & A_{m-1,m} \\
        0 & 0 & 0 & \cdots & 0 & A_{m,m}
    \end{pmatrix}
\end{equation}

Блоки $A_{i,i} (i = 1,2,\ldots,m)$ являются неразложимыми неотрицательными матрицами. Не уменьшая общности, будем считать их положительными и непериодичными. Этого можно добиться с помощью метода укрупнения правил грамматики. Пусть $r_i$ --- перронов корень матрицы $A_{i,i}$. По построению матрицы $A$, $r = \max_i \{r_i\}$ и $r > 0$.

\section{Свойства матрицы первых моментов}

Обозначим $J = \{ i : r_i = r \} = \{ i_1 < i_2 < \ldots < i_q \}$ Разобьём множество классов $\mathfrak{K}$ на группы классов $\mathfrak{M_1}, \mathfrak{M_2}, \ldots, \mathfrak{M_w}$. При этом $\mathfrak{M_1} = \{K_1, K_2, \ldots, K_{i_1} \}$, и $\mathfrak{M_l} = \{ K_{i_{l-1} + 1}, \ldots, K_{i_l} \}$, где $l > 1$. При таком разбиении в каждой группе $\mathfrak{M_j}$ содержится ровно один класс с номером из $J$.

Тогда матрицу первых моментов можно представить в виде
\begin{equation}
    A = 
    \begin{pmatrix}
        B_{11} & B_{12} & \cdots & 0 & 0 \\
        0 & B_{22} & \cdots & 0 & 0 \\
        \vdots & \vdots & \ddots & \vdots & \vdots \\
        0 & 0 & \cdots & B_{w-1,w-1} & B_{w-1,w} \\
        0 & 0 & \cdots & 0 & B_{w,w}
    \end{pmatrix},
\end{equation}
где $B_{ij}$ --- блок, находящийся на пересечении строк, соответствующих нетерминалам классов группы $\mathfrak{M_i}$, и столбцов, соответствующим нетерминалам классов группы $\mathfrak{M_j}$. Очевидно, каждой из матриц $B_{i,i}$ соответствует перронов корень равный $r$.

Рассмотрим матрицу
\begin{equation}
    A^t = 
    \begin{pmatrix}
        B_{11}^t & B_{12}^{(t)} & \cdots & B_{1,w-1}^{(t)} & B_{1,w}^{(t)} \\
        0 & B_{22}^t & \cdots & B_{2,w-1}^{(t)} & B_{2,w}^{(t)} \\
        \vdots & \vdots & \ddots & \vdots & \vdots \\
        0 & 0 & \cdots & B_{w-1,w-1}^t & B_{w-1,w}^{(t)} \\
        0 & 0 & \cdots & 0 & B_{w,w}^t
    \end{pmatrix}.
\end{equation}
Для установления её вида требуется определить вид блоков $B_{i,j}^{(t)}$ при $j > i$.

Рассмотрим блок $B_{11}$. Разобьём группу $\mathfrak{M_1}$ на подгруппы $(\mathfrak{M_{11}}, \mathfrak{M_{12}}, \mathfrak{M_{13}})$. К группе $\mathfrak{M_{12}}$ отнесём класс с номером из $J$, к группе $\mathfrak{M_{11}}$ --- предшествующие ему классы, к группе $\mathfrak{M_{13}}$ --- последующие классы. В соответствии с таким разбиением $B_{11}$ принимает вид
\begin{equation}
    B = 
    \begin{pmatrix}
        C_{11} & C_{12} & 0 \\
        0 & C_{22} & C_{23} \\
        0 & 0 & C_{33}
    \end{pmatrix},
\end{equation}
А матрица $B^t$ представляется в виде
\begin{equation}
    B^t = 
    \begin{pmatrix}
        C_{11}^t & C_{12}^{(t)} & C_{13}^{(t)} \\
        0 & C_{22}^t & C_{23}^{(t)} \\
        0 & 0 & C_{33}^t
    \end{pmatrix}.
\end{equation}

Известно, что для неразложимой положительной матрицы $A$
\begin{equation}
    A^t = u v r^t (1 + o(1)),
\end{equation}
где $r$ --- перронов корень $A$, $u$ и $v$ --- соответственно правый и левый собственные векторы, соответствующие $r$, причём $u > 0$, $v > 0$, $vu = 1$.

Таким образом, асимптотика матриц $C_{11}^t$, $C_{22}^t$, $C_{33}^t$ известна.

Исследуем собственные векторы матрицы $B_{11}$, соответствующие числу $r$. Пусть $u = (u^{(1)}, u^{(2)}, u^{(3)})$ и $v = (v^{(1)}, v^{(2)}, v^{(3)})$ --- соответственно правый и левый такие собственные векторы. Тогда
\begin{equation}
    \begin{split}
        &C_{11} u^{(1)} + C_{12} u^{(2)} + C_{13} u^{(3)} = r u^{(1)} \\
        &C_{22} u^{(2)} + C_{23} u^{(3)} = r u^{(2)} \\
        &C_{33} u^{(3)} = r u^{(3)}
    \end{split}.
\end{equation}
Поскольку все собственные числа $C_{33}$ строго меньше $r$, $u^{(3)} = 0$ и $u^{(2)}$ --- правый собственный вектор $C_{22}$, относящийся к $r$, а $u^{(1)} = (rE - C_{11})^{-1} C_{12} u^{(2)}$.

Аналогично, рассматривая левый собственный вектор $v = (v^{(1)}, v^{(2)}, v^{(3)})$, имеем систему
\begin{equation}
    \begin{split}
        &v^{(1)} C_{11} = r v^{(1)} \\
        &v^{(1)} C_{12} + v^{(2)} C_{22} = r v^{(2)} \\
        &v^{(1)} C_{13} + v^{(2)} C_{23} + v^{(3)} C_{33} = r v^{(3)}
    \end{split},
\end{equation}
откуда $v^{(1)} = 0$, $v^{(2)}$ --- левый собственный вектор $C_{12}$, и $v^{(3)} = v^{(2)} C_{23} (rE - C_{33})^{-1}$.

Выберем именно такие $u$ и $v$, что $vu = 1$.

Рассмотрим асимптотику матрицы $B_{11}^t$. Нетрудно видеть, что
\begin{equation}
    C_{12}^{(t)} = \sum_{i + j = t-1} C_{11}^i C_{12} C_{22}^j.
\end{equation}
Разобьём эту сумму так, что $C_{12}^{(t)} = \Sigma_1 + \Sigma_2$, где
\begin{equation}
    \begin{split}
        &\Sigma_1 : \left\{
        \begin{split}
            &i = \overline{ t - \lfloor \log \log t \rfloor, t-1 } \\
            &j = \overline{ 0, \lfloor \log \log t \rfloor - 1 }
        \end{split} \right. \\
        &\Sigma_2 : \left\{ 
        \begin{split}
            &i = \overline{ 0, t - \lfloor \log \log t \rfloor - 2 } \\
            &j = \overline{ \lfloor \log \log t \rfloor + 1, t-1 }
        \end{split} \right.
    \end{split}
\end{equation}

Рассмотрим вначале $\Sigma_1$. $C_{22}^t = u_2 v_2 r^t (1 + o(1)) \le c_2$ при любых $t$. $C_{11}^t = u_1 v_1 (r')^t (1 + o(1)) \le c_1 (r')^{t - \lfloor \log \log t \rfloor - 1}$. Отсюда,
\begin{equation}
    \Sigma_1 \le с (r')^{t - \lfloor \log \log t \rfloor} \lfloor \log \log t \rfloor = O(\log \log t (\log t)^{c_3} (r')^t) = o(r^t).
\end{equation}

Для $\Sigma_2$ $C_{22}^j = u_2 v_2 r^j (1 + o(1))$, поэтому $\Sigma_2 = \sum_{i + j = t-1} C_{11}^i C_{12} H r^j (1 + o(1)) = r^{t - 1} \sum_{i = 0}^{t - \lfloor \log \log t \rfloor - 1} \left(\frac{C_{11}}{r}\right)^i C_{12} H (1 + o(1))$. Нетрудно видеть, что $\frac{1}{r} \sum_{i = 0}^{\infty} \left(\frac{C_{11}}{r}\right)^i = (rE - C_{11})^{-1}$. Матрица $(rE - C_{11})^{-1}$ существует, так как все собственные числа $C_{11}$ строго меньше $r$. Отсюда
\begin{equation}
    \Sigma_2 = r^t (rE - C_{11})^{-1} C_{12} u^{(2)} v^{(2)} (1 + o(1)) = C_{12}^{(t)}.
\end{equation}

Аналогично
\begin{equation}
    C_{23}^{(t)} = \sum_{i + j = t-1} C_{22}^i C_{23} C_{33}^j,
\end{equation}
откуда, проводя аналогичные вычисления, имеем
\begin{equation}
    C_{23}^{(t)} = r^t u^{(2)} v^{(2)} C_{23} (rE - C_{33})^{-1} (1 + o(1))
\end{equation}

Подстановкой проверяется, что
\begin{equation}
    C_{13}^{(t)} = \sum_{i + j = t-1} C_{12}^{(i)} C_{23} C_{33}^j
\end{equation}
Подставляя в это выражение асимптотику для $C_{12}^{(t)}$, получаем:
\begin{equation}
    C_{13}^{(t)} = r^t u^{(1)} v^{(2)} C_{23} (rE - C_{33})^{-1} (1 + o(1)) = r^t u^{(1)} v^{(3)} (1 + o(1))
\end{equation}

В результате, имеем:
\begin{equation}
    B_{11} = 
    \begin{pmatrix}
        0 & u^{(1)} v^{(2)} & u^{(1)} v^{(3)} \\
        0 & u^{(2)} v^{(2)} & u^{(2)} v^{(3)} \\
        0 & 0 & 0
    \end{pmatrix}
    r^t + o(r^t)
\end{equation}

Получим теперь асимптотику всей матрицы $A^t$. Вначале пусть $w = 2$. Тогда
\begin{equation}
    A^t = 
    \begin{pmatrix}
        B_{11}^t & B_{12}^{(t)} \\
        0 & B_{22}^t
    \end{pmatrix}
\end{equation}

Матрица $B_{22}^t$ исследуется аналогично $B_{11}^t$, в результате имеем
\begin{equation}
    B_{22}^t = 
    \begin{pmatrix}
        u^{(22)} v^{(22)} & u^{(22)} v^{(32)} \\
        0 & 0
    \end{pmatrix}
    r^t + o(r^t)
\end{equation}

Для $B_{12}^{(t)}$ имеем
\begin{equation}
    B_{12}^{(t)} = \sum_{i + j = t-1} B_{11}^i B_{12} B_{22}^j
\end{equation}
Подставляя выражения для $B_{11}^i$ и $B_{22}^j$, получаем:
\begin{equation}
    B_{12}^{(t)} = 
    \begin{pmatrix}
        0 & u^{(1)} v^{(2)} & u^{(1)} v^{(3)} \\
        0 & u^{(2)} v^{(2)} & u^{(2)} v^{(3)} \\
        0 & 0 & 0
    \end{pmatrix}
    \cdot B_{12} \cdot
    \begin{pmatrix}
        u^{(22)} v^{(22)} & u^{(22)} v^{(32)} \\
        0 & 0
    \end{pmatrix}
    \cdot t r^t + o(t r^t)
\end{equation}

Представляя $B_{12}$ в блочном виде
\begin{equation}
    B_{12} = 
    \begin{pmatrix}
        D_{11} & D_{12} \\
        D_{21} & D_{22} \\
        D_{31} & D_{32}
    \end{pmatrix},
\end{equation}
и производя перемножение, получаем
\begin{equation}
    B_{12}^{(t)} = 
    \begin{pmatrix}
        u'^{(11)} v^{(22)} & u'^{(11)} v^{(32)} \\
        u'^{(21)} v^{(22)} & u'^{(21)} v^{(32)} \\
        0 & 0
    \end{pmatrix}.
\end{equation}

Сформулируем теорему, определяющую вид блока $B_{lh}^{(t)}$ в общем случае.
\begin{theorem}
    \begin{equation}
        B_{lh}^{(t)} = u^{(l)} v^{(h)} t^{s_{lh} - 1} r^t (1 + o(1)),
    \end{equation}
    при $t \rightarrow \infty$, где $u^{(l)}$ и $v^{(h)}$ не зависят от $t$, и $s_{lh}$ --- число классов с номерами из $J$ среди $K_l, K_{l+1}, \ldots, K_h$.
\end{theorem}

\textbf{Доказательство.} Доказательство проведём индукцией по $w$. При $w = 2$ теорема выполняется.

Пусть утверждение теоремы верно для $w-1$ групп. Тогда
\begin{equation}
    A = 
    \begin{pmatrix}
        D_1 & E_1 \\
        0 & B_{w,w}
    \end{pmatrix}
    =
    \begin{pmatrix}
        B_{11} & E_2 \\
        0 & D_2
    \end{pmatrix},
\end{equation}
где
\begin{equation}
    E_1 = 
    \begin{pmatrix}
        B_{1,w} \\
        \vdots \\
        B_{w-1,w}
    \end{pmatrix},
    \quad{}E_2 = 
    \begin{pmatrix}
        B_{12} & \cdots & B_{1,w}
    \end{pmatrix}
\end{equation}

Тогда
\begin{equation}
    A^t = 
    \begin{pmatrix}
        D_1^t & E_1^{(t)} \\
        0 & B_{w,w}^t
    \end{pmatrix}
    =
    \begin{pmatrix}
        B_{11}^t & E_2^{(t)} \\
        0 & D_2^t
    \end{pmatrix}
\end{equation}

Для матриц $D_1$, $D_2$ утверждение теоремы по индукции справедливо. Для доказательства теоремы достаточно рассмотреть
\begin{equation}
    B_{1,w}^{(t)} = \sum_{l = 1}^{w-1} \sum_{i + j = t-1} B_{1,l}^{(i)} B_{l,w} B_{w,w}^j
\end{equation}
По предположению индукции слагаемое, содержащее $B_{1,w-1}^i = O(i^{s_{1,w-1} - 1} r^i)$, преобладает над остальными. Поэтому
\begin{equation}
    B_{1,w}^{(t)} = \sum_{i + j = t-1} B_{1,w-1}^{(i)} B_{w-1,w} B_{w,w}^j (1 + o(1))
\end{equation}

Подставляя по индукции выражение для $B_{1,w-1}$, получаем:
\begin{equation}
    B_{1,w}^{(t)} = 
    \begin{pmatrix}
        u'^{(11)} v^{(2w)} & u'^{(11)} v^{(3w)} \\
        u'^{(21)} v^{(2w)} & u'^{(21)} v^{(3w)} \\
        0 & 0
    \end{pmatrix}
    t^{s_{1,w} - 1} r^t + o(t^{s_{1,w}-1} r^t)
\end{equation}

\end{document}
