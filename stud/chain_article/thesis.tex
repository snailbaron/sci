\documentclass[12pt]{article}
%\usepackage[T2A]{fontenc}
\usepackage[utf8]{inputenc}
\usepackage[russian]{babel}
\usepackage{amsmath}
\usepackage{amssymb}

\oddsidemargin=0mm
\textwidth=160mm
\topmargin=10mm
\headheight=0mm
\headsep=0mm
\textheight=220mm

\renewcommand{\leq}{\leqslant}
\renewcommand{\geq}{\geqslant}
\renewcommand{\epsilon}{\varepsilon}

\newtheorem{theorem}{Теорема}
\newtheorem{lemma}{Лемма}

\title{Исследование разложимых КС-языков, имеющих вид <<цепочки>>}
\author{Игорь Мартынов}
\begin{document}

\clearpage
\setlength{\parindent}{0pt}
\setlength{\parskip}{8pt}

\begin{theorem}
	Энтропия языка $L^t = \left\{ \alpha \in L_G : \left| \alpha \right| = t \right\}$, где $G$ --- разложимая стохастическая КС-грамматика, имеющая вид <<цепочки>>, выражается формулой
	\begin{equation*}
		H(L^t) \tilde \sum_{i \in I_l} \sum_{j = 1}^{k_i} d_i H(R_i) \cdot t^2,
	\end{equation*}
	где	$H(R_i) = - \sum_{j = 1}^{k_i} p_{ij} \log p_{ij}$ --- энтропия множества $R_i$ правил вывода с нетерминалом $A_i$ в левой части, и $l$ --- номер критического класса, наиболее удалённого от начала цепочки.
\end{theorem}

Энтропия языка $L^t$ определяется количеством правил, нетерминал в левой части которых находится в наиболее удалённом от начала цепочки критическом классе. Число правил, нетерминал в левой части которых находится в других критических классах, есть $o(t^2)$, поэтому такие правила не оказывают влияния на асимптотику энтропии. Число правил, нетерминал в левой части которых находится в докритических классах, есть $O(1)$, поэтому такие правила также не влияют на энтропию языка $L^t$.

\end{document}