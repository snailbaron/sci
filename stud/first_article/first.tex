\documentclass[12pt]{article}
%\usepackage[T2A]{fontenc}
\usepackage[utf8]{inputenc}
\usepackage[russian]{babel}
\usepackage{amsmath}
\usepackage{amssymb}

\oddsidemargin=0mm
\textwidth=160mm
\topmargin=0mm
\headheight=0mm
\headsep=0mm
\textheight=240mm

\renewcommand{\leq}{\leqslant}
\renewcommand{\geq}{\geqslant}
\renewcommand{\epsilon}{\varepsilon}

\newtheorem{theorem}{Теорема}
\newtheorem{lemma}{Лемма}

\title{Исследование разложимых КС-языков, имеющих вид <<цепочки>>}
\author{Игорь Мартынов}
\begin{document}

\clearpage
\setlength{\parindent}{0pt}
\setlength{\parskip}{8pt}


\section{Основные определения}

\textit{Стохастической КС-грамматикой} \cite{fu-struct} называется система $G = <V_T, V_N, R, s>$, где $V_T$ и $V_N$ --- конечные множества терминальных и нетерминальных символов (терминалов и нетерминалов) соответственно, $s \in V_N$ --- аксиома, $R$ --- множество правил. Множество $R$ можно представить в виде $R = \cup_{i = 1}^k R_i$, где $k$ --- мощность алфавита $V_N$ и $R_i = \left\{r_{i1}, \ldots, r_{i,n_i}\right\}$. Каждое правило $r_{ij}$ из $R_i$ имеет вид
\begin{equation}
	r_{ij} : A_i \xrightarrow{p_{ij}} \beta_{ij},\qquad j = 1, \ldots, n_i,
\end{equation}
где $A_i \in V_N$, $\beta_{ij} \in (V_N \cup V_T)^*$ и $p_{ij}$ --- вероятность применения правила $r_{ij}$, причём
\begin{equation}
\label{eq:p_values}
	0 < p_{ij} \leq 1,\qquad \sum_{j = 1}^{n_i} p_{ij} = 1.
\end{equation}

Для $\alpha, \gamma \in (V_N \cup V_T)^*$ будем обозначать $\alpha \Rightarrow \gamma$, если существуют $\alpha_1, \alpha_2 \in (V_N \cup V_T)^*$, для которых $\alpha = \alpha_1 A_i \alpha_2$, $\gamma = \alpha_1 \beta_{ij} \alpha_2$ и в грамматике имеется правило $A_i \xrightarrow{p_{ij}} \beta_{ij}$. Через $\Rightarrow_*$ обозначим рефлексивное транзитивное замыкание отношения $\Rightarrow$. Грамматика $G$ задаёт контекстно-свободный язык $L_G = \left\{ \alpha \in V_T^* : s \Rightarrow_* \alpha\right\}$.

Выводом слова $\alpha$ назовём последовательность правил $\omega(\alpha) = (r_{i_1 j_1}, r_{i_2 j_2}, \ldots, r_{i_q j_q})$, с помощью последовательного применения которых слово $\alpha$ выводится из аксиомы $s$. Если при этом каждое правило применяется к самому левому нетерминалу в слове, такой вывод называется левым. Для вывода $\omega(\alpha) = (r_{i_1 j_1}, \ldots, r_{i_q j_q})$ определим величину $p(\omega(\alpha)) = p_{i_1 j_1} \cdot \ldots \cdot p_{i_q j_q}$.

Важное значение имеет понятие \textit{дерева вывода} \cite{aho-ulman-syntax}. Дерево вывода для слова $\alpha$ строится следующим образом. Корень дерева помечается аксиомой $s$. Далее последовательно рассматриваются правила левого вывода слова $\alpha$. Пусть на очередном шаге рассматривается правило $A_i \xrightarrow{p_{ij}} b_{i_1} b_{i_2} \ldots b_{i_m}$, где $b_{i_l} \in (V_N \cup V_T)$ ($l = 1,\ldots,m$). Тогда из самой левой вершины-листа дерева, помеченной символом $A_i$, проводится $m$ дуг в вершины следующего яруса, которые помечаются слева направо символами $b_{i1}, \ldots, b_{i,m}$ соответственно. После построения дуг и вершин для всех правил в выводе листья дерева помечены терминальными символами (либо пустым словом $\lambda$, если применяется правило вида $A_i \xrightarrow{p_{ij}} \lambda$) и само слово получается при обходе листьев дерева слева направо. \textit{Высотой} дерева вывода будем называть максимальную длину пути от корня к листу.

Обозначим $p(\alpha) = \sum \omega(\alpha)$, где сумма берётся по всем левым выводам слова $\alpha$. Грамматика $G$ называется \textit{согласованной}, если
\begin{equation}
\label{eq:soglas}
	\lim_{n \rightarrow \infty} \sum_{\substack{\alpha \in L_G\\\left|\alpha\right| \leq n}} p(\alpha) = 1.
\end{equation}
Согласованная грамматика $G$ задаёт распределение вероятностей $P$ на множестве $L_G$ и определяет \textit{стохастический КС-язык} $\mathcal{L} = (L_G, P)$. В этом случае величина $p(\alpha)$ --- вероятность слова $\alpha$ в $L_G$. В дальнейшем будем рассматривать только согласованные грамматики.

Определим многомерные производящие функции \cite{fu-struct}:
\begin{equation}
\label{eq:f-def}
	F_i(s_1, s_2, \ldots, s_k) = \sum_{j = 1}^{n_i} p_{ij} s_1^{l_1} s_2^{l_2} \ldots s_k^{l_k}\quad (1 \leq i \leq k),
\end{equation}
где $n_i$ --- число правил вывода с нетерминалом $A_i$ в левой части, и $l_m = l_m(i,j)$ --- число вхождений нетермина $A_m$ в правую часть правила $A_i \xrightarrow{p_{ij}} \beta_{ij}$.

Величины
\begin{equation}
\label{eq:aij-definition}
	a^i_j = \left. \frac{\partial F_i(s_1, s_2, \ldots, s_k)}{\partial s_j} \right|_{s_1 = \ldots = s_k = 1}
\end{equation}
называются первыми моментами грамматики и образуют матрицу первых моментов $A = (a^i_j)$. Аналогично определяются вторые моменты
\begin{equation}
\label{eq:bij-definition}
	b^i_{jl} = \left. \frac{\partial^2 F_i(s_1, s_2, \ldots, s_k)}{\partial s_l \partial s_j} \right|_{s_1 = \ldots = s_k = 1}
\end{equation}

Для нетерминалов $A_i, A_j$ будем обозначать $A_i \rightarrow A_j$, если в грамматике имеется правило $A_i \xrightarrow{p_{ij}} \alpha_1 A_j \alpha_2$, где $\alpha_1, \alpha_2 \in (V_N \cup V_T)^*$. Рефлексивное транзитивное замыкание отношения $\rightarrow$ обозначим $\rightarrow_*$. Если одновременно $A_i \rightarrow_* A_j$ и $A_j \rightarrow_* A_i$, будем обозначать $A_i \leftrightarrow_* A_j$. Отношение $\leftrightarrow_*$ разбивает множество нетерминалов грамматики на классы
\begin{equation}
	K_1, K_2, \ldots, K_m.
\end{equation}
Множества номеров нетерминалов, входящих в класс $K_j$ обозначим через $I_j$. При $m \geq 2$ грамматика называется \textit{разложимой}.

Разложимой грамматике соответствует разложимая матрица \cite{gantmaher-matrix-theory} первых моментов. Обозначим $K_i \prec K_j$ , если $i \neq j$ и существуют такие $A_1 \in K_i$ и $A_2 \in K_j$, что $A_1 \rightarrow A_2$. Будем говорить, что грамматика имеет вид <<цепочки>>, если она разложима, и для множества классов выполняется соотношение $K_1 \prec K_2 \prec \ldots \prec K_m$. При этом граф, построенный на множестве классов по отношению $\prec$, имеет вид:

\begin{picture}(1000, 40)
	\put(20, 20){\circle{100}}
	\put(14, 16){$K_{i_1}$}
	\put(40, 20){\vector(1,0){35}}
	\put(95, 20){\circle{100}}
	\put(89, 16){$K_{i_2}$}
	\put(115, 20){\vector(1,0){35}}
	\put(170, 10){...}
	\put(195, 20){\vector(1,0){35}}
	\put(250, 20){\circle{100}}
	\put(244, 16){$K_{i_m}$}
\end{picture}

Назовём класс $K$ \textit{особым}, если он содержит ровно один нетерминал $A_i$, и в грамматике отсутствует правило вида $A_i \xrightarrow{p_{ij}} \alpha_1 A_i \alpha_2$, где $\alpha_1, \alpha_2 \in (V_N \cup V_T)^*$. Не уменьшая общности, будем считать, что грамматика не имеет особых классов.

Матрицу первых моментов грамматики, имеющей вид <<цепочки>> можно представить следующим образом.
\begin{equation}
\label{eq:amatrix}
	A =
	\begin{pmatrix}
		A_{11} & A_{12} & 0      & \cdots & 0           & 0          \\
		0      & A_{22} & A_{23} & \cdots & 0           & 0          \\ 
		\vdots & \vdots & \vdots & \ddots & \vdots      & \vdots     \\
		0      & 0      & 0      & \cdots & A_{m-1,m-1} & A_{m-1, m} \\
		0      & 0      & 0      & \cdots & 0           & A_{m,m}    \\
	\end{pmatrix}.
\end{equation}
Блок $A_{ii}$ соответствует классу $K_i$ и является неразложимой неотрицательной матрицей.

По определению ($\ref{eq:aij-definition}$), матрицы $A_{11}, A_{22}, \ldots, A_{m,m}$ неразложимы и неотрицательны. Согласно теореме Фробениуса \cite{gantmaher-matrix-theory}, неразложимая неотрицательная матрица всегда имеет простое положительное максимальное по модулю собственное число (перронов корень). Обозначим перроновы корни матриц $A_{11}, \ldots, A_{m,m}$ через $r_1, \ldots, r_m$ соответственно. Тогда $r = \max\{r_1, \ldots, r_m\}$ --- перронов корень всей матрицы $A$. В данной работе рассматривается случай $r = 1$. По аналогии с теорией ветвящихся процессов \cite{sevast-processes} будем называть этот случай \textit{критическим}. Классы, для которых перронов корень соответствующих подматриц равен $1$, также будем называть критическими.

\sloppy{
Обозначим $s_{lh}$ (при $l \leq h$) --- число критических классов среди подцепочки $K_l, K_{l+1}, \ldots, K_h$. Разобьём последовательность классов $K_1, K_2, \ldots, K_m$ на группы $\mathcal{M}_1, \mathcal{M}_2, \ldots, \mathcal{M}_w$, где $w = s_{1m}$. Класс $K_l$ отнесём к группе $\mathcal{M}_w$ при $s_{lw} <= 1$ и к группе $M_{w-j+1}$ при $s_{lw} = j$ ($j = 2,\ldots,w$).

Тогда матрицу $A$ можно представить в виде:
\begin{equation*}
	A =
	\begin{pmatrix}
		B_{11} & B_{12} & 0      & \cdots & 0           & 0          \\
		0      & B_{22} & B_{23} & \cdots & 0           & 0          \\ 
		\vdots & \vdots & \vdots & \ddots & \vdots      & \vdots     \\
		0      & 0      & 0      & \cdots & B_{w-1,w-1} & B_{w-1, w} \\
		0      & 0      & 0      & \cdots & 0           & B_{w,w}    \\
	\end{pmatrix},
\end{equation*}
где матрица $B_{lh}$ находится на пересечении строк для классов из группы $\mathcal{M}_l$ и столбцов для классов из группы $\mathcal{M}_h$. Матрицы $B_{lh}$, в свою очередь, имеют вид
\begin{equation*}
	B_{lh} =
	\begin{pmatrix}
		C_{11} & C_{12} & 0      \\
		0      & C_{22} & C_{23} \\
		0      & 0      & C_{33} \\
	\end{pmatrix},
\end{equation*}
где $C_{22}$ --- блок, стоящий на пересечении строк для $l$-го критического класса и столбцов для $h$-го критического класса. При $l = h$ этот блок является неразложимой матрицей. Блоки $C_{11}$ и $C_{33}$ стоят на пересечении строк и столбцов, соответствующих докритическим классам. При $l,h < w$ блок $B_{lh}$ имеет вид
\begin{equation*}
	B_{lh} =
	\begin{pmatrix}
		C_{11} & C_{12} \\
		0      & C_{22} \\
	\end{pmatrix}.
\end{equation*}

Блок, находящийся на позиции блока $B_{lh}$ в матрице $A^t$, обозначим $B^{(t)}_{lh}$.

В \cite{zhiltsova-about-matrix} доказана следующая теорема.
\begin{theorem}
	При $t \rightarrow \infty$
	\begin{equation*}
		B^{(t)}_{lh} = 
		\begin{pmatrix}
			0 & b \cdot U^{(l)}_I V^{(h)}_{II}    & b \cdot U^{(l)}_I V^{(h)}_{III} \\
			0 & b \cdot U^{(l)}_{II} V^{(h)}_{II} & b \cdot U^{(l)}_{II} V^{(h)}_{III} \\
			0 & 0 & 0 \\
		\end{pmatrix} =
		b \cdot U^{(l)} V^{(h)} t^{s_{lh} - 1} r^t \cdot (1 + o(1)),
	\end{equation*}
	где $U^{(q)}$ и $V^{(q)}$ --- правый и левый собственные векторы матрицы $B_{qq}$, и $b = V^{(l)} B_{lh} U^{(h)}$.
\end{theorem}

\section{Вероятности продолжения}

Введём обозначения
\begin{equation}
\begin{split}
	&F_i(\mathbf{s}) = F_i(0, \mathbf{s}) = \sum_{j = 1}^{n_i} p_{ij} s_1^{l_1} s_2^{l_2} \ldots s_k^{l_k}\quad (1 \leq i \leq k), \\
	&F_i(t, \mathbf{s}) = F_i(\mathbf{F}(t-1, \mathbf{s}))\quad (t > 0,\; 1 \leq i \leq k),
\end{split}
\end{equation}
где $\mathbf{s} = (s_1, \ldots, s_k)$, $0 \leq s_j \leq 1$ и $\mathbf{F}(t, \mathbf{s}) = (F_1(t, \mathbf{s}), \ldots, F_k(t, \mathbf{s}))$.

В силу согласованности грамматики $F_i(t, \mathbf{0}) \rightarrow 1$ при $t \rightarrow \infty$. В самом деле, из ($\ref{eq:soglas}$) следует
\begin{equation*}
	\sum_{d \in D} p(d) = \lim_{t \rightarrow \infty} \sum_{d \in D^{\leq t}} p(d) = 1,
\end{equation*}
откуда
\begin{equation*}
	\lim_{t \rightarrow \infty} \sum_{d \in D^{> t}} p(d) = 0
\end{equation*}
и
\begin{equation*}
\label{eq:Fi-limit}
	F_i(t, \mathbf{0}) = 1 - \sum_{d \in D^{> t}} p(d) \rightarrow 1.
\end{equation*}

Раскладывая $F_i(\mathbf{s})$ в ряд Тейлора в окрестности $\mathbf{s} = (1, \ldots, 1)$, и учитывая равенство $F_i(1, 1, \ldots, 1) = 1$, получаем:
\begin{equation}
	1 - F_i(\mathbf{s}) = \sum_{j = 1}^{n_i} a^i_j(1 - s_j) - \frac{1}{2} \sum_{1 \leq j,l \leq n_i} b^i_{jl} (1 - s_j) (1 - s_l) + O\left(\sum_{j = 1}^k \left| 1 - s_j \right|^3\right),
\end{equation}

Подставляя в качестве $\textbf{s}$ вектор $\textbf{F}(t, s) = (F_1(t, s), F_2(t, s), \ldots, F_k(t, s))$, получаем:
\begin{multline}
\label{eq:basic_fi}
	1 - F_i(t + 1, s) = \sum_{i = 1}^k a^i_j (1 - F_j(t,s)) - \frac{1}{2} \sum_{1 \leq j,l \leq k} b^i_{jl} (1 - F_j(t,s)) (1 - F_l(t,s)) + \\
	+ O\left( \sum_{j = 1}^k \left| 1 - F_j(t,s) \right|^3 \right)
\end{multline}

Введём вектор \textit{вероятностей продолжения} $Q(t) = (Q_1(t), Q_2(t), \ldots, Q_k(t))^T$, положив
\begin{equation}
	Q_i(t) = 1 - \left. F_i(t, \textbf{s}) \right|_{s_1 = s_2 = \ldots = s_k = 0}
\end{equation}
В силу ($\ref{eq:Fi-limit}$) $Q_i(t) \rightarrow 0$ при $t \rightarrow \infty$.

Тогда уравнение ($\ref{eq:basic_fi}$) примет вид
\begin{equation}
\label{eq:basic_qi}
	Q_i(t+1) = \sum_{i = 1}^k a^i_j Q_i(t) - \frac{1}{2} \sum_{1 \leq j,l \leq k} b^i_{jl} Q_j(t) Q_l(t) + O \left( \sum_{j = 1}^k \left| Q_j(t) \right|^3 \right)
\end{equation}

Для каждого из классов $K_n$ будем рассматривать вектор $Q^{(n)}(t)$ --- вектор-столбец, содержащий вероятности продолжения для нетерминалов из класса $K_n$ в порядке их нумерации. Тогда
\begin{equation}
	Q(t) =
	\begin{pmatrix}
		Q^{(1)}(t) \\
		Q^{(2)}(t) \\
		\vdots \\
		Q^{(m)}(t)
	\end{pmatrix},
	\quad Q^{(j)}(t) \in \mathbb{R}^{k_j},
\end{equation}
где $k_j = \left| K_j \right|$. Обозначим через $I_n$ иножество индексов нетерминалов, входящих в класс $K_n$. Используя это обозначение, уравнение ($\ref{eq:basic_qi}$) можно записать в виде
\begin{align}
	&Q_i(t+1) = \sum_{j \in I_n} a^i_j Q_j(t) + \sum_{i \in I_{n+1}} a^i_j Q_j(t) \cdot (1 + o(1)) & &(i \in I_n, n < m) \\
	&Q_i(t+1) = \sum_{j \in I_m} a^i_j Q_j(t) \cdot (1 + o(1)) & &(i \in I_m)
\end{align}
или, используя вид ($\ref{eq:amatrix}$) матрицы первых моментов,
\begin{equation}
	Q^{(n)}(t+1) = A_{n,n} Q^{(n)}(t) + A_{n,n+1} Q^{(n+1)}(t) (1 + o(1))
\end{equation}
Для всего вектора $Q(t)$ верно равенство
\begin{equation}
\label{eq:q_t+1_from_q_t}
	Q(t+1) = (A - A(t)) Q(t),
\end{equation}
где $A(t)$ --- матрица, составленная из элементов $a_{ij} = \frac{1}{2} \sum_{l = 1}^k b^i_{jl} Q_l(t)$ ($1 \leq i,j \leq k$). В силу согласованности грамматики $Q(t) \rightarrow 0$ и, следовательно, $A(t) \rightarrow 0$ при $t \rightarrow \infty$.

Докажем, что компоненты вектора $Q^{(n)}(t)$ пропорциональны некоторому вектору $U^{(n)}$. Доказательство аналогичного факта для случая двух классов принадлежит А.~Борисову. Здесь мы проведём похожие рассуждения.

Зафиксируем некоторое $\tau \geq 0$. Тогда из ($\ref{eq:q_t+1_from_q_t}$) получаем
\begin{equation}
\label{eq:q_t+1_from_q_tau}
	Q(t+1) = (A - A(t)) \cdot \ldots \cdot (A - A(\tau)) Q(\tau)
\end{equation}
Обозначим
\begin{equation}
	\begin{split}
		&A^*(t) = (A - A(t)) \cdot (A - A(t-1)) \cdot \ldots \cdot (A - A(\tau + 1)) \\
		&\tilde{A}^*_{ij} = \frac{A^*_{ij}(t)}{t^{s_{ij}}} \\
		&\tilde{A}_{ij} = \frac{A^{(t)}_{ij}}{t^{s_{ij}}},
	\end{split}
\end{equation}
где $A^{(t)}_{ij}$ --- блоки, расположенные на месте блоков $A_{ij}$ в матрице $A^t$ и $s_{ij}$ --- число критических классов в подцепочке $K_i, K_{i+1}, \ldots, K_j$.


Из исследования асимптотики матрицы $A^t$ известно \cite{zhiltsova-about-matrix}, что $\tilde{A}_{ij}(t) \rightarrow \tilde{a}_{ij} U^{(i)} V^{(j)}$, где $\tilde{a}_{ij}$ --- некоторые константы, $U^{(i)}$ --- вектор-строка длины $k_i$, а $V^{(j)}$ --- вектор-столбец длины $k_j$.

Выберем произвольные $\epsilon_1, \epsilon_2$, такие что $0 < \epsilon_1, \epsilon_2 < 1$. Тогда существуют функции $l(\epsilon_1)$ и $n(\epsilon_2)$, такие что
\begin{equation}
	\begin{split}
		&\left| \tilde{A}_{ij}(l(\epsilon_1)) - \tilde{a}_{ij} U^{(i)} V^{(j)} \right| < \epsilon_1 E \\
		&\forall t \geq n(\epsilon_2)\quad A(t) < \epsilon_2 A
	\end{split}.
\end{equation}

Рассмотрим произвольный вектор-столбец $x > \mathbf{0}$ длины $k$. Тогда выполняется оценка
\begin{equation}
	(1 - \epsilon_2)^l A^l x^{(\tau)} \leq A^*(t) x^{(\tau)} \leq A^l x^{(\tau)},
\end{equation}
где $x^{(\tau)} = (A - A(\tau)) x$. Записывая это неравенство отдельно для блоков $A_{ij}$, получаем
\begin{equation}
	(1 - \epsilon_2)^l A_{ij}^l x^{(\tau)}_j \leq A^*_{ij}(l) x^{(\tau)}_j \leq A^{(l)}_{ij} x^{(\tau)}_j,
\end{equation}
откуда
\begin{equation}
	(1 - \epsilon_2)^l \tilde{A}_{ij}(l) x^{(\tau)} \leq \tilde{A}^*_{ij}(l) x^{(\tau)}_j \leq \tilde{A}_{ij}(l) x^{(\tau)}
\end{equation}
Вычитая из всех частей неравенства $\tilde{A}_{ij}(l) x^{(\tau)}_j$, получаем оценку
\begin{equation}
	\left| \left( \tilde{A}^*_{ij}(l) - \tilde{A}_{ij}(l) \right) x^{(\tau)}_j \right| \leq (1 - (1 - \epsilon_2)^l) \tilde{A}_{ij}(l) x^{(\tau)}
\end{equation}

Используя эту оценку, можем записать
\begin{multline}
	\left| \tilde{A}^*_{ij}(t) - \tilde{a}_{ij} U^{(i)} V^{(j)} x^{(\tau)}_j \right| \leq \left| \left( \tilde{A}^*_{ij}(t) - \tilde{A}_{ij}(t) \right) x^{(\tau)} \right| + \\
	+ \left| \left( \tilde{A}_{ij}(l) - \tilde{a}_{ij} U^{(i)} V^{(j)} \right) x^{(\tau)}_j \right| \leq (1 - (1 - \epsilon_2)^l) \tilde{A}_{ij}(l) x^{(\tau)}_j + \epsilon_1 x^{(\tau)}_j \leq \\
	\leq (1 - (1 - \epsilon_2)^l) h k_j x^{(\tau)}_j + \epsilon_1 x^{(\tau)}_j \leq \left( (1 - 1 - \epsilon_2)^l) h k_j + \epsilon_1 \right) x^*_j(\tau),
\end{multline}
где $h = \max_{i,j,l} \left\{ \tilde{A}_{ij}(l) \right\}$ и $x^*_j(\tau) = \max_i (x^{(\tau)}_j)_i$.

Устремляем $\epsilon_2$ к нулю, затем $\epsilon_1$ к нулю таким образом, чтобы выполнялось условие
\begin{equation}
	l(\epsilon_1) \log(1 - \epsilon_2) \rightarrow -\infty
\end{equation}

Тогда 
\begin{equation}
	\left| \tilde{A}^*_{ij}(t) - \tilde{a}_{ij} U^{(i)} V^{(j)} x^{(\tau)}_j \right| \leq \epsilon x^*_j(\tau)\quad (\epsilon \rightarrow 0).
\end{equation}
Домножая слева на $V^{(i)}$, имеем
\begin{equation}
	\left| V^{(i)} \tilde{A}^*_{ij}(t) x^{(\tau)}_j - \tilde{a}_{ij} V^{(j)} x^{(\tau)}_j \right| \leq \epsilon k_i \max \left\{ (V^{(i)} \right\} x^*_j(\tau) \leq \epsilon^* V^{(j)} x^{(\tau)}_j.
\end{equation}

Отсюда,
\begin{equation}
	\left| \frac{\tilde{A}^*_{ij}(t) x^{(\tau)}_j}{V^{(i)} \tilde{A}^*_{ij}(t) x^{(\tau)}_j} - \frac{\tilde{a}_{ij} U^{(i)} V^{(i)} x^{(\tau)}_j}{\tilde{a}_{ij} V^{(j)} x^{(\tau)}_j} \right| = \left| \frac{\tilde{A}^*_{ij}(t) x^{(\tau)}_j}{V^{(i)} \tilde{A}^*_{ij}(t) x^{(\tau)}_j} - U^{(i)} \right| \rightarrow 0
\end{equation}
или же
\begin{equation}
	\left| \frac{A^*_{ij}(t) x^{(\tau)}_j}{V^{(i)} A^*_{ij}(t) x^{(\tau)}_j} - U^{(i)} \right| \rightarrow 0,
\end{equation}
откуда
\begin{equation}
	(A - A(t)) \cdot \ldots \cdot (A - A(\tau)) \cdot x_j = U^{(i)} V^{(i)} (A - A(t)) \cdot \ldots \cdot (A - A(\tau)) \cdot x_j \cdot (1 + o(1))
\end{equation}

Ввиду полученного выражения и ($\ref{eq:q_t+1_from_q_tau}$) компоненты каждого из векторов $Q^{(n)}(t)$ пропорциональны компонентам вектора $U^{(n)}$.

Оценим теперь асимптотику элементов вектора $Q^{(n)}(t)$ при $t \rightarrow \infty$.

Положим $V^{(n)} Q^{(n)}(t) = Q^{(n)}_*(t)$, и домножим уравнение $(\ref{eq:basic_qi})$ скалярно на $V^{(n)}$. Заметим, что
\begin{equation}
\label{eq:q_uq}
	Q^{(n)}(t) = U^{(n)} Q^{(n)}_*(t) (1 + o(1)).
\end{equation}
\begin{multline}
\label{eq:q_star}
	Q^{(n)}_*(t+1) = Q{(n)}_*(t) + V^{(n)} B_{n,n+1} U^{(n+1)} Q^{(n+1)}_*(t) - \\
	- \frac{1}{2} \sum_{1 \leq i,j,l \leq k_n} V^{(n)}_i b^i_{jl}(n) U^{(n)}_j U^{(n)}_l \left( Q^{(n)}_*(t) \right)^2 (1 + o(1)).
\end{multline}
Обозначим $\delta Q^{(n)}_*(t) = Q^{(n)}_*(t+1) - Q^{(n)}_*(t)$, а также
\begin{equation*}
	\begin{split}
		&b_n = V^{(n)} B_{n,n+1} U^{(n+1)} \\
		&B_n = \sum_{1 \leq i,j,l \leq k_n} V^{(n)}_i b^i_{jl}(n) U^{(n)}_j U^{(n)}_l \\
	\end{split}
\end{equation*}
Тогда уравнение $(\ref{eq:q_star})$ перепишется как
\begin{equation}
\label{eq:delta_q_star_q_star}
	\delta Q^{(n)}_*(t) = b_n Q^{(n+1)}_*(t) - \frac{1}{2} B_n (Q^{(n)}_*(t))^2 (1 + o(1))
\end{equation}
Выражение для $\delta Q^{(n)}_*(t)$ также можно получить из $(\ref{eq:basic_qi})$, вычитая это уравнение из себя с заменой $t \rightarrow t+1$:
\begin{multline*}
	\delta Q^{(n)}_*(t+1) = \sum_{j = 1}^{k_n} a^i_j(n) \delta Q^{(n)}_j(t) + \sum_{j = 1}^{k_{n+1}} a^i_j(n) \delta Q^{(n+1)}_j(t) - \\
	- \frac{1}{2} \sum_{1 \leq j,l \leq k_n} b^i_{jl}(n) \left( Q^{(n)}_j(t+1) Q^{(n)}_l(t+1) - Q^{(n)}_j(t) Q^{(n)}_l(t) \right) (1 + o(1))
\end{multline*}
Скалярно домножая на $V^{(n)}$, получим
\begin{multline}
\label{eq:delta_q_star}
	\delta Q^{(n)}_*(t+1) = \delta Q^{(n)}_*(t) + b_n \delta Q^{(n+1)}_*(t) - \\
	- \frac{1}{2} B_n \delta Q^{(n)}_*(t) \left( Q^{(n)}_*(t+1) + Q^{(n)}_*(t) \right) (1 + o(1))
\end{multline}

Для последнего класса
\begin{equation}
\label{eq:q_t_for_last_class}
	Q^{(w)}_*(t) = c_w t^{-1} (1 + o(1)),
\end{equation}
что следует из неразложимого случая. Проведём рассуждение по индукции. Пусть для группы с номером $n+1$ верно
\begin{equation*}
	Q^{(n+1)}_*(t) = c_{n+1} t^{-\alpha} (1 + o(1)),
\end{equation*}
где $0 < \alpha \leq 1$. Положим
\begin{equation*}
	z(t) = t^{\alpha} \delta Q^{(n)}_*(t)
\end{equation*}
Произведя замену в уравнении $(\ref{eq:delta_q_star})$, и имея в виду, что $Q^{(n)}_*(t+1) = O(Q^{(n)}_*(t))$, получаем
\begin{equation*}
	\frac{z(t+1)}{(t+1)^{\alpha}} - \frac{z(t)}{t^\alpha} = b_n \delta Q^{(n+1)}_*(t) (1 + o(1)) - \frac{1}{2} B_n \frac{z(t)}{t^\alpha} \cdot 2 Q^{(n)}_*(t) (1 + o(1))
\end{equation*}
Преобразуем выражение в левой части уравнения:
\begin{multline*}
	\frac{z(t+1)}{(t+1)^\alpha} - \frac{z(t)}{t^\alpha} = \frac{t^\alpha z(t+1) - (t+1)^\alpha z(t)}{t^\alpha (t+1)^\alpha} = \\
	= \frac{t^\alpha z(t+1) - t^\alpha \left(1 + \frac{\alpha}{t} + o\left( \frac{1}{t} \right) \right) z(t)}{t^\alpha (t+1)^\alpha} = \frac{\delta z(t)}{(t+1)^\alpha} - \frac{\alpha z(t) (1 + o(1))}{t (t + 1)^\alpha}
\end{multline*}
Тогда
\begin{equation*}
	\frac{\delta z(t)}{(t+1)^\alpha} - \frac{\alpha z(t) (1 + o(1))}{t (t+1)^\alpha} = b_n \delta Q^{(n)}_*(t) - \frac{B_n}{t^\alpha} Q^{(n)}_*(t) z(t) (1 + o(1))
\end{equation*}
По предположению индукции, $\delta Q^{(n+1)}_*(t) = -\frac{c_{n+1} \alpha}{t (t+1)^\alpha} (1 + o(1))$, и тогда
\begin{equation*}
	\frac{\delta z(t)}{(t+1)^\alpha} - \frac{\alpha z(t) (1 + o(1))}{t (t + 1)^\alpha} = -\frac{b_n \alpha c_{n+1}}{t (t+1)^\alpha} - \frac{B_n}{t^\alpha} Q^{(n)}_*(t) z(t) (1 + o(1))
\end{equation*}
Домножая на $(t + 1)^\alpha$, получаем
\begin{equation*}
	\delta z(t) - \frac{\alpha z(t)}{t} = -\frac{b_n \alpha c_{n+1}}{t} - B_n Q^{(n)}_*(t) z(t) (1 + o(1))
\end{equation*}
Заметим, что, в силу предположения индукции, $\frac{1}{t} \leq Q^{(n+1)}_*(t) = o(Q^{(n)}_*(t))$, поэтому можно записать
\begin{equation}
\label{eq:z_t}
	\delta z(t) = -\frac{b_n \alpha c_{n+1}}{t} - B_n Q^{(n)}_*(t) (1 + o(1))
\end{equation}

Известна следующая лемма (доказательство леммы принадлежит А.~Борисову).
\begin{lemma}
\label{lemma:zfg}
	Пусть последовательность z(t) (t = 1,2,\ldots) удовлетворяет рекуррентному соотношению
	\begin{equation*}
		\delta z(t) = f(t) - g(t) z(t),
	\end{equation*}
	где при $t \rightarrow \infty$ выполняются условия
	\begin{equation*}
		g(t) \rightarrow 0, \frac{f(t)}{g(t)} \rightarrow 0, \sum_{k = 1}^t g(k) \rightarrow \infty.
	\end{equation*}
	Пусть также $g(t) > 0$ при любом $t > t_0$. Тогда $z(t) \rightarrow 0$ при $t \rightarrow \infty$.
\end{lemma}

Полагая в уравнении $(\ref{eq:z_t})$ $f(t) = -\frac{b_n \alpha c_{n+1}}{t} (1 + o(1))$, $g(t) = B_n Q^{(n)}_*(t) (1 + o(1))$, замечаем, что для $z(t)$ выполняются все условия леммы $(\ref{lemma:zfg})$, и соответственно, $z(t) \rightarrow 0$ при $t \rightarrow \infty$. Из определения $z(t)$ получаем:
\begin{equation*}
	\delta Q^{(n)}_*(t) = o\left( \frac{1}{t^\alpha} \right).
\end{equation*}
Подставляя эту оценку в $(\ref{eq:delta_q_star_q_star})$, получаем
\begin{equation*}
	o\left( \frac{1}{t^\alpha} \right) = \frac{b_n c_{n+1}}{t^\alpha} (1 + o(1)) - \frac{B_n}{2} \left( Q^{(n)}_*(t) \right)^2 (1 + o(1))
\end{equation*}
Отсюда
\begin{equation*}
	\frac{b_n c_{n+1}}{t^\alpha} (1 + o(1)) = \frac{B_n}{2} \left( Q^{(n)}_*(t) \right)^2 (1 + o(1))
\end{equation*}
Тогда для $Q^{(n)}_*(t)$ получаем оценку
\begin{equation*}
	Q^{(n)}_*(t) = \sqrt{\frac{2 b_n}{B_n} c_{n+1} \frac{1}{t^\alpha}} (1 + o(1)) = \sqrt{\frac{2 b_n}{B_n} k_{n+1}} \cdot t^{-\frac{\alpha}{2}} (1 + o(1))
\end{equation*}
При этом, полагая $c_n = \sqrt{\frac{2 b_n}{B_n} c_{n+1}}$, мы остаёмся в рамках предположения индукции. Учитывая $(\ref{eq:q_t_for_last_class})$, можем записать асимптотику $Q^{(n)}_*(t)$ для произвольной группы $n$:
\begin{multline*}
	Q^{(n)}_*(t) = \sqrt{\frac{2 b_n}{B_n} \sqrt{\frac{2 b_{n+1}}{B_{n+1}} \cdots \sqrt{\frac{2 b_{w-1}}{B_{w-1} B_w} \cdot t^{-\left(\frac{1}{2}\right)^{w - n}}}}} = \\
	= \prod_{k = n}^{w - 1} \left(\frac{2 b_n}{B_n}\right)^{\left(\frac{1}{2}\right)^{w - n + 1}} \cdot \left(\frac{1}{B_w}\right)^{\left(\frac{1}{2}\right)^{w - n}} \cdot t^{-\left(\frac{1}{2}\right)^{w-n}}
\end{multline*}

Учитывая $(\ref{eq:q_uq})$, получаем
\begin{equation*}
	\begin{split}
		&Q_i(t) = c_n U^{(n)}_j t^{-\left(\frac{1}{2}\right)^{w-n}} \cdot (1 + o(1)) \\
		&P_i(t) = \tilde{c}_n U^{(n)}_j t^{-1 -\left(\frac{1}{2}\right)^{w-n}} \cdot (1 + o(1))
	\end{split}
\end{equation*}
где нетерминал $A_i$ находится в последнем критическом классе цепочки или в одном из предшествующих классов, $n$ --- номер группы, в которую входит класс, содержащий $A_i$, $w$ --- число групп, и
\begin{equation*}
	c_n = \prod_{k = n}^{w - 1} \left(\frac{2 b_n}{B_n}\right)^{\left(\frac{1}{2}\right)^{w - n + 1}} \cdot \left(\frac{1}{B_w}\right)^{\left(\frac{1}{2}\right)^{w - n}}
\end{equation*}

\section{Математические ожидания числа применений правила в деревьях вывода}

Обозначим через $q^l_{ij}(t,\tau)$ и $\overline{q}^l_{ij}(t,\tau)$ случайные величины, равные числу применений правила $r_{ij}$ в дереве вывода, соответственно, из $D^t_l$ и $D^{\leq t}_l$, на ярусе $\tau$. Пусть также
\begin{equation*}
	\begin{split}
		&S^l_{ij}(t) = \sum_{\tau = 1}^{t-1} q^l_{ij}(t, \tau) \\
		&\overline{S}^l_{ij}(t) = \sum_{\tau = 1}^{t-1} \overline{q}^l_{ij}(t, \tau)
	\end{split}
\end{equation*}
и $S^l_{ij}(t)$, $\overline{S}^l_{ij}(t)$ --- соответственно число применений правила $r_{ij}$ в дереве из $D^t_l$, $D^{\leq t}_l$. Для удобства записи положим
\begin{align*}
	S_{ij}(t) = S^l_{ij}(t),\quad &\overline{S}_{ij}(t) = \overline{S}^l_{ij}(t), \\
	q_{ij}(t, \tau) = q^l_{ij}(t, \tau),\quad &\overline{q}_{ij}(t, \tau) = \overline{q}^l_{ij}(t, \tau)
\end{align*}
Рассмотрим математические ожидания некоторых из введённых величин. Обозначим
\begin{equation*}
	M^l_{ij}(t) = M[S^l_{ij}(t)],\quad \overline{M}^l_{ij}(t) = [\overline{S}^l_{ij}(t)].
\end{equation*}

Для нахождения величин $\overline{M}^l_{ij}(t)$ и $M^l_{ij}(t)$ будут использованы следующие три леммы.

\begin{lemma}
\label{l:first}
	\cite{sevast-processes} Пусть $s, d$ --- натуральные числа, $m = (m_1, \ldots, m_s)$ --- вектор целых неотрицательных чисел, $y = (y_1, \ldots, y_s)$ --- вектор, и $\overline{m} = \sum_{j = 1}^s m_j$. Тогда
	\begin{equation*}
		(1 - y_1)^{n_1} \ldots (1 - y_s)^{n_s} = \sum_{\substack{\overline{m} < d \\ m \geq 0}} \binom{n_1}{m_1} \binom{n_2}{m_2} \ldots \binom{n_s}{m_s} (-1)^{\overline{m}} y^m + R_d(n_1, \ldots, n_s, y),
	\end{equation*}
	где $y^m = y_1^{m_1} \ldots y_s^{m_s}$, и остаточный член представим в виде
	\begin{equation*}
		R_d(n_1, \ldots, n_s, y) = \sum_{\substack{\overline{m} = d \\ m \geq 0}} (-1)^d \epsilon_m(n_1, \ldots, n_s, y) y^m,
	\end{equation*}
	причём
	\begin{equation*}
		0 \leq \epsilon_m(n_1, \ldots, n_s, y') \leq \epsilon_m(n_1, \ldots, n_s, y) \leq \binom{n_1}{m_1} \ldots \binom{n_s}{m_s}
	\end{equation*}
	при $0 \leq y_i \leq y'_i \leq 1$ ($i = 1,\ldots,s$).
\end{lemma}

\begin{lemma}
\label{l:Ab}
	Пусть $A(t)$ --- последовательность матриц размером $k \times k$, и $A(t) \rightarrow A$ при $t \rightarrow \infty$, причём $A > 0$, и её перронов корень $r = 1$. Пусть $b(t) = b t^\alpha (1 + o(1))$ --- последовательность векторов длины $k$, где $b \geq 0$, $b \neq 0$ и $\alpha$ --- действительное число. Тогда для последовательности векторов $x(t)$ при $t = 1, 2, \ldots$, определяемой рекуррентным соотношением $x(t) = b(t) + A(t) x(t-1)$ при $t \rightarrow \infty$ справедливо соотношение
	\begin{equation*}
		\frac{x_i(t)}{v x(t)} \rightarrow u_i,
	\end{equation*}
	при условии что $x(t_0) > 0$ для некоторого номера $t_0$, где $u, v > 0$ --- соответственно правый и левый собственные векторы матрицы $A$ при нормировке $vu = 1$.
\end{lemma}
Доказательство леммы принадлежит А.~Борисову.

\begin{lemma}
\label{l:xab}
	Пусть последовательность $x_t$, $x_t > 0$ при любом $t \geq 0$, удовлетворяет рекуррентному соотношению
	\begin{equation*}
		x_{t+1} = \alpha t^\alpha (1 + \epsilon_1(t)) + (1 - b t^\beta (1 + \epsilon_2(t))) x_t,
	\end{equation*}
	где $\beta < 0$, $b > 0$, и $\epsilon_1(t), \epsilon_2(t) = o(1)$ при $t \rightarrow \infty$. Тогда верны следующие асимптотические равенства:
	\begin{align}
		&(1)\quad x_t = \frac{\alpha t^{\alpha + 1}}{\alpha + 1} (1 + o(1))     & &\text{при}\quad \beta < -1, \alpha \geq 0 \\
		&(2)\quad x_t = \frac{\alpha t^{\alpha + 1}}{\alpha + b + 1} (1 + o(1)) & &\text{при}\quad \beta = -1, \alpha > -1 \\
		&(3)\quad x_t = \frac{\alpha t^{\alpha - \beta}}{b} (1 + o(1))          & &\text{при}\quad -1 < \beta < 0
	\end{align}
\end{lemma}
Доказательство леммы принадлежит А.~Борисову.

Вначале рассмотрим $\overline{M}^q_{ij}(t)$. Пусть $p(\cdot)$ --- вероятность дерева $d$ в грамматике $G$. Рассмотрим множество $D^{\leq t}_{ql}$ деревьев из $D^{\leq t}_q$, первый ярус которых получен применением правила $r_{ql}$ к корню дерева. Пусть
\begin{equation*}
	\overline{P}^{ij}_{ql}(t) = \sum_{d \in D^{\leq t}_{ql}} p(d) q_{ij}(d),
\end{equation*}
где $q_{ij}(d)$ --- число применений правила $r_{ij}$ в дереве $d$, и $\overline{P}^{ij}_{ql}(t)$ --- вклад деревьев из $D^{\leq t}_{ql}$ в матожидание $\overline{M}^q_{ij}(t)$. Для краткости, обозначим $\overline{P}_{ql} = \overline{P}^{ij}_{ql}$. Тогда
\begin{equation}
\label{eq:m_line_sum}
	\overline{M}^q_{ij}(t) = \sum_{l = 1}^{n_q} \overline{P}_{ql}(t).
\end{equation}
Рассмотрим величину $\overline{P}_{ql}(t)$. Пусть
\begin{equation*}
	q_{ij}(d) = q^{(1)}_{ij}(d) + q^{(2)}_{ij}(d),
\end{equation*}
где $q^{(1)}_{ij}(d)$ --- число применений правила $r_{ij}$ в дереве $d$ на первом его ярусе, а $q^{(2)}_{ij}(d)$ --- на остальных ярусах. Тогда
\begin{equation*}
	\overline{P}_{ql}(t) = \sum_{d \in D^{\leq t}_{ql}} p(d) q_{ij}(d) = \sum_{d \in D^{\leq t}_{ql}} p(d) q^{(1)}_{ij}(d) + \sum_{d \in D^{\leq t}_{ql}} p(d) q^{(2)}_{ij}(d) = \overline{P}^{(1)}_{ql} + \overline{P}^{(2)}_{ql}(t)
\end{equation*}
Очевидно, $q^{(1)}_{ij}(d) = \delta^q_i \delta^l_j$ (где $\delta$ --- символ Кронекера). Тогда
\begin{equation*}
	\overline{P}^{(1)}_{ql}(t) = \delta^q_i \delta^l_j \frac{p_{ij} Q_{s_{ij}}(t-1)}{1 - Q_q(t)},
\end{equation*}
где $Q_X(t)$ --- вероятность наборов деревьев вывода высоты не превосходящей $t-1$, набор корней которых задан характеристическим вектором $X \in \mathbb{N}^k$.

Обозначим также $\delta^i(n) = \left. \left( \delta^i_k \right) \right|_{i = \overline{1, n}} \in {0, 1}^n$.

Вероятность дерева $p(d)$ при $d \in D^{\leq t}_{ql}$ можно выразить как
\begin{equation*}
	p(d) = \frac{p_{ql}}{1 - Q_q(t)} p_1(d) p_2(d) \ldots p_{\overline{s_{ql}}}(d),
\end{equation*}
где $p_j(d)$ --- вероятность поддерева $d$ с корнем в $j$-м узле первого яруса. Тогда
\begin{equation*}
	\overline{P}^{(2)}_{ql}(t) = \frac{p_{ql}}{1 - Q_q(t)} \sum_{d \in D^{\leq t}_{ql}} \prod_{n=1}^{\overline{s}_{ql}} p_n(d) \sum_{m = 1}^{\overline{s}_{ql}} q'^m_{ij}(d),
\end{equation*}
где $q'^m_{ij}(d)$ --- число применений правила $r_{ij}$ в поддереве дерева $d$ с корнем в $m$-том нетерминале первого яруса.

Выделим в $d$ поддеревья $d_1, d_2, \ldots, d_{\overline{s}_{ql}}$, где $d_j$ --- поддерево с корнем в $j$-м узле первого яруса дерева $d$. Тогда
\begin{multline*}
	\overline{P}^{(2)}_{ql}(t) = \frac{p_{ql}}{1 - Q_q(t)} \sum_{m = 1}^{\overline{s}_{ql}} \sum_{d \in D^{\leq t}_{ql}} \left( \prod_{n = 1}^{\overline{s}_{ql}} p_n(d) \right) q'^m_{ij}(d) = \\
	= \frac{p_{ql}}{1 - Q_q(t)} \sum_{m = 1}^{\overline{s}_{ql}} \sum_{d_j : j \neq m} p_1(d) \ldots p_{m-1}(d_{m-1}) p_{m+1}(d_{m+1}) \ldots p_{\overline{s}_{ql}}(d_{\overline{s}_{ql}}) q'^m_{ij}(d) = \\
	= \frac{p_{ql}}{1 - Q_q(t)} \sum_{m = 1}^{\overline{s}_{ql}} Q_{s_{ql} - \delta^m} q_{ij}(d_m) = \frac{p_{ql}}{1 - Q_q(t)} \sum_{m = 1}^k s^m_{ql} \overline{M}^m_{ij}(t-1) Q_{s_{ql} - \delta^m}(t - 1)
\end{multline*}
Зная $\overline{P}_{ql}(t) = \overline{P}^{(1)}_{ql}(t) + \overline{P}^{(2)}_{ql}(t)$, получаем
\begin{equation*}
	\overline{M}^q_{ij}(t) = \frac{1}{1 - Q_q(t)} \left( \delta^q_i p_{ij} Q_{s_{ij}}(t - 1) + \sum_{l = 1}^{n_q} p_{ql} \sum_{m = 1}^k s^m_{ql} \overline{M}^m_{ij}(t - 1) Q_{s_{ql} - \delta^m}(t - 1) \right)
\end{equation*}
Обозначая
\begin{equation*}
	\overline{M}'^q_{ij}(t) = \overline{M}^q_{ij}(t) (1 - Q_q(t)),
\end{equation*}
имеем
\begin{equation}
\label{eq:m_line_checkpoint}
	\overline{M}'^q_{ij}(t) = \delta^q_i p_{ij} Q_{s_{ij}}(t - 1) + \sum_{l = 1}^{n_q} p_{ql} \sum_{m = 1}^k s^m_{ql} \overline{M}'^m_{ij}(t - 1) Q_{s_{ql} - \delta^m}(t - 1)
\end{equation}
Рекуррентное соотношение $(\ref{eq:m_line_checkpoint})$ является опорной точкой для вычисления $\overline{M}^q_{ij}(t)$. Получим аналогичное уравнение для $M^q_{ij}(t)$.

Аналогично $(\ref{eq:m_line_sum})$
\begin{equation*}
	M^q_{ij}(t) = \sum_{l = 1}^{n_q} P_{ql}(t),
\end{equation*}
где $P_{ql}(t)$ --- вклад деревьев из $D^t_{ql}$ в матожидание $M^q_{ij}(t)$. Положим $P_{ql}(t) = P^{(1)}_{ql}(t) + P^{(2)}_{ql}(t)$, аналогично тому, как это сделано для $\overline{P}_{ql}(t)$. При этом
\begin{equation*}
	P^{(1)}_{ql}(t) = \delta^q_i \delta^l_j \frac{p_{ij} R_{s_{ij}}(t - 1)}{P_q(t)},
\end{equation*}
где $R_X(t)$ --- вероятность наборов деревьев из $D^{\leq t}$, набор корней которых задан характеристическим вектором $X$, и высота хотя бы одного из которых достигает $t - 1$. $P^{(2)}_{ql}(t)$ можно представить в виде
\begin{equation*}
	P^{(2)}_{ql}(t) = \sum_{m = 1}^{\overline{s}_{ql}} P^{(2)m}_{ql}(t),
\end{equation*}
где $P^{(2)m}_{ql}(t)$ --- вклад деревьев с $m$-м корнем на первом ярусе в $M^q_{ij}(t)$.

Обозначим через $S_1$ вклад в $P^{(2)m}_{ql}(t)$ наборов деревьев, в которых ярус $t$ достигается деревом с корнем в $m$-м нетерминале первого яруса. Очевидно,
\begin{equation*}
	S_1 = \frac{(1 - Q_{z_m}(t-1)) Q_{s_{ql} - \delta^{z_m}}(t - 1) M^{z_m}_{ij}(t - 1)}{P_q(t)},
\end{equation*}
где $z_m$ --- $m$-й нетерминал первого яруса.

Пусть $S_2$ --- вклад наборов, где ярус $t$ достигается через другие деревья. Тогда
\begin{equation*}
	S_2 = \frac{(1 - Q_{z_m}(t - 1)) R_{s_{ql} - \delta^m}(t - 1) \overline{M}^{z_m}_{ij}(t - 1)}{P_q(t)}
\end{equation*}
В результате, для $M^q_{ij}(t)$ получаем
\begin{multline*}
	M^q_{ij}(t) = \sum_{l = 1}^{n_q} \left( P^{(1)}_{ql}(t) + \sum_{m = 1}^{\overline{s}_{ql}} P^{(2)m}_{ql}(t) \right) = \\
	= \frac{1}{P_q(t)} [ \delta^q_i p_{ij} R_{s_{ij}}(t - 1) + \sum_{l = 1}^{n_q} p_{ql} \sum_{m = 1}^k (P_m(t - 1) Q_{s_{ql} - \delta^m}(t - 1) M^m_{ij}(t - 1) + \\
	+ (1 - Q_m(t - 1)) R_{s_{ql} - \delta^m}(t - 1) \overline{M}^m_{ij}(t - 1)) ]
\end{multline*}

Из леммы $(\ref{l:first})$ следуют выражения для $Q_X(t)$ и $R_X(t)$
\begin{equation}
\label{eq:qx_rx}
	\begin{split}
		&Q_X(t) = \prod_{i = 1}^k (1 - Q_i(t))^{x_i} = 1 - \sum_{i = 1}^k x_i Q_i(t) + \Theta \left( \sum_{1 \leq i,j \leq k} x_i x_j Q_i(t) Q_j(t) \right) \\
		&R_X(t) = Q_X(t) - Q_X(t - 1) = \sum_{i = 1}^k x_i P_i(t) + \Theta \left( \sum_{1 \leq i,j \leq k} x_i x_j Q_i(t) Q_j(t) \right)
	\end{split}
\end{equation}
Теперь приступим к вычислению $\overline{M}'^q_{ij}(t)$ и $M'^q_{ij}(t)$.

Пусть вначале $A_q$ и $A_i$ принадлежат классам, находящимся в одной группе $\mathcal{M}_n$. Тогда $\overline{M}'^{(\alpha)}_{ij}(t) = 0$ для всех $A_\alpha \in K \in \mathcal{M}_m$, таких что $m > n$. Для $\overline{M}'^{(n)}_{ij}(t)$ получаем:
\begin{equation*}
	\overline{M}'^q_{ij}(t) = \delta^q_i p_{ij} Q_{s_{ij}}(t - 1) + \sum_{l = 1}^{n_q} p_{ql} \sum_{\alpha : A_\alpha \in K \in \mathcal{M}_n} s_{ql}^\alpha \overline{M}'^\alpha_{ij}(t - 1) Q_{s_{ql} - \delta^\alpha}(t-1)
\end{equation*}
Подставляя выражение $(\ref{eq:qx_rx})$ для $Q_X(t)$, получаем
\begin{multline}
\label{eq:m_line_no_ab}
	\overline{M}'^q_{ij}(t) = \delta^q_i p_{ij} (1 + o(1)) \sum_{\alpha : A_\alpha \in K \in \mathcal{M}_n} \sum_{l = 1}^{n_q} p_{ql} s_{ql}^\alpha \overline{M}'^\alpha_{ij}(t-1) - \\
	- \sum_{\alpha : A_{\alpha} \in K \in \mathcal{M}_n} \sum_{l = 1}^{n_q} p_{ql} s_{ql}^\alpha \overline{M}'^\alpha_{ij}(t-1) \sum_{\beta : A_\beta \in K \in \mathcal{M}_n} (s_{ql}^\beta - \delta^\alpha_\beta) Q_\beta(t-1) (1 + o(1))
\end{multline}

Непосредственной проверкой устанавливается, что
\begin{equation}
\label{eq:ab}
	\begin{split}
		&a^q_\alpha = \sum_{l = 1}^{n_q} p_{ql} s_{ql}^\alpha \\
		&b^q_{\alpha \beta} = \sum_{l = 1}^{n_q} p_{ql} s_{ql}^\alpha (s_{ql}^\beta - \delta^\alpha_\beta)
	\end{split}
\end{equation}

Заменяя соответствующие выражения в $(\ref{eq:m_line_no_ab})$, а также подставляя асимптотику для $Q^{(n)}(t)$, получаем
\begin{multline}
\label{eq:m_line_ab}
	\overline{M}'^q_{ij}(t) = \delta^q_i p_{ij} (1 + o(1)) + \sum_{\alpha : A_\alpha \in K \in \mathcal{M}_n} a^q_\alpha \overline{M}'^\alpha_{ij}(t-1) - \\
	- c_n t^{-\left(\frac{1}{2}\right)^{w-n}} \sum_{\alpha,\beta} b^q_{\alpha \beta} U_\beta \overline{M}'^\alpha_{ij}(t-1) (1 + o(1))
\end{multline}

Применяя лемму $(\ref{l:Ab})$ для вектора $(\overline{M}'^{q_1}_{ij}(t), \overline{M}'^{q_2}_{ij}(t), \ldots, \overline{M}'^{q_\gamma}(t))$, где $A_{q_1}, A_{q_2}, \ldots, A_{q_\gamma}$ --- нетерминалы классов группы $\mathcal{M}_n$, имеем
\begin{equation*}
\label{eq:muv}
	\overline{M}'^q_{ij}(t) = U_q \sum_{l : A_l \in K \in \mathcal{M}_n} V_l \overline{M}'^l_{ij}(t) = U_q M^{(n)}_*(t).
\end{equation*}

Домножая $(\ref{eq:m_line_ab})$ на $V^{(n)}$ слева, получаем
\begin{equation*}
	\delta \overline{M}^{(n)}_*(t) = V_i p_{ij} (1 + o(1)) - c_n t^{-\left(\frac{1}{2}\right)^{w-n}} \sum_{q,\alpha,\beta} V_q b^q_{\alpha \beta} U_\alpha U_\beta = V_i p_{ij} - c_n t^{-\left(\frac{1}{2}\right)^{w-n}} B_n
\end{equation*}

Нетрудно видеть, что величина $\overline{M}^{(n)}_*(t)$ удовлетворяет условиям леммы $(\ref{l:xab})$. Применяя её, получаем
\begin{align*}
	&\overline{M}^{(n)}_*(t) = \left( \frac{V_i p_{ij}}{c_n B_n + 1} \right) \cdot t \cdot (1 + o(1)), & &\text{если}\quad n = w \\
	&\overline{M}^{(n)}_*(t) = \left( \frac{V_i p_{ij}}{c_n B_n} \right) \cdot t^{-\left(\frac{1}{2}\right)^{w-n}} \cdot (1 + o(1)), & &\text{если}\quad n < w
\end{align*}

Пусть теперь классы, содержащие $A_q$ и $A_i$, находятся в различных группах. Тогда
\begin{equation*}
	\overline{M}'^q_{ij}(t) = \delta^q_i p_{ij} Q_{s_{ij}}(t-1) + \sum_{\alpha : A_\alpha \in K \in \mathcal{M}_n \cup \mathcal{M}_{n+1}} \sum_{l = 1}^{n_q} p_{ql} s_{ql}^\alpha \overline{M}'^\alpha_{ij}(t-1) Q_{s_{ql} - \delta^\alpha}(t-1)
\end{equation*}
Учитывая $Q^{(n+1)}(t) = o(Q^{(n)}(t))$, получаем
\begin{multline*}
\label{eq:muv}
	\overline{M}'^q_{ij}(t) = O(p_{ij}) + \\
	+ \sum_{\alpha : A_\alpha \in K \in \mathcal{M}_n} \sum_{l = 1}^{n_q} p_{ql} s_{ql}^\alpha \overline{M}'\alpha_{ij}(t-1) \left( 1 - \sum_{\beta : A_\beta \in K \in \mathcal{M}_n} (s_{ql}^\alpha - \delta^\alpha_\beta) Q_\beta(t-1) \right) (1 + o(1)) + \\
	+ \sum_{\alpha : A_\alpha \in K \in \mathcal{M}_{n+1}} \sum_{l = 1}^{n_q} p_{ql} s_{ql}^\alpha \overline{M}'^\alpha_{ij}(t-1) (1 + o(1))
\end{multline*}

Положим $\overline{M}'^\alpha_{ij}(t-1) = \overline{d}'_{n+1} t^{\gamma(n+1)} (1 + o(1))$, что выполняется для $n+1 = w$. Подставляя это выражение, а также $(\ref{eq:ab})$, получаем
\begin{multline*}
	\overline{M}'^q_{ij}(t) = \sum_{\alpha : A_\alpha \in K \in \mathcal{M}_n} a^q_\alpha \overline{M}'^\alpha_{ij}(t-1) - c_n t^{-\left(\frac{1}{2}\right)^{w-n}} \sum_{\alpha,\beta} b^q_{\alpha \beta} U_\beta \overline{M}'^\alpha_{ij}(t-1) (1 + o(1)) + \\
	+ \overline{d}'_{n+1} t^{\gamma(n+1)} \sum_{\alpha : A_\alpha \in K \in \mathcal{M}_{n+1}} a^q_\alpha U_\alpha (1 + o(1))
\end{multline*}
Домножая на $V_q$, имеем
\begin{multline*}
	\delta \overline{M}^{(n)}_*(t) = \overline{d}'_{n+1} \left( \sum_{\substack{q : A_q \in K \in \mathcal{M}_n \\ \alpha : A_\alpha \in K \in \mathcal{M}_{n+1}}} V_q a^q_\alpha U_\alpha \right) \cdot t^{\gamma(n+1)} - \\
	- c_n t^{-\left(\frac{1}{2}\right)^{w-n}} \cdot \left( \sum_{q,\alpha,\beta} V_q b^q_{\alpha \beta} U_\alpha U_\beta \right) \cdot \overline{M}^{(n)}_*(t-1) (1 + o(1)) = \\
	= \overline{d}'_{n+1} b_{n+1} t^{\gamma(n+1)} (1 + o(1)) - c_n B_n t^{-\left(\frac{1}{2}\right)^{w-n}} \overline{M}^{(n)}_*(t-1) (1 + o(1))
\end{multline*}
Рассматривать случай $n = w$ не имеет смысла, поэтому в выражении $t^{-\left(\frac{1}{2}\right)^{w-n}}$ показатель всегда будет больше $-1$. Учитывая это, и применяя лемму $(\ref{l:xab})$, получаем
\begin{equation*}
	\overline{M}^{(n)}_*(t) = \frac{\overline{d}'_{n+1} b_{n+1} t^{\gamma(n+1) + \left(\frac{1}{2}\right)^{w-n}}}{c_n B_n}
\end{equation*}
Отсюда,
\begin{equation*}
	\overline{M}^{(n)}_*(t) = \prod_{j = n}^{h-1} \left( \frac{b_{j+1}}{c_j B_j} \right) \cdot \left( \frac{V_i p_{ij}}{c_h B_h + \delta^\alpha_h} \right) \cdot t^{\left( \left(\frac{1}{2}\right)^{\alpha - h - 1} - \left(\frac{1}{2}\right)^{\alpha - n} \right)}
\end{equation*}

Подставляя $(\ref{eq:muv})$ и $\overline{M}'^q_{ij}(t) = \overline{M}^q_{ij}(t) (1 - Q_q(t))$, получаем
\begin{equation*}
	\overline{M}^{(n)}_*(t) = \frac{U_q}{1 - Q_q(t)} \prod_{j = n}^{h-1} \left( \frac{b_{j+1}}{c_j B_j} \right) \cdot \left( \frac{V_i p_{ij}}{c_h B_h + \delta^\alpha_h} \right) \cdot t^{\left( \left(\frac{1}{2}\right)^{\alpha - h - 1} - \left(\frac{1}{2}\right)^{\alpha - n} \right)}
\end{equation*}

Перейдём к вычислению $M^q_{ij}(t)$. Вначале пусть нетерминалы $A_q$ и $A_i$ принадлежат классам из одной группы $\mathcal{M}_n$. Полагая $M'q_{ij}(t) = M^q_{ij}(t) P_q(t)$, получаем
\begin{multline*}
	M'q_{ij}(t) = O(t^{-1 -\left(\frac{1}{2}\right)}) + \sum_{\alpha : Q_\alpha \in K \in \mathcal{M}} \sum_{l=1}^{n_q} p_{ql} s_{ql}^\alpha M'^\alpha_{ij}(t-1) - \\
	- \sum_{\alpha,\beta : A_\alpha, A_\beta \text{в группе} \mathcal{M}_n} \sum_{l=1}^{n_q} p_{ql} s_{ql}^\alpha (s_{ql}^\beta - \delta^\alpha_\beta) Q_n(t-1) M'^\alpha_{ij}(t-1) + \\
	+ \sum_{\alpha,\beta} \sum_{l=1}^{n_q} p_{ql} s_{ql}^\alpha (s_{ql}^\beta - \delta^\alpha_\beta) P_n(t-1) \overline{M}'^\alpha_{ij}(t-1)
\end{multline*}
Подставляя выражение $(\ref{eq:ab})$ для первых и вторых моментов, получаем
\begin{multline*}
	M'^q_{ij}(t) = \sum_{\alpha : A_\alpha \in K \in \mathcal{M}_n} q^q_\alpha M'^\alpha_{ij}(t-1) - c_n t^{-\left(\frac{1}{2}\right)^{w-n}} \sum_{\alpha,\beta} b^q_{\alpha \beta} U_\beta M'^\alpha_{ij}(t-1) (1 + o(1)) + \\
	+ \tilde{c}_n t^{-1 - \left(\frac{1}{2}\right)^{w-n}} \sum_{\alpha,\beta} b^q_{\alpha \beta} U_\beta \overline{M}^\alpha_{ij}(t-1) (1 + o(1))
\end{multline*}
Подставляя выражение для $\overline{M}'^\alpha_{ij}(t-1)$, имеем
\begin{multline}
\label{eq:mqvec}
	M'^q_{ij}(t) = \sum_{\alpha} a^q_\alpha M'^\alpha_{ij}(t-1) - c_n t^{-\left(\frac{1}{2}\right)} \sum_{\alpha,\beta} b^q_{\alpha \beta} U_\beta M'^\alpha_{ij}(t-1) (1 + o(1)) + \\
	+ \tilde{c}_n d_n t^{-1} \sum_{\alpha,\beta} b^q_{\alpha \beta} U_\alpha U_\beta (1 + o(1))
\end{multline}
Применяя лемму $(\ref{l:Ab})$, получаем
\begin{equation*}
	\begin{split}
		M'q_{ij}(t) = U_q M^{(n)}_*(t) (1 + o(1)) \\
		M^{(n)}_*(t) = \sum_{\alpha : A_\alpha \in K \in \mathcal{M}_n} V_\alpha M'^\alpha_{ij}(t)
	\end{split}
\end{equation*}
Домножая $(\ref{eq:mqvec})$ на $V^{(n)}$, получаем
\begin{equation*}
	\delta V^{(n)}_*(t) = \tilde{c}_n d_n B_n t^{-1} - c_n t^{-\left(\frac{1}{2}\right)^{w-n}} B_n V^{(n)}_*(t)(t-1) (1 + o(1))
\end{equation*}
Применяя лемму $(\ref{l:xab})$, получаем в результате
\begin{equation*}
	M^{(n)}_*(t) = \left\{
	\begin{split}
		&\tilde{c}_n \overline{d}'_n B_n (1 + o(1)),\quad \text{при}\quad \alpha = n \\
		&\frac{\tilde{c}_n \overline{d}'_n}{c_n} t^{-1 -\left(\frac{1}{2}\right)^{w-n}} (1 + o(1)),\quad \text{при}\quad \alpha > n
	\end{split}
	\right.
\end{equation*}

Пусть теперь $A_q$ и $A_i$ находятся в классах, принадлежащих разным группам $\mathcal{M}_n$ и $\mathcal{M}_h$ ($h > m$). Тогда
\begin{multline*}
	M'^q_{ij}(t) = \delta^q_i p_{ij} R_{s_{ij}}(t-1) + \sum_{\alpha: A_\alpha \in K \in \mathcal{M}_n \cup \mathcal{M}_{n+1}} \sum_{l=1}^{n_q} p_{ql} s_{ql}^\alpha [Q_{s_{ql}}(t-1) M'^\alpha_{ij}(t-1) + \\
	+ (1 - Q_\alpha(t-1)) R_{s_{ql} - \delta^\alpha}(t-1) \overline{M}^{\alpha}_{ij}(t-1)]
\end{multline*}
откуда
\begin{multline*}
	M'^q_{ij}(t) = \sum_{\alpha : A_\alpha \in K \in \mathcal{M}_n} a^q_\alpha M'^\alpha_{ij}(t-1) - \sum_{\alpha,\beta\; \text{в группе}\; \mathcal{M}_n} b^q_{\alpha \beta} Q_\beta(t-1) M'^\alpha_{ij}(t-1) (1 + o(1)) + \\
	+ \sum_{\alpha,\beta\; \text{в группе}\; \mathcal{M}_n} b^q_{\alpha \beta} P_\beta(t-1) \overline{M}^\alpha_{ij}(t-1) (1 + o(1))
\end{multline*}
Домножая на $V^{(n)}$, имеем
\begin{multline*}
	\delta M^{(n)}_*(t) = \tilde{c}_n \overline{d}_n B_n t^{-1 -\left(\frac{1}{2}\right)^{w-n} + \left(\frac{1}{2}\right)^{w-h} \left(2 - \left(\frac{1}{2}\right)^{h - n}\right)} (1 + o(1)) - \\
	- c_n B_n t^{-\left(\frac{1}{2}\right)^{w-n}} M^{(n)}_*(t-1) (1 + o(1))
\end{multline*}
Поскольку $n < w$, показатель степени в выражении $t^{-\left(\frac{1}{2}\right)}$ всегда больше $-1$, и по лемме $(\ref{l:xab})$ получаем
\begin{equation*}
	M^{(n)}_*(t) = \frac{\tilde{c}_n \overline{d}_n}{c_n} t^{-1 + \left(\frac{1}{2}\right)^{w-h} \left(2 - \left(\frac{1}{2}\right)^{h-n}\right)}
\end{equation*}
Объединяя результаты для $n < w$ и $n = w$, получаем
\begin{equation*}
	M^{(n)}_*(t) = \frac{\tilde{c}_n \overline{d}_n B_n}{\delta^n_w (c_n B_n - 1) + 1} \cdot t^{-1 + \left(\frac{1}{2}\right)^{w-h} \left(2 - \left(\frac{1}{2}\right)^{h-n} \right)} (1 + o(1))
\end{equation*}
Откуда
\begin{equation*}
	M^{(n)}_*(t) = \frac{U_q}{P_q(t)} \frac{\tilde{c}_n \overline{d}_n B_n}{\delta^n_w (c_n B_n - 1) + 1} \cdot t^{-1 + \left(\frac{1}{2}\right)^{w-h} \left(2 - \left(\frac{1}{2}\right)^{h-n} \right)} (1 + o(1))
\end{equation*}

\section{Энтропия}
Пусть $L^t$ --- множество слов языка $L_G$, порождаемых деревьями вывода из $D^t$. Будем рассматривать грамматики с однозначным выводом.

По определению, энтропия языка $L^t$ есть
\begin{equation*}
	H(L^t) = - \sum_{\alpha \in L^t} p_t(\alpha) \log p_t(\alpha),
\end{equation*}
где $p_t(\alpha) = p(\alpha : \alpha \in L^t) = p(\alpha) / p(L^t)$. Используя это выражение для $p_t(\alpha)$, получаем
\begin{multline*}
	H(L^t) = - \frac{1}{P(L^t)} \sum_{\alpha \in L^t} p_t(\alpha) \left( \log p(\alpha) - \log P(L^t) \right) = \\
	= \frac{\log P(L^t)}{P(L^t)} \sum_{\alpha \in L^t} p(\alpha) - \frac{1}{P(L^t)} \sum_{\alpha \in L^t)} p_t(\alpha) \log p(\alpha) = \\
	= \log P(L^t) - \frac{1}{P(L^t)} \sum_{\alpha \in L^t} p(\alpha) \log p(\alpha)
\end{multline*}
Выразим вероятность слова $\alpha$ через вероятности правил вывода $r_{ij}$. Поскольку рассматривается грамматика с однозначным выводом, каждому слову $\alpha$ из $L^t$ соответствует единственное дерево $d(\alpha)$ из $D^t$ и единственный левый вывод $\omega_l(\alpha) = (r_{i_1,j_1}, r_{i_2,j_2}, \ldots, r_{i_s,j_s})$. Получаем
\begin{equation*}
	p(\alpha) = p(r_{i_1,j_1}) \cdot \ldots \cdot p(r_{i_s,j_s}) = \prod_{i=1}^k \prod_{j=1}^{n_i} p_{ij}^{q_{ij}(\alpha)},
\end{equation*}
где $q_{ij}(\alpha)$ --- число применений правила $r_{ij}$ при выводе слова $\alpha$ (учитывая единственность дерева вывода, это число определяется единственным образом). Тогда
\begin{equation*}
	\sum_{\alpha \in L^t} p(\alpha) \log p(\alpha) = \sum_{\alpha \in L^t} p(\alpha) \sum_{i=1}^k \sum_{j=1}^{n_i} q_{ij}(\alpha) \log p_{ij} = \sum_{i=1}^k \sum_{j=1}^{n_i} \log p_{ij} \sum_{\alpha \in L^t} q_{ij}(\alpha) p(\alpha)
\end{equation*}
Пользуясь определением $M(S_{ij}(t))$, получаем
\begin{equation*}
	\sum_{\alpha \in L^t} p(\alpha) \log p(\alpha) = \sum_{i=1}^k \sum_{j=1}^{n_i} \log p_{ij} M(S_{ij}(t)) P(L^t)
\end{equation*}
Отсюда
\begin{equation*}
	H(L^t) = \log P(L^t) - \sum_{i=1}^k \sum_{j=1}^{n_i} M(S_{ij}(t)) \log p_{ij} (1 + o(1))
\end{equation*}
По определению, $P(L^t) = P_1(t) = O(t^{-1 -\left(\frac{1}{2}\right)^{w-1}})$, и $\log P(L^t) = O(\log t)$. Подставляя выражение для $M(S_{ij}(t)) = M_{ij}(t)$, получаем
\begin{equation*}
	H(L^t) = \sum_{i=1}^k \sum_{j=1}^{n_i} H(R_i) d_i t^2 (1 + o(1)),
\end{equation*}
где $H(R_i) = - \sum_{j=1}^{n_i} p_{ij} \log p_{ij}$ --- энтропия множества $R_i$ правил вывода. Асимптотика $t^2$ задаётся величиной $M^q_{ij}(t)$ для последнего критического класса и классов, следующих за ними.

Сформулируем теорему:
\begin{theorem}
	Энтропия языка $L^t$, состоящего из слов, порождаемых в разложимой стохастической КС-грамматике вида <<цепочки>> с однозначным выводом деревьями высоты $t$, выражается формулой
	\begin{equation*}
		H(L^t)  \sum_{i \in I} \sum_{j=1}^{n_i} d_i H(R_i) \cdot t^2,
	\end{equation*}
	где $d_i > 0$, $H(R_i) = \sum_{j=1}^{n_i} p_{ij} \log p_{ij}$ --- энтропия множества $R_i$ правил вывода с нетерминалов $A_i$ в левой части, и $I$ --- множество индексов нетерминалов, содержащихся в последнем критическом классе, а также классах, следующих за ним.
\end{theorem}

\section{Заключение}

В результате проведённого исследования были изучены основные вероятностные характеристики грамматик заданного класса. Полученные асимптотические оценки позволяют непосредственно перейти к построению алгоритма асимптотически оптимального кодирования для рассматриваемого класса языков сообщений, а также существенно упрощают исследование этой задачи для КС-языков в общем случае.

\newpage

\begin{thebibliography}{99}
	\bibitem{shennon-mts}
	\textbf{Шеннон К.} Математическая теория связи. М.: ИЛ, 1963
	\bibitem{markov-coding}
	\textbf{Марков А. А.} Введение в теорию кодирования. М.: Наука, 1982
	\bibitem{fu-struct}
	\textbf{Фу К.} Структурные методы в распознавании образов. М.: Мир, 1977
	\bibitem{aho-ulman-syntax}
	\textbf{Ахо А., Ульман Дж.} Теория синтаксического анализа, перевода и компиляции. Том 1. М.: Мир, 1978
	\bibitem{sevast-processes}
	\textbf{Севастьянов Б. А.} Ветвящиеся процессы. --- M.: Наука, 1971 --- 436 с.
	\bibitem{gantmaher-matrix-theory}
	\textbf{Гантмахер Ф. Р.} Теория матриц. --- 5-е изд., --- М.: ФИЗМАТЛИТ, 2010
	\bibitem{zhiltsova-about-matrix}
	\textbf{Жильцова Л. П.} О матрице первых моментов разложимой стохастической КС-грамматики. УЧЁНЫЕ ЗАПИСКИ КАЗАНСКОГО ГОСУДАРСТВЕННОГО УНИВЕРСИТЕТА, Том 151, кн. 2, 2009
	\bibitem{zhiltsova-zakonom}
	\textbf{Жильцова Л. П.} Закономерности применения правил грамматики в выводах слов стохастического контекстно-свободного языка // Математические вопросы кибернетики. Выр. 9. М.: Наука, 2000. С. 100-126.
	\bibitem{zhiltsova-cost}
	\textbf{Жильцова Л. П.} О нижней оценке стоимости кодирования и асимптотически оптимальном кодировании стохастического контекстно-свободного языка // Дискретный анализ и исследование операций. Серия 1, т. 8, \No 3. Новосибирск: Издательство Института математики СО РАН, 2001. С. 26-45.
	\bibitem{borisov-zakonom}
	\textbf{Борисов А. Е.} Закономерности в словах стохастических контекстно-свободных языков, порождённых грамматиками с двумя классами нетерминальных символов. Вопросы экономного кодирования.
\end{thebibliography}

\end{document}
