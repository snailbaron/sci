\documentclass{beamer}
\usepackage[utf8]{inputenc}
\usepackage[russian]{babel}
\usepackage{sansmathaccent}
\usepackage{hyperref}
\hypersetup{unicode=true}
\usepackage{units}
\usepackage{multicol}
\usepackage{qtree}
\usepackage{graphics}

\renewcommand{\leq}{\leqslant}
\renewcommand{\geq}{\geqslant}

\begin{document}
	\setbeamertemplate{navigation symbols}{}
	
	% Титульный слайд
	\begin{frame}
		\begin{center}
			{\LARGE О числе нетерминалов в деревьях вывода разложимой стохастической КС-грамматики }
		\end{center}
		
		\vfill
		
	
		\begin{center}
			{\Large Матрынов Игорь Михайлович}
			
			\vfill
			
			{\small Нижегородский государственный университет им. Н.И.~Лобачевского}
		\end{center}
	\end{frame}
	
	\setbeamertemplate{footline}[text line]
	{
		\parbox{\linewidth}{\vspace*{-8pt} \color{gray} Мартынов И.М.\hfill О числе нетерминалов в деревьях вывода \hfill\insertpagenumber}
	}

	\begin{frame}
		\frametitle{КС-грамматика (КСГ)}
		$$
			G = <V_N, V_T, S, R>
		$$
		
		$V_N$ и $V_T$ --- конечные алфавиты терминальных и нетерминальных символов
		$S \in V_N$ --- аксиома
		$R = \cup_{i=1}^k R_i$, где $R_i$ --- множество правил вида:
		\begin{equation*}
		r_{ij} : A_i \rightarrow \beta_{ij}\quad (j = 1,2,\ldots,n_i)
		\end{equation*}
		$\beta_{ij} \in (V_N \cup V_T)^*$
		
		Грамматика порождает слово $\alpha$, если:
		$$
			S \rightarrow \beta_1 \rightarrow \ldots \rightarrow \alpha,
		$$
		где $\rightarrow$ означает замену левой части некоторого правила на его правую часть.
	\end{frame}

	\begin{frame}
		\frametitle{Пример вывода слова}
		\framesubtitle{Язык арифметических выражений c a,b без скобок}
		\begin{multicols}{2}
			{\small
				\begin{equation*}
				\begin{split}
				&V_N = \{E, M, T\}\\
				&V_T = \{+, -, *, /, a, b\}\\
				&S = E\\
				&{} \\
				&\text{Правила вывода:} \\
				&r_{11} : E \rightarrow M + E \\
				&r_{12} : E \rightarrow M - E \\
				&r_{13} : E \rightarrow M \\
				&r_{21} : M \rightarrow T * M \\
				&r_{22} : M \rightarrow T / M \\
				&r_{23} : M \rightarrow T \\
				&r_{31} : T \rightarrow a \\
				&r_{32} : T \rightarrow b
				\end{split}
				\end{equation*}
			}
			
			$$
				\alpha = a + b * a - b
			$$
			
			{\footnotesize
				\renewcommand{\qtreepadding}{1.8pt}
				\Tree [ .E [ .M [ .T a ] ] + [ .E [ .M [ .T b ] * [ .M [ .T a ] ] ] - [ .E [ .M [ .T b ] ] ] ] ]
			}
			
			\begin{multline*}
				E \rightarrow M + E \rightarrow T + E \rightarrow \ldots \rightarrow \alpha
			\end{multline*}
			
		\end{multicols}
	\end{frame}
	

	\begin{frame}
		\frametitle{Стохастическая КС-грамматика (СКСГ)}
		\begin{equation*}
			G = <V_N, V_T, S, R>
		\end{equation*}
		
		$V_N$, $V_T$, $S$ --- имеют тот же смысл.
		
		$R = \cup_{i=1}^k R_i$, где $R_i$ --- множество правил вида:
		\begin{equation*}
			r_{ij} : A_i \xrightarrow{p_{ij}} \beta_{ij}\quad (j = 1,2,\ldots,n_i)
		\end{equation*}
		$\beta_{ij} \in (V_N \cup V_T)^*$, $p_{ij}$ --- вероятность применения правила
		\begin{equation*}
			0 < p_{ij} \leq 1\quad \sum_{j=1}^{n_j} p_{ij} = 1
		\end{equation*}
		
		Вероятность дерева вывода: $p(d) = \prod_{p_{ij} \in d} p_{ij}$.
		
		Вероятность слова: $p(\alpha) = \sum_{d} p(d)$.
		
		Свойство согласованности: $\sum_{\left| \alpha \right| < \infty} p(\alpha) = 1$.
	\end{frame}

	\begin{frame}
		\frametitle{Пример вывода слова в СКСГ}
		\framesubtitle{Язык арифметических выражений c a,b без скобок}
		\begin{multicols}{2}
			{\small
				\begin{equation*}
				\begin{split}
				&V_N = \{E, M, T\},\;V_T = \{+, -, *, /, a, b\}\\
				&S = E\\
				&r_{11} : E \xrightarrow{1/4} M + E \\
				&r_{12} : E \xrightarrow{1/4} M - E \\
				&r_{13} : E \xrightarrow{1/2} M \\
				&r_{21} : M \xrightarrow{1/3} T * M \\
				&r_{22} : M \xrightarrow{1/3} T / M \\
				&r_{23} : M \xrightarrow{1/3} T \\
				&r_{31} : T \xrightarrow{1/4} a \\
				&r_{32} : T \xrightarrow{3/4} b
				\end{split}
				\end{equation*}
			}
			
			$$
			\alpha = a + b * a - b
			$$
			
			{\footnotesize
				\renewcommand{\qtreepadding}{1.8pt}
				\Tree [ .E [ .M [ .T a ] ] + [ .E [ .M [ .T b ] * [ .M [ .T a ] ] ] - [ .E [ .M [ .T b ] ] ] ] ]
			}
			
		\end{multicols}
	\end{frame}
	

	\begin{frame}
		\frametitle{Отношение следования на множестве нетерминалов}
		$A_i \rightarrow A_j$, если существует правило $A_i \xrightarrow{p_{ij}} \alpha A_j \beta$, где $\alpha, \beta \in (V_N \cup V_T)^*$.
		\vspace{5pt}
		
		$(\rightarrow_*)$ --- рефлексивное транзитивное замыкание $(\rightarrow)$.
		\vspace{5pt}
		
		$A_i \leftrightarrow_* A_j$, если $A_i \rightarrow_* A_j$ и $A_j \rightarrow_* A_i$.
		\vspace{5pt}
		
		Разобьём нетерминалы на классы по отношению $(\leftrightarrow_*)$. Тогда:
		\begin{equation*}
			\forall A_i, A_j \in K\quad A_i \rightarrow_* A_j
		\end{equation*}
		Не рассматриваем особые классы: $K = \{ A \}$, $A \not \rightarrow A$.
		\vspace{5pt}
		
		$K_1 \prec K_2$, если $\exists A_1 \in K_1, A_2 \in K_2 : A_1 \rightarrow A_2$.
		\vspace{5pt}
		
		$(\prec_*)$ --- рефлексивное транзитивное замыкание $(\prec)$.
		
		Особые классы: $K_i = \{A_j\}$, не $A_j \rightarrow A_j$. Предполагаем, что их нет.
	\end{frame}
	
	\begin{frame}
		\frametitle{Производящие функции}
		\begin{equation*}
			F_i(s_1, s_2, \ldots, s_k) = \sum_{j=1}^{n_i} p_{ij} s_1^{l_1} s_2^{l_2} \ldots s_k^{l_k},
		\end{equation*}
		где $l_s$ --- число нетерминалов $A_s$ в правой части правила $A_i \xrightarrow{p_{ij}} \beta{ij}$.
		\vspace{15pt}
		
		Производящие функции с параметром $t$:
		\begin{equation*}
			F_i(t, \bar{s}) = \left\{
			\begin{split}
				&F_i(\bar{s}),  & &\text{при}\;\; t = 1 \\
				&F_i(\bar{F}(t-1, \bar{s})), & &\text{при}\;\; t = 2,3,\ldots
			\end{split}
			\right.
		\end{equation*}
	\end{frame}
	
	\begin{frame}
		\frametitle{Первые моменты}
		\begin{equation*}
			\left. \frac{\partial F_i(s_1, s_2, \ldots, s_k)}{\partial s_j} \right|_{s_1 = s_2 = \ldots = s_k = 1} = a^i_j
		\end{equation*}
		$a^i_j$ --- математическое ожидание числа нетерминалов $A_j$ при однократном применении случайного правила к $A_i$.
		\vspace{10pt}
		
		Матрица $A = (a^i_j)$ --- матрица первых моментов.
		\vspace{10pt}
		
		$r$ --- максимальное по модулю собственное число матрица $A$ (перронов корень). Будем рассматривать критический случай $r = 1$.
	\end{frame}
	
	\begin{frame}
		\frametitle{Пример: матрица первых моментов}
		\framesubtitle{Язык арифметических выражений c a,b без скобок}
		\begin{multicols}{2}
			{\small
				\begin{equation*}
				\begin{split}
				&V_N = \{E, M, T\},\;V_T = \{+, -, *, /, a, b\}\\
				&S = E\\
				&r_{11} : E \xrightarrow{1/4} M + E \\
				&r_{12} : E \xrightarrow{1/4} M - E \\
				&r_{13} : E \xrightarrow{1/2} M \\
				&r_{21} : M \xrightarrow{1/3} T * M \\
				&r_{22} : M \xrightarrow{1/3} T / M \\
				&r_{23} : M \xrightarrow{1/3} T \\
				&r_{31} : T \xrightarrow{1/4} a \\
				&r_{32} : T \xrightarrow{3/4} b
				\end{split}
				\end{equation*}
			}
	
		$$
		A =
			\begin{pmatrix}
				\frac{1}{2} & 1 & 0 \\
				0 & \frac{2}{3} & 1 \\
				0 & 0 & 0
			\end{pmatrix}
		$$
		
		Классы: $\{E\}$, $\{M\}$, $\{T\}$.
		
		$\{T\}$ --- особый класс!
		
		Как избавиться:
		
		\begin{equation*}
		\begin{split}
		&r_{21} : M \xrightarrow{1/12} a * M\\
		&r_{22} : M \xrightarrow{3/12} b * M\\
		&\cdots
		\end{split}
		\end{equation*}
			
		\end{multicols}
	\end{frame}
	
	\begin{frame}
		\frametitle{Матрица первых моментов}

		Упорядочим классы:				
		\begin{equation*}
			K_i \prec_* K_j \Rightarrow i < j
		\end{equation*}
		
		Матрица первых моментов примет вид:
		\begin{equation*}
			A =
			\begin{pmatrix}
				A_{11} & A_{12} & A_{13} & \cdots & A_{1,m-1} & A_{1,m} \\
				0      & A_{22} & A_{23} & \cdots & A_{2,m-1} & A_{2,m} \\
				0      & 0      & A_{33} & \cdots & A_{3,m-1} & A_{3,m} \\
				\vdots & \vdots & \vdots & \ddots & \vdots & \vdots \\
				0      & 0      & 0      & \cdots & A_{m-1,m-1} & A_{m-1,m} \\
				0      & 0      & 0      & \cdots & 0      & A_{mm}
			\end{pmatrix}
		\end{equation*}
		
		Блоки $A_{ii} (i = 1,2,\ldots,m)$ неразложимы, имеют перроновы корни $r_i$.
		
		Перронов корень всей матрицы $r = \max\{r_i\}$.
	\end{frame}
	
	\begin{frame}
		\frametitle{Матрица первых моментов (2)}
		
		$s_{ij}^*$ -- максимальное число классов с перроновым корнем $r$ в цепочке $K_i \prec \ldots \prec K_j$.
		
		Дополнительно упорядочим классы по возрастанию $s_{1i}^*$. При равных $s_{1i}^*$, сначала критические.
		
		\begin{equation*}
		A =
		\begin{pmatrix}
		B_{11} & B_{12} & \cdots & B_{1,w-1} & B_{1,w} \\
		0      & B_{22} & \cdots & B_{2,w-1} & B_{2,w} \\
		\vdots & \vdots & \ddots & \vdots & \vdots \\
		0      & 0      & \cdots & 0      & B_{ww}
		\end{pmatrix}
		\end{equation*}

		$\mathcal{M}_1$ : $s_{1i}^* \leq 1$. $\mathcal{M}_k$ : $s_{1i}^* = k$ ($k > 1$).
	\end{frame}
	
	\begin{frame}
		\frametitle{Результат: степень матрицы первых моментов}
		Пусть $B_{ij}^{(t)}$ --- блок матрицы $A^t$ на позиции блока $B_{ij}$ матрицы $A$.
		
		При $t \rightarrow \infty$
		\begin{equation*}
			B_{ij}^{(t)} \sim H_{ij} \cdot r^t \cdot t^{s_{ij} - 1}
		\end{equation*}
		\begin{flushright}
			\textit{Л. П. Жильцова}
		\end{flushright}
	\end{frame}

	\begin{frame}
		\frametitle{Вероятности продолжения: цепочка}
		$Q_i(t)$ --- вероятность деревьев вывода высоты $\geq t$ с корнем $A_i$ (вероятности продолжения).
		
		$P_i(t)$ --- вероятность деревьев высоты $t$.
		
		При $t \rightarrow \infty$
		\begin{equation*}
			\begin{split}
				&Q_i(t) = c_n \cdot U_i \cdot t^{-\left(\frac{1}{2}\right)^{s_{nm} - 1}} \cdot (1 + o(1)) \\
				&P_i(t) = \tilde{c}_n \cdot U_i \cdot t^{-1 -\left(\frac{1}{2}\right)^{s_{nm} - 1}} \cdot (1 + o(1))
			\end{split},
		\end{equation*}
		где $A_i \in K_n$, $U_i, c_n, \tilde{c}_n > 0$.
		
		\begin{picture}(1000, 45)
			\put(20, 20){\circle*{20}}
			\put(30, 20){\vector(1,0){20}}
			\put(60, 20){\circle{20}}
			\put(70, 20){\vector(1,0){20}}
			\put(100, 20){\circle*{20}}
			\put(110, 20){\vector(1,0){20}}
			\put(140, 20){\circle{20}}
			\put(150, 20){\vector(1,0){20}}
			\put(180, 20){\circle*{20}}
			\put(190, 20){\vector(1,0){20}}
			\put(220, 20){\circle*{20}}
			\put(230, 20){\vector(1,0){20}}
			\put(260, 20){\circle{20}}
			\put(270, 20){\vector(1,0){20}}
			\put(300, 20){\circle{20}}
			\put(140, -10){\vector(0,1){20}}
			\put(142, -5){$n$}
			\put(140, -10){\vector(1,0){170}}
			\put(200, -2){$s_{nm} = 2$}
		\end{picture}
	\end{frame}
	
	\begin{frame}
		\frametitle{Вероятности продолжения}
		При $t \rightarrow \infty$
		\begin{equation*}
		\begin{split}
		&Q_i(t) \sim c_i \cdot t^{-\left(\frac{1}{2}\right)^{q_l - 1}} \\
		&P_i(t) \sim c_i \cdot t^{-1 -\left(\frac{1}{2}\right)^{q_l - 1}}
		\end{split},
		\end{equation*}
		где $A_i \in K_l$, $c_i > 0$, $q_l = s_{lm}^*$.
	\end{frame}
	
	\begin{frame}
		\frametitle{Число применений правила: цепочка}
		$M_{ij}(t)$ --- математическое ожидание числа применений правила $r_{ij}$ в деревьях высоты $t$
		
		При $t \rightarrow \infty$
		\begin{equation*}
			M_{ij}(t) = d_i \cdot p_{ij} \cdot t^{\left(\frac{1}{2}\right)^{s_{nm} - 1 - \xi}} \cdot (1 + o(1)),
		\end{equation*}
		где $A_i \in K_n$, $d_i > 0$ и
		\begin{equation*}
			\xi = \left\{
			\begin{split}
				&1,\quad K_n \text{ --- критический} \\
				&0,\quad K_n \text{ --- докритический}
			\end{split}
			\right.
		\end{equation*}
		
		\begin{picture}(1000, 45)
			\put(20, 20){\circle*{20}}
			\put(30, 20){\vector(1,0){20}}
			\put(60, 20){\circle{20}}
			\put(70, 20){\vector(1,0){20}}
			\put(100, 20){\circle*{20}}
			\put(110, 20){\vector(1,0){20}}
			\put(140, 20){\circle{20}}
			\put(150, 20){\vector(1,0){20}}
			\put(180, 20){\circle*{20}}
			\put(190, 20){\vector(1,0){20}}
			\put(220, 20){\circle*{20}}
			\put(230, 20){\vector(1,0){20}}
			\put(260, 20){\circle{20}}
			\put(270, 20){\vector(1,0){20}}
			\put(300, 20){\circle{20}}
			\put(140, -10){\vector(0,1){20}}
			\put(142, -5){$n$}
			\put(140, -10){\vector(1,0){170}}
			\put(200, -2){$s_{nm} = 2$}
		\end{picture}
	\end{frame}
	
	\begin{frame}
		\frametitle{Число применений правила}
		$M_{ij}(t)$ --- математическое ожидание числа применений правила $r_{ij}$ в деревьях высоты $t$
		
		При $t \rightarrow \infty$
		\begin{equation*}
		M_{ij}(t) = d_i \cdot p_{ij} \cdot t^{\left(\frac{1}{2}\right)^{s_{nm} - 1 - \xi}} \cdot (1 + o(1)),
		\end{equation*}
		где $A_i \in K_n$, $d_i > 0$ и
		\begin{equation*}
		\xi = \left\{
		\begin{split}
		&1,\quad K_n \text{ --- критический} \\
		&0,\quad K_n \text{ --- докритический}
		\end{split}
		\right.
		\end{equation*}
	\end{frame}
	
	% Список литературы
	\begin{frame}
		\frametitle{Использованная литература}
		{\footnotesize
		\begin{itemize}
			\item \textbf{Севастьянов Б. А.} Ветвящиеся процессы. --- M.: Наука, 1971 --- 436 с.
			\item \textbf{Фу К.} Структурные методы в распознавании образов. М.: Мир, 1977
			\item \textbf{Гантмахер Ф. Р.} Теория матриц. --- 5-е изд., --- М.: ФИЗМАТЛИТ, 2010
			\item \textbf{Ахо А., Ульман Дж.} Теория синтаксического анализа, перевода и компиляции. Том 1. М.: Мир, 1978
			\item \textbf{Жильцова Л. П.} О матрице первых моментов разложимой стохастической КС-грамматики. УЧЁНЫЕ ЗАПИСКИ КАЗАНСКОГО ГОСУДАРСТВЕННОГО УНИВЕРСИТЕТА, Том 151, кн. 2, 2009
			\item \textbf{Жильцова Л. П.} Закономерности применения правил грамматики в выводах слов стохастического контекстно-свободного языка // Математические вопросы кибернетики. Выр. 9. М.: Наука, 2000. С. 100-126.
			\item \textbf{Борисов А. Е.} Закономерности в словах стохастических контекстно-свободных языков, порождённых грамматиками с двумя классами нетерминальных символов. Вопросы экономного кодирования.
		\end{itemize}
		}
	\end{frame}
	
	\begin{frame}
		\begin{center}
			{\Huge \textbf{Спасибо за внимание!}}
		\end{center}
	\end{frame}
\end{document}