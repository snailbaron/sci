\documentclass[12pt,russian]{article}
%\documentstyle{article}
\usepackage{amsmath,amssymb,latexsym}
\usepackage[utf8]{inputenc}
\usepackage[russian]{babel}
\usepackage[T2A]{fontenc}
%\setlength{\topmargin}{15mm} \setlength{\headheight}{0pt}
%\setlength{\headsep}{0pt} \setlength{\topskip}{0pt}


\topmargin=13mm           % веpхнее поле свеpх одного дюйма (до колонтитула)
\textwidth=160mm           % шиpина текста
\oddsidemargin=0.6cm % отступ от левого однодюймового поля
\textheight=235mm % высота текста, pассчитываемая так:
\headsep=0mm % pасстояние от колонтитула (хоть и пустого) до текста
\hoffset=0mm % сдвиг всей стpаницы впpаво
\voffset=-10mm % сдвиг всей стpаницы вниз

\begin{document}

\section{Моменты}
{\sloppy 
Пусть $\Xi=(\xi_1,\ldots,\xi_k) -$ случайный вектор,
$\alpha^*=(\alpha_1, \ldots,\alpha_k )$ --- фиксированный вектор
с целочисленными неотрицательными компонентами и
$\alpha=\alpha_1+\ldots+\alpha_k.$ Обозначим
$$
\Xi^{[\alpha^*]}=\xi_1^{[\alpha_1]} \ldots \xi_n^{[\alpha_k]},
$$
где $x^{[a]}=x(x-1)\ldots (x-a+1).$
Математическое ожидание $M\Xi^{[\alpha^*]}$ будем называть $\alpha^*$-{\em моментом\/} $\Xi$ [Севаст].

Пусть $A_i$ - некоторый нетерминальный символ грамматики $G$. Через $L_i$ обозначим язык, порожденный грамматикой $G_i,$ которая получается из $G$ заменой аксиомы на $A_i.$ Будем считать, что аксиомой исходной грамматики является нетерминал $A_1$ и $L=L_1$ для исходного языка $L.$ 
Через $D_i$ обозначим множество деревьев вывода для слов из $L_i.$ 

Пусть $x^i_j(t) $  --- число нетерминалов $A_j$ в дереве вывода из
$D_i$ на ярусе $t.$ Через $M^i_{\alpha^*}(t)$ обозначим
$\alpha^*-$момент вектора
$X^i(t)=(x^i_1(t),\ldots,x^i_k(t)).$

Примем специальные обозначения для моментов первых
четырех порядков.
Факториальные моменты первого порядка будем обозначать через $a^i_j(t).$
Для факториальных моментов второго порядка введем обозначения
$b^i_{jn}(t).$ Таким образом, $b^i_{jj}(t)=Mx^i_j(t)(x^i_j(t)-1)$ и
$b^i_{jn}(t)=Mx^i_j(t)x^i_n(t)$ при $j\neq n.$
Для факториальных моментов третьего и четвертого порядков введем обозначения
$c^i_{jnq}(t)$ и $f^i_{jnql}(t)$ соответственно.

Нетрудно заметить, что $a^i_j(1)$ -- элементы матрицы первых моментов,
для которых мы ввели ранее обозначения $a^i_j.$

Будем также применять далее обозначения
$b^i_{jn}$ для $b^i_{jn}(1).$

Нас интересуют оценки для первых четырех моментов.

Свойства первых моментов исследованы в [Казань, матрица перв мом], так как $a^i_j(t)$ - элемент матрицы $A^t$.  

Для вторых моментов известна следующая формула из [Севаст]:
%$$
\begin{equation}
b^i_{jn}(t)=
\sum_{\tau=1}^t \sum_{l,m,s}a^i_l(t-\tau) b^l_{ms}
a^m_j(\tau-1) a^s_n(\tau-1).   
%$$
\label{25}
\end{equation}
Пусть $a^i_l$ принадлежит подматрице $A_{h_i h_l},$ $a^m_j$ --- подматрице $A_{h_m h_j},$ и $a^s_n$ --- подматрице $A_{h_s h_n}.$
Подставим в (\ref{25}) представление для первых моментов:
$$
b^i_{jn}(t)=\sum_{\tau=1}^t \sum_{l,m,s} c_{il} \cdot \left(t-\tau\right)^{\delta_1}\cdot \left(1+O\left(\frac{1}{t-\tau}\right)\right)\cdot r^{t-\tau}\cdot b^l_{ms} \cdot c_{mj}\cdot\left( \tau-1\right)^{\delta_2} \cdot r^{\tau-1}\times
$$
$$
c_{sn}\cdot\left(\tau-1\right)^{\delta_3}\cdot r^{\tau-1}\cdot \left(1+O\left(\frac{1}{\tau}\right)\right).
$$
Здесь $\delta_1=s_{h_i h_l}-1,$ $\delta_2=s_{h_m h_j}-1,$ и $\delta_3=s_{h_s h_n}-1,$ и $c_{il},$ $c_{mj}$ и  $c_{sn}$ ---
коэффициенты в соответствующих элементах матрицы $A^t.$ 


Проведем несложные преобразования в полученном равенстве:
$$
b^i_{jn}(t)= r^t 
\sum_{l}   c_{il}\,t^{\delta_1}
\sum_{\tau=1}^t\left(1-\frac{\tau}{t} \right)^{\delta_1} \left( 1+O \left(\frac{1}{t-\tau}\right)\right)
\sum_{m,s} b^l_{ms}c_{mj} c_{sn} \cdot \tau^{\delta_2+\delta_3} \times
$$
$$
\left(1-\frac{1}{\tau}\right)^{\delta_2+\delta_3} r^{\tau-2} \cdot \left( 1+O \left(\frac{1}{\tau} \right) \right).
$$
Ряд 
$$
\sum_{\tau=1}^{\infty}\left(1-\frac{\tau}{t} \right)^{\delta_1} \left( 1+O \left(\frac{1}{t-\tau}\right)\right)
\sum_{m,s} b^l_{ms}c_{mj} c_{sn} \cdot \tau^{\delta_2} 
\left(1-\frac{1}{\tau}\right)^{\delta_2+\delta_3}  r^{\tau-2}\cdot \left( 1+O \left(\frac{1}{\tau} \right) \right)
$$
сходится. Обозначим его сумму через $g^i_{jn}(l).$ 
Отметим, что $g^i_{jn}(l)>0$ в тех случаях, когда существуют $m$ и $s$ такие, что $b^l_{ms}>0,$ 
$$
K_{h_i}\prec_*K_{h_l}\prec_*K_{h_m}\prec_*K_{h_j} \, \, \, \mbox{и} \, \, \, K_{h_i}\prec_*K_{h_l}\prec_*K_{h_s}\prec_*K_{h_n}.
$$
Условие $b^l_{ms}>0$ выполняется тогда и только тогда, когда в грамматике существует правило с нетерминалом $A_l$ в левой части, содержащее в правой части оба нетерминала $A_m$ и $A_s.$
 
При $t\rightarrow \infty$ 
$$
b^i_{jn}(t)=\sum_{l} g^i_{jn}(l) \cdot t^{\delta_1} r^t (1+o(1)),
$$
где суммирование ведется по тем $l,$ для которых $g^i_{jn}(l)>0$.

Очевидно, определяющими в этой сумме являются слагаемые с теми значениями $l,$ для которых ${\delta_1}$ имеет наибольшее значение. Обозначим его через $\delta^i_{jn}.$ Поэтому формулу для $b^i_{jn}(t)$ можно записать в следующем виде:
$$
b^i_{jn}(t)= g^i_{jn} t^{\delta^i_{jn}} r^t  \cdot (1+o(1)).
$$
Здесь $g^i_{jn}=\sum_{l} g^i_{jn}(l),$ где суммирование ведется по значениям $l,$ удовлетворяющим перечисленным выше условиям. 
Так как $l\leq j$ и $l \le n,$ то $\delta^i_{jn} \leq \max \{s_{h_ih_j}~-~1,s_{h_ih_n}~-~1\}.$ Поэтому $b^i_{jn}(t) \le O\left(~ a^i_j(t)~+~a^i_n(t)~\right).$

Используя результаты из [Севаст], запишем формулу для третьего момента:
$$
c^i_{jnq}(t)=\sum_{\tau=1}^t \sum_l a^i_l(t-\tau)\cdot z^l_{jnq}(\tau-1).
$$
В этой формуле $z^l_{jnq}(\tau-1)$ состоит из конечного числа
слагаемых двух типов. Слагаемые первого типа имеют вид: $C
a^s_q(\tau~-~1) \cdot a^m_n(\tau~-~1) \cdot a^l_j(\tau~-~1) $ для некоторых
$s,m,l,$ где $C$ --- некоторая константа, зависящая от слагаемого;
слагаемые второго типа имеют вид: $C
a^l_j(\tau-1) \cdot b^m_{nq}(\tau-1)$ для некоторых $l,m$ и константы
$C.$

Поэтому вычисление $c^i_{jnq}(t)$ сводится к вычислению конечного числа
сумм вида
$$
S_1(t)= \sum_{\tau=1}^t a^i_l(t-\tau)\cdot a^l_j(\tau-1)\cdot a^s_q(\tau-1)
\cdot a^m_n(\tau-1)
$$
и вида
$$
S_2(t)= \sum_{\tau=1}^t a^i_l(t-\tau)\cdot a^s_j(\tau-1)\cdot b^m_{nq}(\tau-1)
$$
для некоторых значений $l,m,s.$

Оценим $S_1(t)$  и $S_2(t)$, используя оценки
$a^i_j(t)=O(t^{s_{ij}-1} r^t ),\,$ $b^i_{jl}(t) \le O\left(a^i_j(t)+a^i_l(t)\right).$ Применяя очевидное неравенство $s_{ij} \le w,$ где $w$ --число групп, получим, что $S_1(t)\leq O(t^{w-1}r^t)$ и $S_2(t)\leq O(t^{w-1}r^t)$.
Поэтому 
$$
c^i_{jnq}(t)\leq O\left(t^{w-1}r^t \right).
$$

Аналогичный результат может быть получен и для четвертого момента.
Для него известна формула [Севаст]:
$$
f^i_{j_1 j_2 j_3 j_4}(t)=
\sum_{\tau=1}^t \sum_{j=1}^k a^i_j(t-\tau) z^j_{j_1 j_2 j_3 j_4}(\tau-1),
$$
где
$z^j_{j_1 j_2 j_3 j_4}(\tau-1)$ состоит из конечного числа слагаемых
следующих четырех видов:
$$
S_1(\tau-1)=  a^{i_1}_{j_1}(\tau-1)\cdot a^{i_2}_{j_2}(\tau-1)
\cdot a^{i_3}_{j_3}(\tau-1)\cdot a^{i_4}_{j_4}(\tau-1);
$$
$$
S_2(\tau-1)=  a^{i_1}_{j_1}(\tau-1)\cdot a^{i_2}_{j_2}(\tau-1)
\cdot b^{i_3}_{j_3 j_4}(\tau-1);
$$
$$
S_3(\tau-1)= b^{i_1}_{j_1 j_2}(\tau-1)
\cdot b^{i_2}_{j_3 j_4}(\tau-1);
$$
$$
S_4(\tau-1)=  a^{i_1}_{j_1}(\tau-1)\cdot c^{i_2}_{j_2 j_3 j_4}(\tau-1)
$$
для некоторых значений $i_1, i_2, i_3, i_4.$

Используя оценки для первых трех моментов и проведя элементарные
преобразования, получаем оценку по порядку для четвертого момента:
$$
f^i_{j_1 j_2 j_3 j_4}(t)\leq O\left(t^{w-1}r^t \right).
$$
\end{document}
}